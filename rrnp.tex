\documentclass[fleqn]{amsart}
\usepackage{amssymb,amsmath,amsthm,hyperref}

\def\Var{\mathop{\rm Var}}
\def\Cov{\mathop{\rm Cov}}
\newcommand{\RR}{\mathbb{R}}

\title{The Risks of Risk-Neutral Pricing\\
{\small (Things your mother never told you about Risk-Neutral Pricing.)}}
\author{Keith A. Lewis}
\email{kal@kalx.net}
\address{KALX, LLC \tt{\url{http://kalx.net}}}

\begin{document}
\maketitle

\begin{section}{Risk-Neutral Pricing}

\begin{itemize}

\item Assumes no arbitrage and perfect liquidity.

\item No arbitrage? Time to go home. That's where you make real money.

\item Perfect liquidity? Often the most important consideration.

\item ``Price'' is a slippery concept.
\begin{itemize}
	\item Size matters.
	\item Bid/Ask spread.
	\item Market/limit order mechanics.
\end{itemize}

\end{itemize}

\end{section}

\begin{section}{Foundations of RNP}

\begin{itemize}

\item If an option has a perfect hedge, its value is the
cost of the hedge.

\item This cost can be computed as the expected value of
the discounted option payoff.

\item Hedging is delta hedging:  ratio of option value change to
underlying change.

\end{itemize}

\end{section}

\begin{section}{Problems with RNP}

\begin{itemize}

\item If an option does not have a perfect hedge, the theory
is silent.

\item The FTAP implies risk-neutral measures exists, but often
they are not unique.

\item All hedging in the real world is discrete time hedging, hence
imperfect.

\item Geared towards single trades.
\begin{itemize}
\item Ignores portfolio effect.
\item Credit risk is non-linear.
\end{itemize}

\end{itemize}

\end{section}

\begin{section}{Observations}

\begin{itemize}

\item There is only one model -- a stochastic process -- the whole
game is how to parameterize it.

\item Arbitrage {\em must} be defined relative to a particular model.

\item Hedge Funds don't hedge.

\end{itemize}

\end{section}

\begin{section}{1-2-3 Example{...}}

Single period binomial model.

\begin{itemize}

\item Stock price is $1$, goes to $3$ with probability $0.9$, stays at
$1$ with probability $0.1$.

\item Bond price is $1$, goes to $2$ over period.

\item What is value of call option with strike $2$?

\end{itemize}

\end{section}

\begin{section}{{...}Answer}

\begin{itemize}

\item Wrong answer: $0.45 = (1/2)(0\times 0.1 + 1\times 0.9)$. (Note $1 \not=
(1/2)(1\times 0.1 + 3\times 0.9)$.

\item ``Right'' answer: $0.25 = (1/2)(0\times 0.5 + 1\times 0.5$. (Note $1 =
(1/2)(1\times 0.5 + 3\times 0.5)$.

\item Paying $0.25$ to make $(1/2)0.9 = 0.45$ $90\%$ of the time sounds like
a good trade!

\item The theory of risk-neutral pricing is risk blind.

\end{itemize}

\end{section}

\begin{section}{One Period Binomial}

\begin{itemize}

\item Stock price is $s$, goes to $S_+$ or $S_-$.

\item Bond price is $1$, goes to $R$ over period.

\item Option has payoff $f(S_\pm) = F_\pm$.

\end{itemize}

Replicate option by finding $m$ and $n$ so that

\begin{eqnarray*}
	mR + nS_- &=& F_-\\
	mR + nS_+ &=& F_+\\
\end{eqnarray*}

\end{section}

\begin{section}{{...}OPB}

Delta hedge, $n$, and option value, $v$, are

\begin{eqnarray*}
	n &=& \frac{F_+ - F_-}{S_+ - S_-} \\
	v &=& m + ns \\
	  &=& \frac{1}{R}\left[\frac{S_+ - Rs}{S_+ - S_-} F_-
		+ \frac{Rs - S_-}{S_+ - S_-} F_+\right] \\
\end{eqnarray*}

Note $n = \partial v/\partial s$.

\end{section}

\begin{section}{Wrong Parameterization}

\begin{itemize}

\item This parameterization is not standard. Given an up move, $u$,
and a down move, $d$, define $S_+ = su$ and $S_- = sd$, so
$$
	v = \frac{1}{R}\left[\frac{u - R}{u - d} f(sd)
		+ \frac{R - d}{u - d} f(su)\right].
$$

\item Oops, $n \not= \partial v/\partial s$ in general.

\item However, $ns = (\partial/\partial R)(Rv)$ for {\em both}
parameterizations.

\end{itemize}

\end{section}

\begin{section}{Normal Facts}

For any functions, $g$, $h$, and $k$,

\begin{itemize}

\item if $N$ is normal, then $E[e^N g(N)] = E[e^N] E[g(N + \Var N)]$.

\item if $S$ is lognormal, then $E[S h(S)] = E[S] E[h(Se^{\Var\log S})]$.

\item if $S$, $S_1$,... are jointly lognormal, then
$E[S k(S_1,...)] = E[S] E[k(Se^{\gamma_1},...)]$, where
$\gamma_j = \Cov(\log S, \log S_j)$.

\end{itemize}

\end{section}


\begin{section}{Black-Scholes Pricing}

Let's try to use this with Black-Scholes pricing, $v = (1/R) E[f(S)]$,
where $S = Rse^{\sigma B_t - \sigma^2 t/2}$.

\begin{eqnarray*}
\partial v/\partial s &=& (1/R) E[f'(S) S/s]\\
	&=& E[f'(Se^{\sigma^2 t})]\\
\partial(Rv)/\partial R &=& E[f'(S) S/R]\\
	&=& s E[f'(Se^{\sigma^2 t})]\\
\end{eqnarray*}

If $f(x) = (x - k)^+$, this shows delta is $N(d_1)$.

\end{section}

\begin{section}{Delta Skelter}

\begin{itemize}

\item
For a contract with payoff $\log(S)$,
$\partial (1/R) E[\log(S)]/\partial s = 1/Rs$.

\item
For a contract with payoff $\log(S/s)$,
$\partial (1/R) E[\log(S/s)]/\partial s
= (1/R) E[\log R + \sigma B_t - \sigma^2t/2]/\partial s = 0$.

\item However, $\partial E[\log(S/s)]/\partial R = 1/R$ gets it right.

\end{itemize}

\end{section}

\begin{section}{HiP Delta}

Let $S = RsM$, where $E[M] = 1$ and $M$ does not depend on
$R$, or $s$ and $C(k) = E[(S  - k)^+]/R$. From before,

\begin{eqnarray*}
ns &=& E[1(S > k) S/R]\\
	&=& E[(S - k)1(S > k)]/R + k P(S > k)/R\\
	&=& C(k) + k C'(k)\\
\end{eqnarray*}

You can compute dollar delta with your hands in your pocket. 

\end{section}

\begin{section}{CAPM}

Bond is $1$, stock is $s$ at beginning of period. Bond return
is $R$ and stock goes to $S$. Option pays $F$ at end of period.
Minimize $E[(mR + nS - F)^2]$ over $m$, $n$.

\begin{eqnarray*}
n &=& \Cov(F,S)/\Var(S)\\
m + ns &=& E[F]/R - n (E[S]/R - s)\\
\end{eqnarray*}

Assuming $S$ and $F$ do not depend on $s$, we have
$\partial(Rv)/\partial R = ns$, where $v = m + ns$.

\end{section}

\begin{section}{Cones and Arbitrage}

\begin{itemize}

\item A {\em cone} is a subset of a vector space that is closed under
vector addition and multiplication by a positive scalar.

\item Cones are convex, but not vice versa.

\item The set of arbitrage portfolios is a cone.

\end{itemize}

\end{section}

\begin{section}{One Period General}

\begin{itemize}

\item Model: market with $m$ assets.
\begin{itemize}

\item $x\in\RR^m$ - prices at beginning of period.

\item $\Omega$ - set of outcomes over the period.

\item $X\colon\Omega\to\RR^m$ - prices at end of period.

\end{itemize}

\item Arbitrage means there exists $\xi\in\RR^m$ with $\xi\cdot x < 0$
and $\xi\cdot X(\omega)\ge0$ for all $\omega\in\Omega$.

\end{itemize}

\end{section}

\begin{section}{FTAP}

\medskip

{\bf Theorem:} {\rm (Fundamental Theorem of Asset Pricing)} {\it No arbitrage
if and only if $x$ belongs to the smallest closed cone containing
$X(\Omega)$.}

\medskip

\begin{itemize}

\item This implies there is a positive measure, $\pi$, on $\Omega$
with $x = \int_\Omega X\,d\pi$.

\item If a bond with return $R$ exists, this is equivalent to $x$
belonging to the smallest closed convex set containing $X(\Omega)/R$.

\end{itemize}

\end{section}

\begin{section}{Application}

\begin{itemize}

\item Rates are zero, stock is 100. Can go to 90, 100, or 110.  What are
the no arbitrage constraints on a call option with strike 100?

\item Model: $x = (1, 100, v)$, $\Omega = \{90, 100, 110\}$,
$X(\omega) = (1, \omega, (\omega - 100)^+)$. 

\item No arbitrage if and only if $0\le v\le 5$.

\end{itemize}

\end{section}

\begin{section}{FX Puzzle}

Suppose USD/EUR volatility is 9\%, EUR/GBP volatility is
9\%, and GBP/USD volatility is 20\%. These volatilities
do not form the sides of a triangle, so arbitrage exists
in the Black-Scholes constant volatility model.
What is the arbitrage?

\end{section}

\end{document}
