\documentclass[fleqn]{amsart}
\usepackage{amssymb,amsmath,amsthm,hyperref}
\def\Var{\mathop{\rm Var}}
\def\Cov{\mathop{\rm Cov}}
\def\ker{\mathop{\rm ker}}
\def\ran{\mathop{\rm ran}}
\newcommand{\R}{\mathbb{R}}
\newcommand{\F}{\mathcal{F}}
\newcommand{\G}{\mathcal{G}}
\newcommand{\B}{\mathcal{B}}
\newcommand{\onehalf}{\frac{1}{2}}

\newtheorem{definition}{Definition}[section]
%\newtheorem{theorem}[definition]{Theorem}
\newtheorem{theorem}{Theorem}[section]
\newtheorem{corollary}[theorem]{Corollary}
\newtheorem{lemma}[theorem]{Lemma}
\newtheorem{example}{Example}
\newtheorem{exercise}{Exercise}

\title{Pricing Derivative Securities}
\author{Keith A. Lewis}
\email{kal@kalx.net}
\address{KALX, LLC \tt{\url{http://kalx.net}}}

\begin{document}
\maketitle

\begin{section}{The Game Plan}

The (so-called) risk-neutral approach to pricing derivative securities
(without cash flows) consists of specifying a vector-valued stochastic
process $(X_t)_{0\le t\le T}$ for market instruments,
and finding a scalar stochastic
process $(\Pi_t)_{0\le t\le T}$ of positive {\em deflators} such that
$(X_t\Pi_t)_{0\le t\le T}$ is a martingale. If there is a self-financing
strategy $(\Xi_t)_{0\le t\le T}$ such that $F_T = \int_0^T \Xi_s\cdot
\,dX_s$, then the cost of setting up the initial hedge is $\Xi_0\cdot
X_0 = E[F_T\Pi_T]$. To find $\Xi_0$ we can compute
the derivative with respect to $X_0$ of the right-hand side.
This is the Black-Scholes/Merton delta hedge.

\end{section}

\begin{section}{Ito Processes}
This section contains the bare bones facts you need to know about
Ito processes. The {\em stochastic differential equation}
$dX = \mu\,dt + \sigma\,dB$ with $X_0 = x_0$ is the same
as saying $X_t = x_0 + \int_0^t \mu\,ds + \int_0^t \sigma\,dB_s$
where the second integral is an {\em Ito Integral}. Such an
$(X_t)_{t\ge0}$ is called an {\em Ito Process}.

The {\em Ito Lemma} states that if $(X_t)_{t\ge0}$ is an Ito process and
$f\colon\R\times\R\to R$ is sufficiently smooth, then $Y_t = f(t, X_t)$
is also an Ito process and $dY_t = (f_t + \mu f_x + \onehalf\sigma^2
f_{xx})\,dt + \sigma f_x\,dB$.	Here $f_t = \partial f/\partial t$,
$f_{xx} = \partial^2 f/\partial x^2$.  If one uses the {\em Ito Calculus}
$dt\,dt = dt\,dB = dB\,dt = 0$ and $dB\,dB = dt$, then one can derive
this formula as follows using Taylor series by \begin{eqnarray*} df(t,
X) &=& f_t\,dt + f_x\,dX + \onehalf f_{xx} (dX)^2\\ &=& f_t\,dt +
f_x\,(\mu\,dt + \sigma\,dB)
	+ \onehalf f_{xx} (\mu\,dt + \sigma\,dB)^2\\
&=& (f_t\,dt + f_x \mu + \onehalf f_{xx} \sigma^2)dt + f_x\sigma\,
dB,\\
\end{eqnarray*}
since $(\mu\,dt + \sigma\,dB)^2 = \sigma^2\,dt$
by the Ito calculus.

Whenever $\mu = 0$, $(X_t)_{t\ge0}$ is a martingale. If we take $\mu(t,x)
= \sigma x$ and let $dS_t = \mu(t,S_t)\,dB_t$, $S_0 = 1$, then $dS/S =
\sigma\,dB$. By Ito's formula $d\log S = (1/S)\,dS - \onehalf (1/S^2)
(dS)^2 = \sigma\,dB - \onehalf\sigma^2\,dt$. Integrating we get $\log S_t
= \sigma B_t - \onehalf\sigma^2 t$ so $S_t = e^{\sigma B_t - \sigma^2t/2}$
is a martingale.

\end{section}

\begin{section}{The Black-Scholes/Merton Model}
The B-S/M model specifies a bond and a stock price process $X_t = (\rho
e^{rt}, se^{\mu t + \sigma B_t})$ where $r$ is the risk-free (funding)
rate and $\rho$, $s$, $\mu$, and $\sigma$ are parameters. If we take the
deflator $\Pi_t = e^{-rt}$ then clearly the bond process deflates to
a martingale.

\begin{exercise}
Show $se^{\mu t + \sigma B_t} e^{-rt}$ is a martingale if and only
if $\mu = r - \sigma^2/2$.
\end{exercise}

Sticking to the game plan gives the price of a call with strike
$K$ expiring at time $T$ as $C(K,T) = e^{-rt}E[(S_T - K)^+]$.

\end{section}

\end{document}
