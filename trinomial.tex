\documentclass[12pt,letterpaper]{report}
\usepackage[latin1]{inputenc}
\usepackage{amsmath}
\usepackage{amsfonts}
\usepackage{amssymb}
\usepackage{lmodern}
\usepackage{kpfonts}
\author{Keith A. Lewis}
\title{Trinomial Model}
\begin{document}
\maketitle
Fix \(t_0 < t_1 < \cdots < t_n\) and let \(R_j\) be the realized return
from \(t_j\) to \(t_{j+1}\), i.e., 1 unit invested at time \(t_j\)
pays \(R_j\) at time \(t_{j+1}\) where \(R_j\) is known at
time \(t_j\). The stochastic discount to time \(t_j\)
is \(D_j = \Pi_{i<j}1/R_i\).

If the price of an instrument at time \(t_j\) is \(X_j\) and pays a cash flow
\(C_j\) then an arbitrage free model satisfies
\[
X_j D_j = E_j\left[\sum_{j<i\le k} C_i D_i + X_k D_k\right]
\]
where \(E_j\) is the conditional expectation at time \(t_j\).

If \(D(i,j)\) is the price at time \(t_i\) of a zero coupon bond maturing at
time \(t_j\) then 
\[D(i,j) = E_iD_j/D_i.
\]
Assume we are given \(D(0,j)\)
for all \(j\). Define \(f_j\) by \(D(0,j) = \Pi_{i<j}e^{-f_i\Delta t_i}\)
and \(r_j\) by \(D_j = \Pi_{i<j}e^{-r_i\Delta t_i}\).

A forward contract has price \(0\) at time \(t_i\) and has cash flows
\(-1\) at time \(t_j\) and \(1 + F(i;j,k)\delta_{jk}\) at time \(t_k\)
where \(\delta_{jk}\) is the day count fraction from \(t_j\) to
\(t_k\). Assuming no arbitrage, 
\[
	F(i;j,k) = \frac{1}{\delta_{jk}}\left(D(i,j)/D(i,k) - 1\right).
\]

The state space for the trinomial model is \(\Omega = \{(i,j)\colon 1\le i\le n, -i\le j\le i\}\)
and probabilities \(p_{ij}^\delta\) for the
transitions \((i,j)\to (i + 1,j + \delta)\) for \(\delta\in\{1,0,-1\}\).
Suppose \(D_{ij}\) is the value of \(D_i\) in state \((i-1,j)\).
(Recall \(D_j\) is predictable.)
To tune the model to the spot curve we require \(D(0,j) = E D_j\).
Recall \(D_0 = 1\). Since \(E D_1 = D_1\) we have \(D_1 = D(0,1)\).
From \(D(0,2) = ED_2\) we have \(D(0,2) = D_{11}p_1^+ + D_{10}p_1 + D_{1-1}p_1^-\).

The trinomial model is \(r_j = f_j(1 + \delta\sigma)\) where
\(\delta = \epsilon\) with probability \(p_\epsilon\) for
\(\epsilon\in\{-1,0,1\}\).

Clearly \(f_0 = r_0\). Since \(D(0,2) = ED_2\) we have
\(e^{-f_1\Delta t_1} = Ee^{-r_1\Delta t_1}\) so \(
1 = e^{\Delta\phi}p_- + (1 - p_- - p_+) +e^{-\Delta\phi}p_+
\) where \(\Delta\phi = f\sigma\Delta t\), hence
\(p_+ = e^{\Delta\phi} p_-\).

Start with \(D_j = \Pi_{i<j}e^{-f_i(1+\sigma)^{W_i}\Delta t_i}\).
If \(\sigma = 0\) then \(D_j = D(0,t_j)\).
Note \(dED_j/d\sigma = \Pi_{i<j}e^{-f_i(1+\sigma)^{W_i}\Delta t_i}\).

\end{document}