\documentclass[11pt,fleqn]{amsproc}

\newcommand{\R}{{\bf R}}
\newcommand{\Var}{\mathop{\rm{Var}}}
\newcommand{\Cov}{\mathop{\rm{Cov}}}

%\newtheorem{prob}{Problem}[section]
%\newtheorem{cor}[thm]{Corollary}
\newtheorem{xca}{Exercise}

\renewcommand{\labelenumi}{(\alph{enumi})}

\begin{document}

\title{Derivative Securities Homework 2}
\author{Due 7:10PM, 8 Feb 2010}

\maketitle

\begin{xca}{(40 points)}
Consider the one period model $x = (1, s, v)$, $\Omega = \{S^+, S^-\}$ and
$X = (R, \omega, f(\omega))$.  Write $F^+ = f(S^+)$ and $F^- = f(S^-)$.
\begin{enumerate}
\item Find $m$ and $n$ such that $m R + n S^+ = F^+$ and
$m R + n S^- = F^-$. Show no arbitrage implies
$v = (qF^+ + (1 - q)F^-)/R$, where $q = (Rs - S^-)/(S^+ - S^-)$.
\item Find scalars $\pi^+$, $\pi^-$ such that
$(1, s) = (R, S^+)\pi^+ + (R, S^-)\pi^-$. Use the fundamental theorem
of asset pricing to derive the formula for the option value in the
previous item.
\item Show $\partial v/\partial s = n$.
\item Show no arbitrage implies $S^-\le Rs \le S^+$.
\end{enumerate}
\end{xca}

\begin{xca}{(30 points)}
Consider the one period model $x = (1, s, v)$,
$\Omega = \{u, d\}$ and $X = (R, s\omega, f(s\omega))$.
\begin{enumerate}
\item Show no arbitrage implies $v = (qF^+ + (1 - q)F^-)/R$, where
$q = (R - d)/(u - d)$, $F^+ = f(su)$, and $F^- = f(sd)$.
\item Find a payoff, $f$, such that $\partial v/\partial s \not= n$, where
$n = (f (su) - f (sd))/(su - sd)$.
\item Show $ns = (\partial/\partial R)(Rv)$ for any payoff $f$.
\item Show $ns = (\partial/\partial R)(Rv)$ for any payoff $f$
where $v$ is defined as in problem 1 (a) above.
\end{enumerate}

\end{xca}

\begin{xca}{(30 points)}
Consider the one period model $x = (1, s, v)$,
$\Omega = [L,H]$, $0\le L < H \le\infty$, and
$X = (R, \omega, \max\{\omega - K, 0\})$.
\begin{enumerate}
\item Assuming $L\le Rs\le H$ and $L\le K\le H$, what are all the conditions on $s$ and $v$ implied by the lack
of arbitrage in the model?
\item What are the conditions on $s$ and $v$ in the case $L = 0$ and
$H = \infty$?
\end{enumerate}
\end{xca}

\end{document}
