\documentclass[11pt,fleqn]{amsproc}
\usepackage{textcomp}

\newcommand{\R}{{\bf R}}
\newcommand{\F}{\mathcal{F}}
\newcommand{\G}{\mathcal{G}}
\renewcommand{\H}{\mathcal{H}}
\newcommand{\K}{\mathcal{K}}
\newcommand{\M}{\mathcal{M}}
\newcommand{\N}{\mathcal{N}}
\renewcommand{\P}{\mathcal{P}}
\newcommand{\X}{\mathcal{X}}

\newcommand{\Var}{\mathop{\rm{Var}}}
\newcommand{\Cov}{\mathop{\rm{Cov}}}
\renewcommand{\ker}{\mathop{\rm{ker}}}
\newcommand{\ran}{\mathop{\rm{ran}}}

%\newtheorem{prob}{Problem}[section]
%\newtheorem{cor}[thm]{Corollary}
\newtheorem{xca}{Exercise}

\renewcommand{\labelenumi}{(\alph{enumi})}

\begin{document}

\title{Derivative Securities Homework 6}
\author{Due 7:10PM, 22 Mar 2009}

\maketitle

\begin{xca}{(40 points)}
Using a model consisting of a bond with price dynamics
$dR/R = \rho\,dt$ and stock $dS/S = \mu\,dt + \sigma\,dB$,
Black-Scholes 1973 assumed the option price $v = v(t, s)$
is a function of time and stock price.
They derive the PDE (equation 7 in their paper)
$v_t = \rho v - \rho sv_s - \sigma^2 s^2 v_{ss}/2$. In Merton 1973
it is assumed the option price $w = w(t, r, s)$ is a function
of time, bond, and stock price. He derives the PDE (equation 33
in the paper)
$\sigma^2 s^2 w_{ss}/2 + \rho sw_s + \rho r w_r + w_t - \rho w = 0$.

If $v(t, s) = w(t, e^{\rho t}, s)$ and $w$ is a solution of
Merton's equation, show $v$ solves the Black-Scholes PDE.
\end{xca}


\begin{xca}{(60 points)}
Let $(B_t)_{t\ge0}$ be standard Brownian motion. Using the
facts (reflection principle)
\begin{equation*}
E[f(B_t)1(\max_{0\le s\le t}B_s > a)]
= E[(f(B_t) + f(2a - B_t))1(B_t > a)]
\end{equation*}
and (Girsanov's theorem)
\begin{equation*}
E[g(B_t, \max_{0\le s\le t}B_s)]
= E[e^{-\theta^2 t/2 + \theta B_t}
g(B_t - \theta t, \max_{0\le s\le t}B_s - \theta s)]
\end{equation*}
find a closed-form formula for the value of a up-in call option
with strike $k$ and barrier $h$.
Assume interest is zero and
$F_t = fe^{-\sigma^2 t/2 + \sigma B_t}$ describes the
dynamics of the underlying.
\end{xca}

\end{document}
