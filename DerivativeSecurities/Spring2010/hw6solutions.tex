\documentclass[11pt,fleqn]{amsproc}
\usepackage{textcomp}

\newcommand{\R}{{\bf R}}
\newcommand{\F}{\mathcal{F}}
\newcommand{\G}{\mathcal{G}}
\renewcommand{\H}{\mathcal{H}}
\newcommand{\K}{\mathcal{K}}
\newcommand{\M}{\mathcal{M}}
\newcommand{\N}{\mathcal{N}}
\renewcommand{\P}{\mathcal{P}}
\newcommand{\X}{\mathcal{X}}

\newcommand{\Var}{\mathop{\rm{Var}}}
\newcommand{\Cov}{\mathop{\rm{Cov}}}
\renewcommand{\ker}{\mathop{\rm{ker}}}
\newcommand{\ran}{\mathop{\rm{ran}}}

%\newtheorem{prob}{Problem}[section]
%\newtheorem{cor}[thm]{Corollary}
\newtheorem{xca}{Exercise}

\renewcommand{\labelenumi}{(\alph{enumi})}

\begin{document}

\title{Derivative Securities Homework 6}
\author{Due 7:10PM, 22 Mar 2009}

\maketitle

\begin{xca}{(40 points)}
Using a model consisting of a bond with price dynamics $dR/R = \rho\,dt$
and stock $dS/S = \mu\,dt + \sigma\,dB$, Black-Scholes 1973 assumed
the option price $v = v(t, s)$ is a function of time and stock price.
They derive the PDE (equation 7 in their paper) $v_t = \rho v - \rho
sv_s - \sigma^2 s^2 v_{ss}/2$. In Merton 1973 it is assumed the option
price $w = w(t, r, s)$ is a function of time, bond, and stock price. He
derives the PDE (equation 33 in the paper) $\sigma^2 s^2 w_{ss}/2 +
\rho sw_s + \rho r w_r + w_t - \rho w = 0$.

If $v(t, s) = w(t, e^{\rho t}, s)$ and $w$ is a solution of
Merton's equation, show $v$ solves the Black-Scholes PDE.
\end{xca}

Note $v_t = w_t + w_r e^{\rho t} \rho$,
$v_s = w_s$, and $v_{ss} = w_{ss}$ so $w$ solving the the Merton PDE
implies $0 = \sigma^2 s^2 v_{ss}/2 + \rho s v_s + v_t - \rho v$.
???

\begin{xca}{(60 points)}
Let $(B_t)_{t\ge0}$ be standard Brownian motion. Using the
facts (reflection principle)
\begin{equation*}
E[f(B_t)1(\max_{0\le s\le t}B_s > a)]
= E[(f(B_t) + f(2a - B_t))1(B_t > a)]
\end{equation*}
and (Girsanov's theorem)
\begin{equation*}
E[g(B_t, \max_{0\le s\le t}B_s)]
= E[e^{-\theta^2 t/2 + \theta B_t}
g(B_t - \theta t, \max_{0\le s\le t}B_s - \theta s)]
\end{equation*}
find a closed-form formula for the value of a up-in call option
with strike $k$ and barrier $h$.
Assume interest is zero and
$F_t = fe^{-\sigma^2 t/2 + \sigma B_t}$ describes the
dynamics of the underlying.
\end{xca}

For this problem it is more convenient to use the formula
\begin{eqnarray*}
&&E[g(B_t) 1(\max_{0\le s\le t}B_s - \theta s > a)]\\
&&\quad = E[e^{-\theta B_t - \theta^2 t/2}
g(B_t + \theta t) 1(\max_{0\le s\le t}B_s > a)]
\end{eqnarray*}

\begin{eqnarray*}
&&E[g(B_t) 1(\max_{0\le s\le t}B_s - \theta s > a)]\\
&&\quad =
E[e^{\theta B_t - \theta^2t/2}
e^{-\theta B_t + \theta^2t/2}
g(B_t) 1(\max_{0\le s\le t}B_s - \theta s > a)]\\
&&\quad =
E[e^{-\theta (B_t + \theta t) + \theta^2t/2}
h(B_t + \theta t) 1(\max_{0\le s\le t}B_s > a)]
\end{eqnarray*}
The option payoff is $\max\{F_t - k, 0\} 1(\max_{0\le s\le t} F_s > h)$.
We write this as $g(B_t)
1(\max_{0\le s\le t} B_s - \sigma t/2 > \frac{1}{\sigma}\log \frac{h}{f})$.
Taking $\theta = \sigma/2$ and $a = \frac{1}{\sigma}\log \frac{h}{f}$,
we have
\begin{eqnarray*}
&&E[g(B_t) 1(\max_{0\le s\le t} B_s - \theta s > a)]\\
&&\quad=
E[e^{-\theta B_t - \theta^2 t/2}
g(B_t + \theta t) 1(\max_{0\le s\le t}B_s > a)]\\
&&\quad=
E[e^{-\theta B_t - \theta^2 t/2}
g(B_t + \theta t) 1(B_t > a)]\\
&&\qquad + E[e^{-\theta (2a - B_t) - \theta^2 t/2}
g(2a - B_t + \theta t) 1(B_t > a)]\\
&&\quad=
E[g(B_t) 1(B_t - \theta t > a)]\\
&&\qquad + e^{-2\theta a} E[g(2a - B_t) 1(B_t - \theta t > a)]
\end{eqnarray*}
where the second to last equality uses reflection.

The first term simplifies to $E[(F_t - k)^+1(F_t > h)]
= E[(F_t - k)1(F_t > h)]$ assuming $h > k$. (Otherwise it has the
same payoff as a standard call option.) We have $e^{-2\theta a}
= f/h$ so the second term reduces
to $E[(he^{-\sigma B_t - \sigma^2t/2} - fk/h)^+1(F_t > h)]$
which can be given a closed form solution.

\end{document}
