\documentclass[11pt,fleqn]{amsproc}

\newcommand{\R}{{\bf R}}
\newcommand{\Var}{\mathop{\rm{Var}}}
\newcommand{\Cov}{\mathop{\rm{Cov}}}

%\newtheorem{prob}{Problem}[section]
%\newtheorem{cor}[thm]{Corollary}
\newtheorem{xca}{Exercise}

\renewcommand{\labelenumi}{(\alph{enumi})}

\begin{document}

\title{Derivative Securities Homework 1}
\author{Due 7:10PM, 1 Feb 2010}

\maketitle

\begin{xca}{(20 points)}
Let $D(t; u, v)$ denote the price at time $t$ of a zero coupon bond
effective at time $u$ and terminating at time $v$, i.e.,
a contract for the sequence of cash flows $-D(t; u, v)$ at time 
$u$ and $1$ at time $v$, having zero price at time $t$.
Show $D(t; u, v) D(t; v, w) = D(t; u, w)$ for $t\le u\le v\le w$
assuming no arbitrage and perfect liquidity.
\end{xca}

\begin{xca}{(20 points)}
If today's EUR/USD exchange rate is $1.4$, the one year USD zero price is 99,
and the one year EUR zero price is 98, what is today's one year forward 
EUR/USD exchange rate assuming no arbitrage and perfect liquidity?
\end{xca}

\begin{xca}{(20 points)}
Find puts with strikes at 1 and 2, and a calls with strikes
at 3 and 4 having payoff piecewise linear and continuous
taking on the values $f(x) = x^2 - 5x + 6$ at
$x = 1$, 2, 3, 4 and having slope -3 for $0 < x < 1$
and slope 3 for $x > 4$.
\end{xca}

\begin{xca}{(20 points)}
Let $c(s, k, t)$ [C(s,k,t)] be the price of a European [American] call
on a non dividend paying stock with current price $s$, strike $k$,
and maturity $t$. Show that in the absence of arbitrage and assuming
perfect liquidity
\begin{enumerate}
\item $c(s, k, t) \le C(s,k,t) \le s$,
\item $C(s, k, t) \ge \max\{s - kD(t),0\}$,
\item $c(s, k, t) = C(s, k, t)$,
\item $c(s, k, t) \le c(s, k, t')$ when $t < t'$,
\end{enumerate}
where $D(t)$ is the discount to time $t$. Assume
$D(t)$ is non-increasing to prove item (iii).
\end{xca}

\begin{xca}{(20 points)}
Use the formula $E[e^N f(N_1,\dots)] = E[e^N] E[f(N_1 + \Cov(N,
N_1),\dots)]$, where $N$, $N_1$,\dots are jointly normal to show
\begin{equation*}
\Cov(N,f(N_1,\dots)) = \sum_i \Cov(N,N_i) E[D_if(N_1,\dots)],
\end{equation*}
where $D_i f$ is the derivative of $f$ with respect to the $i$-th argument.
\end{xca}

\end{document}
