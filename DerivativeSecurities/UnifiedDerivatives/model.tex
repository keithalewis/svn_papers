\chapter{Model}

This chapter introduces the unified model that will be used throughout
the remainder of the book. As is customary, it is assumed to be a set,
$\Omega$, that contains all possible outcomes.  Trades can only occur
at times $t_0 < t_1 < \cdots < t_n$. The information available at time
$t_j$ is represented by an algebra of subsets of $\Omega$, $\F_j$, $0\le
j\le n$.  For the first cut, it is also assumed the market has perfect
liquidity. Every instrument can be bought or sold in any quantity at the
model price.  The price at time $t_j$ of every instrument is specified by
a bounded $\F_j$-measurable function $X_j\colon\Omega\to\R^I$, $0\le j\le
n$, where $I$ is the finite collection of all instruments. The cash flow
associated with owning an instrument at time $t_{j-1}$ is also represented
by a bounded $\F_j$-measurable function $C_j\colon\Omega\to\R^I$,
$0<j\le n$.

\section{Remarks}
An {\em event} is a subset of the space of outcomes. The possible
outcomes from rolling a die are $\{1,2,3,4,5,6\}$. The event 'rolling
an even number' is the subset $\{2,4,6\}$.

Algebras of subsets of outcomes represent partial information.
