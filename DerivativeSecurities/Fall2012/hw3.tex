\documentclass[11pt,fleqn]{amsproc}

\renewcommand{\AA}{{\mathcal A}}
\newcommand{\R}{{\bf R}}
\newcommand{\Var}{\mathop{\rm{Var}}}
\newcommand{\Cov}{\mathop{\rm{Cov}}}

%\newtheorem{prob}{Problem}[section]
%\newtheorem{cor}[thm]{Corollary}
\newtheorem{xca}{Exercise}

\renewcommand{\labelenumi}{(\alph{enumi})}

\begin{document}

\title{ORIE 5914 Equity Derivatives Homework 3}
\author{Due 4:00PM, 10 October 2012}

\maketitle

\begin{xca}{(20 points - dividends)}
Suppose a stock pays dividends \((d_j)\) at times \((t_j)\) in a multi-period model.
Let \((M_j)\) be any martingale on the probability space \((\Omega, P, (\AA_j))\). Show
\(X_j = (R^j, sR^jM_j - \sum_{i \le j} d_iR^{j - i})\), \(C_j = (0, d_j)\), and
\(\Pi_j = R^{-j}P\) is an arbitrage free model.
\end{xca}

\begin{xca}{(20 points - L\'evy processes)}
If \((X_t)_{t\ge0}\) is a stationary process having independent increments
show \(Ee^{sX_{t + u}} = Ee^{sX_t} Ee^{sX_u}\).
\end{xca}

\begin{xca}{(20 points)}
Equation (7) in the original Black-Scholes paper {\it The Pricing of Options and Corporate Liabilities} is 
\[w_2 = r w - rxw_1 - \frac{1}{2}v^2x^2 w_{11}.\]
In the notation used in the notes, this would be written 
\[w_t = \rho w - \rho s w_s - \frac{1}{2}\sigma^2s^2 w_{ss}.\]
however the equation we derive has an extra term of \(\rho r v_r\)
\[v_t = \rho v - \rho rv_r -\rho s v_s - \frac{1}{2}\sigma^2s^2 v_{ss}.\]
Define \(w(t, s) = v(t, e^{\rho t}, s)\) and use the latter equation to
derive the original Black-Scholes equation.
\end{xca}

\begin{xca}{(20 points - digital barrier option)}
Use the reflection principle and Girsanov's theorem to show
\(P(\bar{F}_t > h) = (1 + h/f)P(F_t > h)\), where \(F_t = fe^{-\sigma^2t/2 + \sigma B_t}\)
and \(\bar{F}_t = \max_{0\le s\le t} F_s\). {\rm Hint: take \(\theta = -\sigma/2\).}
\end{xca}

\begin{xca}{(20 points)}
Implement an Excel add-in {\tt UID\_(Forward, Volatility, Barrier, Expiration)} that
calculates the value of \(P(\bar{F}_t > h)\) for barrier \(h\). Document the function
using the XLL.DOC macro and turn in both the {\tt .xll} and {\tt .chm} files.
\end{xca}

\end{document}
