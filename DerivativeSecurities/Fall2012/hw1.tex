\documentclass[11pt,fleqn]{amsproc}

\newcommand{\R}{{\bf R}}
\newcommand{\Var}{\mathop{\rm{Var}}}
\newcommand{\Cov}{\mathop{\rm{Cov}}}

%\newtheorem{prob}{Problem}[section]
%\newtheorem{cor}[thm]{Corollary}
\newtheorem{xca}{Exercise}

\renewcommand{\labelenumi}{(\alph{enumi})}

\begin{document}

\title{ORIE 5914 Equity Derivatives}
\author{Due 4:00PM, 26 September 2012}

\maketitle

\begin{xca}{(20 points)}
In the one period model, show that if there is no arbitrage
and \(\xi\cdot X(\omega) = 0\), \(\omega\in\Omega\), then
\(\xi\cdot x = 0\).
\end{xca}

\begin{xca}{(20 points - cost of carry)}
Consider the one period model \(x = (D, s, 0 )\), \(\Omega = [0,\infty)\),
and \(X(\omega) = (1, \omega, \omega - f)\). Use
exercise one to prove \(-fD + s = 0\).
\end{xca}

\begin{xca}{(20 points - put/call parity)}
Consider the one period model \(x = (D, s, c, p)\), \(\Omega = [0,\infty)\),
and \(X(\omega) = (1, \omega, (\omega - k)^+, (k - \omega)^+)\). Use
exercise one to prove \(-kD + s - c + p = 0\).
\end{xca}

\begin{xca}{(20 points)}
Consider the one period model \(x = (D, s, c_1, c_2)\),  \(\Omega = [0,\infty)\),
and \(X(\omega) = (1, \omega, (\omega - k_1)^+, (\omega - k_2)^+)\) where \(k_1 < k_2\). Use
the fundamental theorem of asset pricing to prove \(-D(k_2 - k1) \le c_2 - c_1 \le 0\).
\end{xca}

\begin{xca}{(20 points - 90/100/110 example with no bond)}
Consider the one period model \(x = (100, 9.9)\),  \(\Omega = [90, 110]\),
and \(X(\omega) = (\omega, (\omega - 100)^+)\). Find the arbitrage.
\end{xca}

\end{document}
