\documentclass{amsart}
\usepackage{amsmath}
\usepackage{amsfonts}

\newcommand{\RR}{\mathbb{R}}
\newcommand{\Var}{\mathop{\rm Var}}


\title{The L\'evy-Khintchine Theorem}
\author{Keith A. Lewis}
\email{kal@kalx.net}

\begin{document}
\maketitle

\begin{section}{Everybody uses it...}

...but nobody proves it (except old-timers).

\medskip

{\bf Theorem.} {\it
If $X$ is infinitely divisible, then there exist scalars
$\mu$, $\sigma$, and a measure $\nu$ on $\RR$ with
$\nu(\{0\}) = 0$ and $\int_\RR (|x|^2\wedge 1)\,d\nu(x) < \infty$
such that the log of the characteristic function of $X$ is}

\begin{quote}
$
	i\mu t - \sigma^2 t^2/2
		+ \int_\RR (e^{itx} - 1 - itx 1_{\{|x|\le1\}})\,d\nu(x).
$
\end{quote}

\end{section}

\begin{section}{Finite Variance Case}

{\bf Theorem.} (Kolomogorov) {\it
If $X$ is infinitely divisible and has finite variance,
then there exists a scalar
$\gamma$, and a nondecreasing function $G$ such that}

\begin{quote}
$
	\log E\exp(itX) = i\gamma t + \int_{-\infty}^\infty (e^{itx} - 1 - itx)/x^2\,dG(x).
$
\end{quote}
{\it Also, $E[X] = \gamma$ and $\Var(X) = G(\infty) - G(-\infty)$.}

\medskip

This is a better parameterization because it is continuous
in $\gamma$ and $G$.

\end{section}

\begin{section}{Definitions}

The {\it cumulative distribution function} of the random variable
$X$ is $F(x) = P(X\le x)$.
Two random variables {\it have the same law} if they have the
same cumulative distribution functions.

\medskip

The {\it characteristic function} of the random variable $X$ is
$\phi(t) = E[\exp(itX)] = \int_{-\infty}^\infty e^{itx}\,dF(x)$.

\medskip

The random variable $X$ is {\it infinitely divisible} if for every
positive integer $n$ there exist i.i.d., $X_j$, with
$X_1 + \cdots + X_n$ having the same law as $X$.

\end{section}


\begin{section}{Cumulative Distributions}

If $F$ is a cumulative distribution function, then
\begin{itemize}
\item $F(-\infty) = 0$ and $F(\infty) = 1$,
\item $F(x) \le F(y)$ if $x < y$,
\item $F$ is right continuous and has left limits.
\end{itemize}

\medskip

The converse is also true.

\end{section}

\begin{section}{Characteristic Functions}

If $\phi$ is a characteristic function, then
\begin{itemize}
\item $\phi(0) = 1$,
\item $|\phi(t)| \le 1$ for all $t$,
\item $\phi(t)$ is uniformly continuous,
\item $\phi(-t) = \overline{\phi(t)}$.
\end{itemize}

\medskip

Two random variables have the same characteristic function if
and only if they have the same law. (Fourier inversion theorem.)

\end{section}

\begin{section}{Infinite Divisibility}
If $X$ is infinitely divisible, then
\begin{itemize}
\item for all $n$ there exists a characteristic function $\phi_n$
with $\phi(t) = \phi_n(t)^n$ for all $t$,
\item $1 - \phi(2t) \le 4(1 - \phi(t))$ if $\phi(t) \ge 0$.
\item the characteristic function does not vanish,
\item $\phi_n(t) \to 1$ as $n\to\infty$.
\end{itemize}

\end{section}

\begin{section}{Helly's Theorem}

{\bf Theorem.} {\it If $F_n$ are nondecreasing and bounded,
then there exists a
nondecreasing $F$ and a subsequence $F_{n_k}$ such that
for any continuous $f$ that vanishes at infinity}
$$\lim_{k\to\infty}\int_{-\infty}^\infty f(x)\,dF_{n_k}(x)
		= \int_{-\infty}^\infty f(x)\,dF(x).$$
\smallskip
This is a simple consequence of the Riesz Representation Theorem,
Aloaglu's Theorem and a one semester course in Functional Analysis.

\end{section}

\begin{section}{Proof of L-K}

\begin{eqnarray*}
\log\phi(t) &=& n\log\phi_n(t)\\
	&=& n\log (1 + \phi_n(t) - 1)\\
	&=& \lim_{n\to\infty} n(\phi_n(t) - 1)\\
	&=&  \lim_{n\to\infty}\int_{-\infty}^\infty (e^{itx} - 1)\,n\,dF_n(x),\\
\end{eqnarray*}
where $F_n$ is the c.d.f. of $\phi_n$.

\smallskip

We want $n\,dF_n$ to converge to something...

\end{section}

\begin{section}{Proof...}

...but they don't since $dF_n$ converges to a point mass at 0.
Note $\gamma = E[X] = nE[X_n]$, and $\Var(X) = n\Var(X_n)$ so
\begin{eqnarray*}
\Var(X) &=& nE[X_n^2] - nE[X_n]^2\\
	&=& \int_{-\infty}^\infty x^2\,ndF_n(x) - \gamma^2/n.\\
\end{eqnarray*}

Define $dG_n(x) = x^2\,ndF_n(x)$, then $G_n$ satisfy the hypotheses
of Helly's Theorem.

\smallskip

{\bf Exercise.} {\it Show $\Var(X) = \int_{-\infty}^\infty dG(x)$.}

\end{section}

\begin{section}{Proof!}

Too bad $(e^{itx} - 1)/x^2$ is not continuous.
However, since $i\gamma t = \int_{-\infty}^\infty itx\,ndF_n(x)$
we have
\begin{eqnarray*}
	\log\phi(t) &=& i\gamma t + \lim_{k\to\infty} \int_{-\infty}^\infty
		\frac{e^{itx} - 1 - itx}{x^2}\,dG_{n_k}(x)\\
	&=& i\gamma t + \int_{-\infty}^\infty
		(e^{itx} - 1 - itx)/x^2\,dG(x)\\
\end{eqnarray*}
Q.E.D.

\end{section}

\end{document}
