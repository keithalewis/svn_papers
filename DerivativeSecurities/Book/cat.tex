\chapter{Category Theory} 
\label{cat} 

\section{Definition}
A {\em category} consists of {\em objects} and {\em arrows}. Every
arrow has a {\em domain} and a {\em codomain} that are objects.
The canonical example is the category of sets where the objects
are sets and the arrows are functions. Other examples are provided by
replacing functions with partial functions, relations, or inclusion
as arrows. 

In a category, an arrow with a codomain that is the same as the
domain of another arrow can be {\em composed} resulting in an
arrow having the domain of the first and the codomain of the second.
The composition is required to be associative, it does not matter
which composition is applied first.

In a category, every object has an {\em identity arrow}. The object is
the domain and codomain of the arrow and composition of any arrow with
an identity arrow results in the original arrow.

\section{Special Arrows}
An arrow is {\em monic} if when the domain of two arrows equals its codomain,
then the two arrows must be the same. For the category of sets, this is
the same as 1-to-1, or injective. 

An arrow is {\em epic} if when the codomain of two arrows equals its domain,
then the two arrows must be the same. For the category of sets, this is
the same as onto, or surjective.

An arrow is {\em iso} if there exist arrows that compose to the identity
arrow with its domain and codomain.

\section{Special Objects}
An object is {\em initial} if it is the domain of an arrow for every
object in the category. An object is {\em terminal} if it is the codomain
of an arrow every object in the category.

\section{Functors}
A {\em functor} maps a category to a category. It assigns objects of the
first to objects of the second and arrows of the first to arrows of the
second while preserving the structure of each category.

Functors form a category. The objects are categories and the arrows
are functors.

\section{Natural Transformations}
Given two functors, a {\em natural transformation} associates the
the objects of the first functor with arrows of the second functor
preserving the structure of each functor. It is a 2-functor. There
are n-functors.
