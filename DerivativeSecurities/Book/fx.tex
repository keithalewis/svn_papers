\chapter{Foreign Exchange}

\section{Introduction}
Foreign exchange markets permit the exchange of one currency for
another. It is a large and very liquid markets run primarily by large
banks and central banks.  Around \$1 trillion turns over in the spot
market every day while \$2 trillion in notional is outstanding for forward
and FX swaps at any given time.  Outside of commercial business, most
trading is done through electronic exchanges that match retail clients
with major banks that provide liquidity.  Until recently, there was very
little regulation in this market.  The exchanges and broker/dealers are
regulated by the CFTC and NFA, but there is no overall regulator for
major banks.

\section{Quoting}
When the {\tt USD|JPY} exchange rate is 95 that means USD 1 can
be exchanged for JPY 95. The first currency is called
the {\em base} and the second is called the {\em counter}.
More precisely, if the quote
is 95.123/97 it means the bid is 95.123 and the ask is 95.197.
This means you can sell USD 1,000 and receive JPY 95,123
or sell JPY 95,197 and receive USD 1,000. The first three
digits of the quote are called the {\em big figure}. The last
two digits are almost always all you need to determine the
bid/ask spread for the {\em pair}.

It is theoretically equivalent to say the {\tt JPY|USD} exchange
rate is .0105127/045 but FX traders will make fun of you.  There is a
market convention about which currency should be the base and which the
counter. In general, the order that makes the quote greater than 1 is
preferred. The major exception being {\tt EUR|GBP}. The currency chosen
for the base is prioritized by EUR, GBP, AUD, NZD, USD, CAD, CHF, and JPY.
These currencies are called the {\em majors} while all others are called
the {\em minors}.

\section{Exchanges}
Nowadays, almost all exchange trading is electronic. You open an account
by wiring an initial {\em margin} specified by the exchange. One exchange
allows an initial margin of \$25, but a more typical amount is \$1000.
Suppose you sold USD 10,000 for JPY 951,230. Even though you only
have \$1,000 in margin, you can trade up to a fixed multiple of that
in the base currency. This multiple is called your {\em leverage} and
is typically in the range 20--200 depending on what view the
exchange takes on your credit worthiness.

Let $X^a(t)$ represent the yen ask price. If you unwound your trade your
USD account would become zero and your JPY account would become $951,230
- 10,000 X^a(t)$.  That could be converted to $(951,230 - 10,000
X^a(t))/X^b(t)$ dollars, where $X^b(t)$ is the yen bid at time $t$.
The exchange does exactly this calculation periodically and adjusts
your margin by this amount. If your margin goes to zero, either
you top off your account or you get taken out of the trade and your
account is closed.

At the end of each trading day there is another adjustment to your margin
that is calculated called the {\em roll}. Each quantity of currency in
your positions gets interest since the last trading day based on market
quotes. For a fun time, you should ask your exchange how they come up
with this number and exactly how they take into account intraday
trading.

If exchange rates are fairly stable between a currency pair, but one
currency has a significantly higher interest rate, then it is possible
to make money on the roll by shorting the low interest rate currency.
This is called a {\em carry} trade.

Exchanges have agreements with large banks or broker/dealers to provide
liquidity. I.e., to supply bid and ask prices at which they are willing
to exchange a minimum specified quantity of each currency pair. The
exchange supplies their clients the best quotes, widened by a couple of
basis points that the exchange intends to keep for themselves. A large
exchange can do a volume of \$1B/day. If they make 2bps that works out
to \$200,000/day.

%\section{Model}
