\chapter{Multi-Period Model}

\section{Introduction}
\label{mpm-intro}

Fix times $t_0 < \cdots < t_n$ and algebras of sets on $\Omega$,
$\F_0\subset\dots\subset\F_n\subset\F$.  Let $X_j\colon\Omega\to\R^m$,
$0\le j \le n$, be $\F_j$ measurable, bounded, vector valued functions
representing a model of the prices of  $m$ market instruments at times
$(t_j)$.  Assume one share of instrument $j$ pays a cash flow, $C_j$,
at time $t_j$ if it was purchased at time $t_{j-1}$. E.g., a stock
dividend, a swap or bond coupon, a spot FX interest rate roll.

Let $\Xi_j\colon\Omega\to\R^m$, $0\le j < n$, be $\F_j$ measurable
and bounded vector valued functions representing the trading position
held in the market from time $t_j$ to time $t_{j+1}$.  The cash flows
associated with this are $-\Xi_0\cdot X_0$ at $t_0$, $\Xi_{j - 1}\cdot
(X_j + C_j) - \Xi_j\cdot X_j$ at $t_j$, $0 < j < n$, and $\Xi_{n-1}\cdot
(X_n + C_n)$ at $t_n$.

We will use the notation $B_j = B(\Omega, \F_j)$ for bounded
$\F_j$ measurable functions on $\Omega$ and
$B_j\otimes\R^m$ for bounded measurable functions taking values in $\R^m$.
The dual of $B_j$ is $B_j^* = ba_j = ba(\Omega, \F_j)$, the space of
finitely additive measures on $(\Omega, \F_j)$.

\begin{definition}
With prices $(X_j)$ and cash flows $(C_j)$ as above, 
arbitrage exists if there are positions
$\Xi_j\in B_j\otimes\R^m$ such that 
$-\Xi_0\cdot X_0 > 0$,
$\Xi_{j - 1}\cdot (X_j + C_j) - \Xi_j\cdot X_j\ge0$, $0 < j < n$,
and $\Xi_{n-1}\cdot (X_n + C_n)\ge0$.
\end{definition}

\begin{theorem}
There is no arbitrage if and only if there exist
$\Pi_j\in ba_j$ with $\Pi_0 > 0$ and $\Pi_j\ge0$ for $0 < j \le n$ 
such that $E[(X_{j+1} + C_{j+1})\Pi_{j+1}|\F_j] = X_j\Pi_j$, $0\le j < n$.
\end{theorem}
\begin{proof}
One direction is easy. If such $(\Pi_j)$ exist, then 
$-\Xi_0\cdot X_0\Pi_0 > 0$, 
$(\Xi_{j - 1}\cdot (X_j + C_j) - \Xi_j\cdot X_j)\Pi_j \ge0$, $0 < j < n$,
and $\Xi_{n-1}\cdot (X_n + C_n)\Pi_n\ge0$. 
The sum of these cash flows is
$\sum_{j=0}^{n-1} \Xi_j\cdot((X_{j+1} + C_{j+1})\Pi_{j+1} - X_j\Pi_j) > 0$.
This is a contradiction if
$E[(X_{j+1} + C_{j+1})\Pi_{j+1} - X_j\Pi_j|\F_j] = 0$ for all $j$.

To prove the converse, define the linear operator 
$A:\oplus_{j = 0}^{n-1} B_j\otimes\R^m\to\oplus_{j = 0}^{n} B_j$ by
\begin{equation*}
A(\oplus_{j=0}^{n-1}\Xi_j) =
\oplus_{j=0}^n \Xi_{j - 1}\cdot (X_j + C_j) - \Xi_j\cdot X_j,
\end{equation*}
where $\Xi_{-1} = 0 = \Xi_n$.
Let $\PP = \{\oplus_0^n P_j:P_0 > 0, P_j\ge0, j > 0\}$ be the
positive cone in $\oplus_{j = 0}^{n} B_j$.
Arbitrage is equivalent to the range of $A$ intersecting this cone.
The adjoint of $A$ is
$A^*\colon\oplus_{j = 0}^{n} ba_j\to
\oplus_{j = 0}^{n-1} ba_j\otimes\R^m$ where
\begin{equation*}
A^*(\oplus_{j=0}^n \Pi_j) =
\oplus_{j=0}^{n-1} E[(X_{j+1} + C_{j+1})\Pi_{j+1}|\F_j] - X_j\Pi_j.
\end{equation*}
By the Hanh-Banach theorem, no arbitrage is equivalent to the kernel of
$A^*$ intersecting
the dual cone $\PP^+$, i.e., there exist
$\Pi_j\in ba_j$ with $\Pi_0 > 0$ and
$\Pi_j\ge0$ for $0 < j \le n$ such that
$A^*(\oplus_{j=0}^n \Pi_j) = 0$.
This is equivalent to $X_j\Pi_j
= E[(X_{j+1} + C_{j+1})\Pi_{j+1}|\F_j]$ for $0\le j < n$.
\end{proof}

Note, it is no loss in generality to assume $\Pi_0 = 1$ if $\F_0$ is
the trivial $\sigma$-algebra.

Given $\Xi_j\in B_j\otimes\R^m$, $0\le j < n$, define the martingale 
$M_j = \sum_{i = 0}^{j-1}\Xi_i\cdot((X_{i+1} + C_{i+1})\Pi_{i+1} - X_i\Pi_i)
= -\Xi_0 X_0 \Pi_0
+ \sum_{i = 1}^j (\Xi_{i-1}\cdot(X_i + C_i) - \Xi_i\cdot X_i)\Pi_i$,
$j < n$, and $M_n = -\Xi_0 X_0 \Pi_0 
+ \sum_{i = 1}^{n-1} (\Xi_{i-1}\cdot(X_i + C_i) - \Xi_i\cdot X_i)\Pi_i + \Xi_{n-1}\cdot X_n\Pi_n$.
If the trading is {\em self-financing}, i.e.,
$\Xi_{i - 1}\cdot(X_i + C_i) - \Xi_i\cdot X_i = 0$ for $0 < i < n$,
then $M_0 = E[M_n]$ shows $\Xi_0\cdot X_0\Pi_0 = E[\Xi_{n - 1}\cdot X_n\Pi_n]$.

This formula is the basis of the statement that the cost of a
derivative security is the expected value of its discounted payoff under
a risk-neutral measure. If the payoff can be expressed as $\Xi_{n -
1}\cdot X_n$ for some $\Xi_{n-1} \in B_{n-1}\otimes\R^m$ that
is the last position in a self-financing strategy, then the cost of
implementing this strategy at time $t_0$ is $\Xi_0\cdot X_0$.

As long as at each trading time a zero coupon bond to the next trading
time exists, it is always possible to make the trading strategy self
financing. If the prices and cash flows are square integrable, the problem
reduces to finding $\Xi\in L^2_{n-1}\otimes\R^m$ such that $\Xi_{n -
1}\cdot X_n = F$, where $F\in L^2_n$ is the option payoff. This problem
can be solved by finding $\Xi$ that minimizes $\|\Xi\cdot X - F\|$.

\begin{lemma}
Given $F\in L^2(\F)$, $X\in L^2(\F)\otimes\R^m$, the minimum
value of $\|\Xi\cdot X - F\|$ for $\Xi\in L^2(\G)\otimes\R^m$,
$\G\subset\F$, is $\|F\|^2 - E[\Xi\cdot X F]$, for $\Xi =
E[XX^T|\G]^{-1}E[XF|\G]$. 
\end{lemma}

\begin{proof}
The Fr\'echet derivative of $\Xi\mapsto\|\Xi\cdot X - F\|^2$ at $\Xi$
is $2E[(\Xi\cdot X - F)X|\G] = 2E[XX^T|\G]\Xi - 2E[FX|\G]$. The minimum
occurs when this vanishes.

\end{proof}

In general markets are not complete, but this method produces the
best $L^2$ approximation.

%\section*{Notes}
