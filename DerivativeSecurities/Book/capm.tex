% ef.tex - Efficient Frontier calculation

\documentclass[11pt,fleqn]{amsart}

\newcommand{\R}{\bf{R}}
\newcommand{\Var}{\mathop{\rm{Var}}}
\newcommand{\Cov}{\mathop{\rm{Cov}}}

\begin{document}
\title{Notes on Portfolio Selection}
\author{Keith A. Lewis}
\address{KALX, LLC}
\email{kal@kalx.net}
\urladdr{http://kalx.net}
\begin{abstract}
Given an endowment and a choice of investments, how does one
choose the ``best'' portfolio? This note gives a historically
motivated account of what some of the best minds in the business
had to say about this problem.
\end{abstract}

\maketitle

\section{Introduction}
The concept of being compensated for risk probably dates back to the
time we still lived in caves, but starting in the 1950's a quantitative
theory of this tradeoff started taking shape.

efficient frontier
factor models
CAPM

Kelly,
Arrow,
Markowitz,
Sharpe,
Fama,
Ross,

\section{The One Period Model}
The one period model assumes there are $n$ instruments available for
trading at prices $x = (x_1,\dots,x_n)$ at the beginning of the trading
period that evolve to random prices $X = (X_1,\dots,X_n)$ at the end of
the period.  No intraperiod trading is allowed. The prices are assumed
to be perfectly liquid, i.e., trades occur exactly at the beginning and
end of the period at the specified price no matter the quantity.

The random vector, $X$, is completely determined by the joint density
of the random variables $F(x_1,\dots, x_n) = P(X_1 \le x_1, \dots, X_n \le
x_n) = P(\{\omega\in\Omega\colon X_1(\omega) \le x_1, \dots, X_n(\omega)
\le x_n\})$, where $\Omega$ is the {\em sample space} of possible outcomes
over the time period, and $P$ is a {\em probability measure} on $\Omega$.

If the components of $X$ are jointly normal, then the joint disribution
is completely specified by the vector mean $E[X]$, and the covariance
matrix $\Cov(X,X) = E[XX^T] - E[X]E[X]^T$. Note this model allows prices
to be negative.

A {\em portfolio}, $\xi = (\xi_1,\dots,\xi_n)$, specifies how much of each
instrument to buy (or sell) at the beginning of the period. The cost of
establishing the portfolio at the beginning of the period is $\xi^T x =
\xi_1 x_1 + \cdots + \xi_n x_n$. The value of the portfolio at the end of
the period if it was liquidated is given by the random variable $\xi^T X$.

Define $R^\xi = \xi^T X/\xi^T x$, to be the {\em realized return} on
portfolio $\xi$. One unit invested at the beginning of the period will
evolve to $R^\xi$ units at the end of the period.

Sometimes it is convenient to work in terms of {\em simple return}
defined as $r = R - 1$, or $r = (R - 1)/\Delta t$ where $R$ is the
realized return and $\Delta t$ is the timespan of the period.

The {\em volatility} of the return is defined to be the square root of
its variance. Note, this is not the same as the Black-Scholes definition
of volatility.

\section{Kelly's Criterion}
In 1956, J. L. Kelly Jr. suggested using a logarithmic utility function
(as did Daniel Bernoulli in 1738 in his resolution of the St. Petersburg
paradox). The maximum of $E[\log R^\xi]$ is obtained at $E[X/\xi^TX]
- x/\xi^Tx = 0$, so $x = E(X/R^\xi)$ determines the optimal portfolio,
$\xi$. Solving for $\xi$ is rarely tractable.

\section{Sharpe Ratio}
Is an investment with a 5\% simple return having 10\% volatility better
than an investment with an 8\% simple return having 20\% volatiltiy? The
{\em information ratio} is the quotient of return by volatility.
Using this measure, the first investment is better.

Now assume that it is possible to get a 3\% return at no risk. A portfolio
with half invested in the risk free asset and half in the second asset has
a volatility of 10\%, the same as the first investment, but a return of
(3 + 8)/2 = 5.5\%. This suggests the second investment might be better
since this portfolio has the same risk but better return than the first.

Given two returns, $r_1$ and $r_2$, with nonzero volatilities, $\sigma_1$
and $\sigma_2$, and expected returns, $\mu_1$ and $\mu_2$. Consider the
portfolio consising of $1 - \sigma_1/\sigma_2$ shares in the riskless
asset and $\sigma_1/\sigma_2$ in the second asset. The volatility
of this portfolio is $\sigma_1$. The Sharpe ratio follows from the
condition that we pefer the second investment if the expected return
of this portfolio is greater than the expected return of the first,
$\mu_1 < (1 - \sigma_1/\sigma_2)r + \sigma_1/\sigma_2 \mu_2$.  This can
be rewritten $(\mu_1 - r)/\sigma_1 < (\mu_2 - r)/\sigma_2$.


The {\em Sharpe Ratio} of a simple return is $(\mu
- r)/\sigma$ where $\mu$ is the expected value of the simple return,
$\sigma$ is its volatility and $r$ is the simple return on a portfolio
having zero volatility (assuming such exists). This is an example
of a {\em utility function}.

This can be generalized somewhat.

\section{Arbitrage}
Arbitrage exists if there is a portfolio $\xi$ with $\xi^T x < 0$ and
$\xi^T X\ge0$. Note that this is equivalent to the condition $R^\xi
< 0$. (It is possible that $R^\xi < 1$ and there is no arbitrage.)
The Fundamental Theorem of Asset Pricing states that arbitrage does not
exist if and only if there is a random variable $\pi > 0$ such that $x
= E[X\pi]$.

One direction is easy. If $x = E[X\pi]$ and $\xi^T X\ge0$, then $\xi^T
x = E[\xi^T X\pi] ge0$. The other direction can be proved using the
Hahn-Banach theorem. (See Appendix.)

\section{Efficient Frontiers}
Maximize $\Phi(\xi, \lambda, \mu) = E[\xi^T X] - \lambda(\xi^T x - 1) -
\mu/2(\Var(\xi^T X) - \sigma^2)$. We have $d\Phi/d\xi = E[X] - \lambda
x - \mu \Var(X)\xi$, where $\Var(X) = E[XX^T] - E[X]E[X]^T$. Setting
$d\Phi/d\xi = 0$ and solving for $\xi$ gives $\xi = V^{-1}(E[X]
- \lambda x)/\mu$, where $V = \Var(X)$. Using $1 = \xi^T x$ and
putting $\Lambda = V^{-1}$ gives
\begin{equation*}
\mu = E[X]^T\Lambda x - \lambda x^T\Lambda x.
\end{equation*}

Using $\sigma^2 = \Var(\xi^T X) = \xi^T V\xi$ yields
\begin{align*}
\sigma^2\mu^2 &= (E[X] - \lambda x)^T \Lambda  (E[X] - \lambda x)\\
 &=  E[X]^T\Lambda E[X] - 2\lambda E[X]^T\Lambda x + \lambda^2 x^T\Lambda x
\end{align*}

Let $A = x^T\Lambda x$, $B = E[X]^T\Lambda x$, and $C = E[X]^T\Lambda E[X]$.

\end{document}
