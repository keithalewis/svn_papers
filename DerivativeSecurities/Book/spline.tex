\documentclass[11pt,fleqn]{article}
\usepackage{amsmath}

\begin{document}

\title{Splines}
\author{Keith A. Lewis}
\maketitle

\begin{abstract}
Notes describing how to represent and compute piecewise polynomial
univariate functions using basis splines.
\end{abstract}

\section{Newton Form}
Fix a polynomial, $p(x)$, of order $n + 1$, so the highest power of $x$
in $p(x)$ is $n$. Choose $n$ points $\xi_0$, \dots, $\xi_{n-1}$. Then
there exist unique coefficients $a_0$, \dots, $a_n$ such that
$p(x) = a_0 + (x - \xi_0)a_1 + (x - \xi_0)(x - \xi_1)a_1 + \cdots
 + (x - \xi_0)(x - \xi_1)\cdots(x - \xi_{n-1})a_n$.
We can also write this formula as
\begin{equation}
p(x) = \sum_{j=0}^n(x - \xi_0)\cdots(x - \xi_{j-1})a_j.
\end{equation}
The $a_k$ are called the Newton coefficients of the polynomial at $\xi_0$,
\dots, $\xi_{n-1}$.

It is clear that $a_0 = p(\xi_0)$. It is slightly less clear $a_1 =
(p(\xi_1) - p(\xi_0))/(\xi_1 - \xi_0)$, if $\xi_0\not=\xi_1$.  To get
the general algorithm for computing the Newton coefficients in the case
that all $\xi_k$ are distinct, define the 0-th divided difference
to be $[\xi_i]p = p(\xi_i)$, and the first divided difference to be
$[\xi_i,\xi_{i+1}]p = (p(\xi_{i+1}) - p(\xi_i))/(\xi_{i+1} - \xi_i)$.
Define the $k$-th divided difference
\begin{equation}
[\xi_i,\dots,\xi_{i+k}]p = \frac{[\xi_{i+1},\dots,\xi_{i+k}]p
	- [\xi_{i},\dots,\xi_{i+k - 1}]p}{t_{i+k} - t_i}.
\end{equation}
We have $a_k = [\xi_0,\dots,\xi_k]p$.

These coefficients provide an efficient way to compute the value of
the polynomial at a point, $\xi$.  Define $b_n = a_n$ and $b_k = a_k +
(\xi - \xi_k)b_{k+1}$ for $k < n$, then $b_0$ is the value of the
polynomial at $\xi$.  Using a single variable, $b$, instead of $b_k$
in this algorithm is called the Horner method, however the intermediate
values are also interesting.

The $b_k$ are the Newton coefficients of $p$ at $\xi$, $\xi_0$, \dots,
$\xi_{n-2}$, so $p(x) = b_0 + (x - \xi)b_1 + (x - \xi)(x - \xi_0)b_2 +
\cdots + (x - \xi)(x - \xi_0)\cdots(x - \xi_{n - 2})b_n$ for any $x$,
not just $x = \xi$. This can be seen by using the divided difference
formulas: $b_0 = [\xi]p$, $b_1 = [\xi, \xi_0]p$, etc.

Rolle's theorem implies there exists $\zeta$
in the smallest interval containing $\xi_i$, \dots, $\xi_{i+k}$
such that $[\xi_i,\dots,\xi_{i+k}]p = p^{(k)}(\zeta)/k!$.
This leads to the general definition of the divided difference as
\begin{equation}
[\xi_i,\dots,\xi_{i+k}]p =
	\begin{cases}
		p^{(k)}(\xi_i)/k! & \text{if $\xi_i = \cdots = \xi_{i+k}$},\\
		\displaystyle{
		 \frac{[\xi_{i},\dots,\hat{\xi_{s}},\dots,\xi_{i+k}]p
		 - [\xi_{i},\dots,\hat{\xi_{r}},\dots,\xi_{i+k}]p}{t_s - t_r}}
		& \text{if $\xi_r \not= \xi_s$,}
	\end{cases}
\end{equation}
where the hat indicates a missing term. There is no restriction on
the ordering of the $\xi_k$.

For computation it is convenient to define
$p_{i,k} = [\xi_i,\dots,\xi_{i+k}]p$. The formula used in the
generalized Horner method is
$p_{i,k-1} = p_{i+1,k-1} + (t_{i+k} - t_i)p_{i,k}$.

\subsection{Examples}
This section illustrates how to use the Newton form
and divided differences to evaluate a polynomial and its derivatives.

Let $p(x) = 1 + 2x + 3x^2 + 4x^3$ and choose $\xi_0 = \xi_1 = \xi_2 = 0$,
$\xi_3 = 1$.

\section{Representations}
Polynomials can be represented in various ways. Suppose we want to find
the polynomial of order $n$ with $p(\xi_i) = \eta_i$, for $0 \le i \le
n$. The method due to Lagrange is to find polynomials $\lambda_i(x)$
of order $n$ with
$\lambda_i(\xi_j) = \delta_{ij}$, where $\delta_{ij}$ is the Kronecker
delta function. Then clearly $p(x) = \sum_0^n \eta_i\lambda_i(x)$
interpolates the given values.

A formula for the $\lambda_i$ is $\lambda_i(x) = \prod_{j\not=i}(x -
\xi_j)/(\xi_i - \xi_j)$, but this is not efficient. It involves
more computations than the Horner method. (Quantify) It gets even
worse when you want to compute derivatives.

Note $a_0 = [\xi_0]p$, $a_1 = [\xi_0, \xi_1]p$, \dots,
$a_n = [\xi_0, \dots, \xi_n]p$.

\end{document}
