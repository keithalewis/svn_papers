% yc.tex - notes on yield curves
\documentclass[12pt]{amsart}
\usepackage{amsmath}

\begin{document}

\title{Yield Curves}
\date{\today}
\author{Keith A. Lewis}
\address{KALX, LLC}
\email{kal@kalx.net}
\maketitle

%\begin{abstract}
%\end{abstract}

\section{Introduction}

A yield curve is used to compute the cost of moving a fixed payment from
one date to another. A zero coupon bond is the simplest example.
The price of a contract that pays 1 unit at time $t$, the maturity
of the bond, is denoted $D(t)$. Assuming no arbitrage opportunities
exist, $D(0) = 1$ and $D(t)$ is a decreasing function of time, $t$.

Short term zero coupon bonds are called {\em cash deposits}.  They are
quoted in terms of a rate, $r$, using a specific {\em day count basis}
and a maturity $t$. A 1.2\% 1 month cash deposit having actual/360 day
count costs 1 unit and pays $1 + 0.012\delta$ at maturity, where the {\em
day count fraction} $\delta$ is, in this case, the number of days in the
one month period, divided by 360. The day count fraction is approximately
the time in years of the life, or {\em tenor}, of the contract.

The cash deposit rate is related to the discount by
$D(t) = 1/(1 + r\delta)$ so $r = (1/D(t) - 1)/\delta$.

A forward contract is just a forward starting zero coupon bond.
They are quoted in terms of a rate, $f$, using a specific day
count basis and an {\em effective} and {\em termination} date.
The contract is quoted in such a way as to have zero initial
cost. Being long the contract means you will pay $1$ unit
on the effective date, and receive $1 + f\delta$ units on the
termination date, where $\delta$ is the day count fraction
corresponding to the interval of time between the effective
and termination dates.

The forward rate is related to the discount by
$D(u)/D(t) = 1/(1 + f\delta)$ so $f = (D(t)/D(u) - 1)/\delta$.

To avoid large exchanges of notional, forward contracts can be specified
to pay $(f - F)\delta$ at termination, where $F$ is the (uncertain)
rate at time time $t$ over the interval from $t$ to $u$. If the unit
paid at time $t$ on in the original contract is invested at this rate,
it will pay $1 + F\delta$ at time termination. Netting this with the
fixed payment gives the payoff described above.

To describe this using formulas, we need to talk about discount
curves that occur in the future. Let $D_s(t, u)$ denote the
price at time $s$ of a 

There are many kinds of swaps, but a generic swap is just a 
portfolio of back-to-back forward contracts.

Floating leg modelled as two zero coupon bonds.
$F_\delta(t; t_0,\dots,t_n)$ notation.

Constructing a yield curve means finding a discount curve that
reprices a reference set of instruments. A typical reference set would
include cash deposits at ..., forward or futures contracts from 3 months
to 4 years, and swaps at 5, 7, 10, 15, 20, and 30 year maturities.
This problem is underdetermined and there are various techniques to come
up with satisfactory solutions.

Bootstrap...

Interpolating $\log D(t)$ using cubic spline...

Parametric Fit...

Least squares...

\section{Discount, Spot, Forward}

The {\em spot rate}, $r(t)$, is defined by $D(t) = \exp(-tr(t))$ and
the {\em forward rate}, $f(t)$ is defined by $D(t) = \exp(-\int_0^t
f(s)\,ds)$. Since $r(t) = (1/t)\int_0^t f(s)\,ds$, we see that the
spot rate is the average value of the forward rate. One consequence
is that the spot rate curve is smoother than the forward rate curve.
Taking derivatives gives $f(t) = r(t) + tr'(t)$. Note the forward is
equal to the spot when the derivative of the spot vanishes, and
is above/below the spot if the slope of the spot is positive/negative.

There is a geometric technique for finding the forward curve from the
spot curve...

The condition that the discount curve is a decreasing function of
time is equivalent to the condition that the forward curve is
always positive. The constraint on the spot curve is less clear.

It is usually best to work with the forward curve since the noarbitrage
constraint is simple to detect.

\section{Interface}

Typical usage of a yield curve model involves specifying the instruments
and their prices, configuring special characteristics of the yield
curve, then calculating the curve that reprices all instruments.
After this has been done, the discount, spot, and forward rates 
can be obtained. For computing hedges, it is convenient to be able
to ``bump'' the yield curve in various ways and it is desirable
to have this operation not be computationally expensive.

\subsection{Instruments}

Yield curve as a container of instruments. Use string tags to identify
instruments. Valuation vs. settlement date. Settlement conventions.

effective, settlement (T + n), tenor, freq, dcb, roll.

\subsection{Configuration}

Type of construction (bootstrap, parametric fit, etc.)
Convexity adjustment, turn of year effect.

\subsection{Calculation}

Describe bootstrap process. Ridicule least squares method.

\subsection{Bumping}

Bumping instrument coupons. Bumping underlying forward curve directly.

\subsection{State Transitions}

States: init, adding, configable, calcable?

\end{document}
