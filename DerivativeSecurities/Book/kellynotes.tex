\documentclass[11pt,fleqn]{article}
\usepackage{amsmath}

\def\tr{\qopname\relax n{tr}}

\begin{document}

\title{Notes on the Kelly Criterion}
\author{Keith A. Lewis}
\maketitle

\begin{abstract}
Notes on J. L. Kelly, Jr.'s paper
{\it A New Interpretation of Information Rate}
\end{abstract}

\section{The Gambler with a Private Wire}

Kelly considers the following problem.  Suppose you have initial capital
$V_0$ and can bet any fraction on a game that pays even money with
probability $q$ and zero with probability $1 - q$.  How should you bet
your money?

The Kelly criteria suggest betting to maximize your expected gain with
gain defined
as $G = \lim_{N\to\infty}(1/N)\log V_N/V_0$, where $V_N$ is your capital
after $N$ bets. He assumes each time you bet some fraction of your wealth,
$l$. In this case $V_N = V_0 (1 + l)^{W_N} (1 - l)^{N - W_N}$ where
$W_N$ is the number of wins after $N$ bets.

He shows that $l = 2q - 1$ for $q \ge 1/2$ and $l = 0$ for $q < 1/2$.
One criticism of the Kelly Criterion is that it is defined using the
limit as $N$ tends to infinity. In fact, this value for $l$ also maximizes
the gain to time $N$, $G_N = (1/N)\log V_N/V_0$.

We can write $V_1 = V_0 (1 + l)^X (1 - l)^{1 - X}$ where $X$ is a
Bernoulli random variables with $P(X = 1) = q$ and $P(X_j = 0) = 1 -
q$. Note $P(V_1 = V_0(1 + l)) = q$ and $P(V_1 = V_0(1 - l)) = 1 - q$.

In general $V_N = V_0 \prod_{j=1}^N (1 + l)^{X_j} (1 - l)^{1 - X_j}
= V_0 (1 + l)^{W_N}(1 - l)^{N - W_N}$, where $W_N = X_1 + \cdots + X_N$
and the $X_j$ are independent Bernoulli random variables with
$P(X_j = 1) = q$ and $P(X_j = 0) = 1 - q$.  Since $E[W_N] = Nq$, we
have $E[G_N] = q\log(1 + l) + (1 - q)\log(1 - l)$, which is identical
to Kelly's formula for $G$.


One can also bet to maximimize expected return. Define the average return
over $N$ bets as  $R_N = (1/N)\sum_1^N (V_j/V_{j - 1} - 1)$. Recall
that $x - 1 \approx \log x$ if $x \approx 1$. Using this formula on
the summands we have $R_N \approx (1/N)\sum_1^N \log V_j/V_{j - 1} =
(1/N)\log V_N/V_0 = G_N$.

Since $E[(1 + l)^{X_j} (1 - l)^{1 - X_j}] = (1 + l)q + (1 - l)(1 - q)$
we have $E[R_N] = (1 + l)q + (1 - l)(1 - q) - 1 = (2q - 1)l$. If $q >
1/2$ then we should clearly take $l = 1$ in order to maximize the return
and if $q < 1/2$ we should take $l = 0$. This is not a very satisfactory
answer.

Maximizing the expected return seems to ignore how risky the betting
strategy is. 

\end{document}
