% ds.tex - notes on computing delta
\documentclass[12pt,fleqn]{amsart}
\usepackage{amsmath}

\newcommand{\Var}{\mathop{\rm Var}}
\newcommand{\Cov}{\mathop{\rm Cov}}

\newtheorem*{theorem}{Theorem}
\newtheorem*{lemma}{Lemma}

\begin{document}

\title{Delta Skelter}
\author{Keith A. Lewis}
\maketitle

\begin{abstract}
The delta hedge of an option is the derivative of the option value
with respect to the price of the underlying for many models used in
mathematical finance. We show that the standard one period binomial
model commonly taught to beginning MBA students does not belong to
this class. The more general formula that the dollar delta is the
derivative of the future value of the option with respect to return 
is shown to hold for this and several other models.
\end{abstract}

The theory of risk-neutral pricing is based on the notion of delta
hedging. Changes in the price of an option are related to the changes
in the price the instrument underlying the option. The ratio of these
changes is called delta. Holding delta shares of the underlying
guarantees this hedge will match the option price change, under
certain assumptions.

...

\section{One Period Binomial}

The one period binomial model assumes there is a stock with price $s$
at the beginning of a period and that the stock price will be either
$S^+$ with probability $p$ or $S^-$ with probability $1 - p$ at the
end of the period. The only other parameter is the return, $R$, over the
period. This represents a risk-free zero coupon bond having price $1/R$ at
the beginning of the period that pays one unit at the end of the period.
We could write $R = \exp(rt)$, where $t$ is the timespan of the period
and $r$ is the continuously compounded interest rate, but don't.

Options pay some function, $f$, of the final stock price.  Let $V^+ =
f(S^+)$ and $V^- = f(S^-)$. We can find an investment strategy using only
the zero coupon bond and the stock that replicates the option payoff.
Let $m$ be the amount of cash invested in the bond and $n$ the number
of shares purchased at the beginning of the period.  We require that
the value of this porfolio matches the option payoff at the end of the
period, i.e., $mR + nS^+ = V^+$ and $mR + nS^- = V^-$.

In the absence of arbitrage, the value of the option is the cost of the
replicating portfolio, $v = m + ns$. Solving the two linear equations
for the two unknowns yields $n = (V^+ - V^-)/(S^+ - S^-)$, $m
= (S^+V^- - S^-V^+)/R(S^+ - S^-)$, and

\begin{equation}
v = \frac{1}{R}\left(\frac{S^+ - Rs}{S^+ - S^-} V^- 
		+ \frac{Rs - S^-}{S^+ - S^-} V^+\right).
\end{equation}

Note that $\partial v/\partial s = (V^+ - V^-)/(S^+ - S^-) = n$, so for
this model the derivative of the option value with respect to the
underlying stock price does result in the correct delta. It even
looks right, delta is the difference in option value divided by the difference
in underlying value.

Just to get some idea of the typical size of these terms, take $s = 100$,
$R = 1.01$, $S^+ = 103$, and $S^- = 99$. For a call with strike $101$,
$m = -49$, $n = 0.5$, and $v = 0.99$. For a put having the same
strike, $m = 51$, $n = 0.5$, and $v = 0.99$.

This is not the parameterization for the one period binomial model
as commonly taught in beginning MBA courses. (\cite{opb}) Instead of $S^+$
and $S^-$, these parameters are replace with $u$ (up) and $d$ (down)
where $S^+ = su$ and $S^- = sd$. For this model, the formula for the
option value is
\begin{equation}
v = \frac{1}{R}\left(\frac{u - R}{u - d} f(sd) 
		+ \frac{R - d}{u - d} f(su)\right).
\end{equation}

Note that, unlike the previous parameterization, the option
payoff depends on the initial stock price. In this case
\begin{equation*}
\frac{\partial v}{\partial s} = \frac{1}{R}
	\left(\frac{u - R}{u - d} f'(sd)d + \frac{R - d}{u - d} f'(su)u\right)
\end{equation*}
so in general $\partial v/\partial s \not= n$. Note also that if $f$ is
a call or put payoff and the up or down stock price is equal to the
strike, this formula becomes problematic because the derivative does
not exist at the strike.

At this point you may think I am trying to trick you, or that
I am simply wrong. There is no trick and the math is so simple
that you can easily convince yourself it it correct. I was surprised
when I stumbled across this result. How can something so simple
be overlooked?

One possible answer is that nobody uses the one period binomial
model to trade. The multiperiod binomial masks this fact since it is
a reasonable approximation to the classical Black-Scholes method of
option pricing.

It is natural to look for one formula that works for both model
parameterizations.  The formula that recovers the correct dollar delta,
$ns$, in both cases is $\partial (Rv)/\partial R$. The dollar delta (the
amount of money required to purchase the delta hedge) is the derivative
of the future value of the option at expiration with respect to the
return over the period.

In the case of formula (1)
\begin{equation*}
\frac{\partial(Rv)}{\partial R} = \left(\frac{-s}{S^+ - S^-} V^- 
		+ \frac{s}{S^+ - S^-} V^+\right) = \frac{V^+ - V^-}{S^+ - S^-}s.
\end{equation*}
and in the case of formula (2)
\begin{equation*}
\frac{\partial(Rv)}{\partial R} = \left(\frac{-1}{u - d} f(sd) 
		+ \frac{1}{u - d} f(su)\right) = \frac{f(su) - f(sd)}{su - sd}s.
\end{equation*}

Surprisingly, this formula also works in a number of other situations,
including ones outside the domain of the theory of risk-neutral pricing.

\section{One Period Model}

The one period model assumes there is a stock with price $s$ at the
beginning of the period and that the stock price will be a random
variable $S$ at the end of the period. We again denote the return over
the period by $R$ and the option payoff by $V = f(S)$ at the end of the
period. If the stock takes on only two values with nonzero probability,
this reduces to the one period binomial model. In general it is not
possible to replicate the option using cash and the stock, however it
is not unreasonable to try to find $m$ and $n$ that minimize
$E[(mR + nS - V)^2]$.

Taking partial derivative with respect to $m$ and $n$ and setting
these equal to zero gives
\begin{eqnarray*}
E[(mR + nS - V)R] &=& 0\\
E[(mR + nS - V)S] &=& 0
\end{eqnarray*}
so $mR + nE[S] = E[V]$ and $mR + nE[S^2] = E[VS]$. Solving these
two linear equations for the unknowns gives
$n = \Cov(V,S)/\Var(S)$ and $m = (E[V] - nE[S])/R$. The cost of
setting up the optimal portfolio is $v = m + ns = E[V]/R - n(E[S]/R -
s)$.

The formula for the minimum give a measure of how well the
hedge performs. Using the conditions on the partial derivatives
vanishing shows the mininum $E[(mR + nS - V)^2]
= E[(mR + nS - V)(mR + nS - V)] = E[(V - mR + nS)V]$

Assuming $S$ does not depend on $s$, it is simple to see that
the same formula for the delta hedge in the stock holds as in the risk
neutral case. Likewise for the dollar delta.

\section{Appendix}

Recall that a random variable, $X$, having a standard normal distribution
has density function $f(x) = \exp(-x^2/2)/\sqrt{2\pi}$, $-\infty < x <
\infty$ \cite{normal}. In this case the mean of $X$ is zero and the
variance of $X$ is one.  Every normally distributed random variable can
be written as $N = \mu + \sigma X$, where $\mu$ is the mean and $\sigma^2$
is the variance of $N$.

\begin{lemma}
If $N$ is a normally distributed random variable,
then $E[\exp(N)] = \exp(E[N] + \Var(N)/2)$.
\end{lemma}

\begin{proof}
The proof involves completing the square and the substitution
$x\to x + \sigma$.
\begin{eqnarray*}
E[\exp(N)]
 &=& \int_{-\infty}^\infty e^{\mu + \sigma x}
 	e^{-x^2/2}\, dx/\sqrt{2\pi}\\
 &=& e^{\sigma^2/2}\int_{-\infty}^\infty e^\mu
 	e^{-(x - \sigma)^2/2}\, dx/\sqrt{2\pi}\\
 &=& e^{\mu + \sigma^2/2}\int_{-\infty}^\infty
 	e^{-x^2/2}\, dx/\sqrt{2\pi}\\
 &=& e^{\mu + \sigma^2/2}
\end{eqnarray*}
\end{proof}

\begin{lemma}
If $N$ is a normally distributed random variable and $f$ is a function,
then $E[\exp(N)f(N)] = E[\exp(N)]E[f(N + Var(N)]$.
\end{lemma}

\begin{proof}
The proof is essentially the same as in the previous lemma.
Assume $\mu = 0$ for now.
\begin{eqnarray*}
E[\exp(N)f(N)]
 &=& \int_{-\infty}^\infty e^{\sigma x} f(\sigma x)
    e^{-x^2/2}\, dx/\sqrt{2\pi}\\
 &=& e^{\sigma^2/2}\int_{-\infty}^\infty f(\sigma x)
    e^{-(x - \sigma)^2/2}\, dx/\sqrt{2\pi}\\
 &=& e^{\sigma^2/2}\int_{-\infty}^\infty
    f(\sigma(x + \sigma))e^{-x^2/2}\, dx/\sqrt{2\pi}\\
 &=& E[e^N] E[f(N + \sigma^2)]
\end{eqnarray*}
Replacing $N$ by $N + \mu$ and noting $\Var(N) = \Var(N + \mu)$
completes the proof.
\end{proof}

\end{document}
