
\chapter{LIBOR Market Model}

\section{Introduction}
The LIBOR Market Model was developed at Bankers Trust in the early
1990's. It has the advantage of having parameters that correspond closely
to market data. It has the disadvantage that there is no direct formula
for the underlying short rate process.

\section{The Model}
Let $t_0$ be today and $t_1$, $t_2$, be the IMM dates (3rd Wednesday
of months divisible by 3: March, June, September, and December.)
going out to the longest maturity 3 month
Eurodollar futures being used to construct the model. Let $\Phi_j$ be the
`price' at time $t_0$ of the Eurodollar futures expiring at $t_j$,
let $\sigma_j$ be the ATM Black-Scholes volatility corresponding
to the $j$-th Eurodollar futures, and let $\theta_j$ be arbitrary
parameters. The LIBOR Market Model specifies the dynamics of the 3-month
Eurodollar futures: $\Phi_j(t) = \Phi_j \exp(-|\Sigma_j|^2t/2 + \Sigma_j
\cdot B_t)$, where $\Sigma_j = \sigma_j(\cos\theta_j, \sin\theta_j)$,
and $B_t = (B_t^1, B_t^2)$ is 2-dimensional Brownian motion.

The values for $\Phi_j$ are $F(t_0; t_j, t_j + 3months)$ minus the
Eurodollar convexity. (Typically proportional to $t_j^2$.) Note that
for each $j$, $\Phi_j(t)$ is geometric Brownian motion so $\sigma_j$
is just the Black-Scholes implied volatility corresponding to, say,
the ATM caplet corresponding to Eurodollar $j$.

The values of $\theta_j$ are determined by swaptions or mid-curve options
by calibration. A particularly simple parametrization is to assume
$\theta_j = \alpha t_j$ for some $\alpha$ and just fit one swaption
with effective date in the middle of the curve on a swap having
maturity close to the length of the spot curve.

\section{Stochastic Discount}
So how does one compute the stochastic discount?  One method
is $D_{t_j} = D(t_0, t_1)/((1 + \Phi_1(t_1)\delta_1)\cdots (1 +
\Phi_{j-1}(t_{j-1})\delta_{j-1}))$, just use a fixed rate up to $t_1$
and compound the forwards at settlement. It is also not a correct
method.

A better method is to use a Ho-Lee style short rate over $t_0$
to $t_1$ using the same volatility as the front caplet then
bootstrap a yield curve using the forwards at settlement so you
can sample the stochastic discount at any point in time.

This tends to underestimate the volatility.

\section{Extensions}
One possible extension is to permit more general processes than
geometric Brownian motion, such as L\'evy processes.
