\chapter{Derivative Securities} 
\label{ds} 

% how to transfer risky future cash flows to certain cash flows today.
% contract to pay amount equal to price of Goog one year from today
% zero coupon bond is extreme example

% the price of a futures is zero

\section{Introduction}
%A {\em derivative security} is a contract between two counterparties that
%specifies an exchange of cash flows. A {\em cash flow} is a quantity
%of an instrument at a given time. These notes discuss the theory of valuing,
%hedging, and managing the risk of derivative securities.
%
Derivative Securities are tools for shaping risk. Just as with any tool,
they can be used well or poorly, but in today's world they are 
used routinely. Futures, swaps and options on stocks, interest
rates and foreign exchange now trade in huge, liquid markets and are
used as a matter of course by most large companies for managing
their risk.

It didn't used to be that way. Exchanges that match speculators with risk
averse customers have been around for quite some time, but never with
the size, transparency, and speed of execution that currently exists.
The Chicago Board Options Exchange only began trading options on stocks
in 1973.  One of the triumphs of financial engineering in the 70's and
80's was the explosive growth of exchanges and the standardization of many
over-the-counter markets for derivative securities.

%International Securities Exchange traded 4 millon contracts per day
%in 2008.
%http://www.ise.com/assets/documents/AboutISE/PressRelease/CompanyNews/2009/20090102$ISE_Reports_Record_Options_Volume_for_2008.pdf

%In 2007 the Bank for International Settlements reported
%an notional of $xxx trillion in FX options traded.

\section{Definitions}
A {\em derivative security} is a contract between a buyer and
a seller to make a sequence of transactions.
A {\em transaction} is an amount received by the buyer from the seller
and and an amount paid by the buyer to the seller on a specified date.
An {\em amount} is a quantity and unit
where the {\em unit} can be cash in a given currency,
a stock, bond, or commodity. It is also possible to have
derivatives based on an index, or any agreed upon number.
There are derivatives based on the weather, the first such
depended on the temperatures recorded at a particular
weather station in New York's Central Park.

\section{Trading}
Derivative securities can be {\em exchange traded} or {\em
over-the-counter} (OTC). Exchanges anonymously match counterparties
whereas OTC derivatives are contracts directly between two
counterparties. Both employ mechanisms to reduce the risk of the
transactions.

Exchanges use margin accounts to mitigate risk. Every client is required
to post an initial margin. At regular intervals, usually daily, the
value of unwinding all positions under current market conditions is
computed, that is, {\em marked to market}.  As this value changes from the
initial mark to market, the margin is adjusted. The most common method
charges/credits the margin by this difference divided by a constant
factor, the {\em leverage}.

Typical values of leverage range from 5 on single stock futures
to 100 or more on foreign exchange spot contracts.

Marking to market is trivial to do for an exchange. They simply
use the current price of the instrument on the exchange. Since
most products trade many times per day, or even second, there
is little uncertainty about the cost of unwinding the position.

OTC derivatives are usually purchased by companies that wish to reduce
the uncertainty in their future cash flows.  Many OTC derivatives, such as
interest rate swaps, are standardized and quoted with very tight spreads,
however it is not uncommon for sophisticated derivative transactions to
take months to negotiate.

The customer typically posts initial {\em collateral} to the bank
with which they do the transaction, similar to how exchanges require
margin. After executing the contract, market conditions can affect the
value of the cash flows that have not yet been exchanged. Marking to
market usually requires an analytic model developed by the bank and
the rules for collateral adjustments are more involved than those for
exchanges.

\section{Assumptions}
The Black-Merton-Scholes theory relies on market traded instruments to
value and hedge derivative securities.  As with any theory, there are
assumptions that must hold for the theory to apply.  One assumption
is that markets are perfectly liquid: trades occur instantaneously at
exactly the specified price, no matter the quantity. In particular prices
are assumed to be infinitely divisible.  An assumption that is required
to get a unique price is that markets are complete: the derivative can
be replicated by hedging using market instruments. Additionally, the
theory assumes there is a riskless asset that can be bought or sold in
any quantity in order to finance the replicating strategy.

This section discusses how reality varies from the assumptions behind
the theory of Derivative Securities.

\subsection{Spread}
The standard theory assumes trades occur instantaneously at exactly
the specified price, no matter the quantity.  As anyone that has ever
traded knows, prices are quoted at the {\em bid} and {\em ask/offer}
price. When a market for a security exists, {\em market makers} indicate
at what price they are willing to buy the security, the bid, and at what
price they are willing to sell the security, the ask or offer. In the
absence of arbitrage, the bid is less than the ask price, otherwise
a counterparty could buy at the ask, sell at the bid and pocket the
difference.  Market makers make this spread on each matching buy and sell
transaction. The greater their {\em volume} of trades, the more they make.

\subsection{Quantity}
Trades also involve a quantity on number of shares/contracts. Typically,
the larger the quantity a market maker is asked to trade, the wider the
bid/ask spread they will quote in order to lay off the risk involved in
taking on a one side of a large position.

Unlike market makers, exchanges typically match up buyers and sellers and
do not take the opposite side of any trade. They make money based on the
bid/ask spread and perhaps commissions or other fees. They maintain an
{\em order book} containing {\em limit orders} and match them with {\em
market orders}. A limit order is an agreement to buy or sell an instrument
at a given price and quantity. Market orders specify a quantity to sell
(buy) at the highest (lowest) bid (ask) price of the limit orders on
the book.  While market orders get executed within seconds, there is no
guarantee they will get done at the quoted price. Other market orders may
arrive earlier and soak up existing limit orders, or limit orders might
be canceled.  Limit orders are guaranteed to be executed at the specified
level, but there is no guarantee when they will be executed, if ever.

\subsection{Completeness}
As the market aphorism goes, ``Hedge when you can, not when you have
to.''  Another assumption made in the standard theory that is required
to get a unique price is that markets are complete: the derivative can
be replicated by hedging using market instruments. This never happens,
it is just a question of how bad the hedge is.

Unfortunately, the theory does not have much to say about this
question. Monte Carlo simulation for a 3 month, 20\% volatility
at-the-money option with daily rehedging has an average hedging error of
around 4\%. This is much larger than the bid/ask spread in the option
price. Fortunately the hedging errors tend to be normally distributed
and unbiased so when running a large option book there is significant
diversification.

\subsection{Funding}
The assumption there is a riskless asset that can be bought or sold in
any quantity at one price is not true. Typically the rate that traders
are quoted from the funding desk is a short term LIBOR rate plus a spread
of approximately 30 basis points.  There are also fixed limits imposed
by risk management on how much they are allowed to borrow.

%The funding desk in turn uses repurchase agreements to fund the traders.
%[...more about repos...]

If you are using your credit card to fund your trading, then your
so-called risk-free rate is whatever your credit card company charges
you. The standard theory does not allow two different risk-free rates,
but in the real world there are barriers that make it impossible to
exploit the apparent arbitrage.

This is related to the fact that the standard theory does not explicitly
take into account different counterparties having different levels of
access to markets. An institutional investor gets better quotes than a
day trader sitting in front of his computer at home.

\section{Caution}
When Black, Merton, and Scholes derived their famous equation for pricing
options, they used it to identify mispriced options, and all of them lost
money! (Black claims to have lost the least.) Their `mispriced' options
were reflecting the fact that the market had priced in information about
future stock movements that their theory did not take into account.

To his dying day, Fischer Black believed markets were efficient, even
if they were sloppy to within a factor of 2 in pricing.  Recent events
notwithstanding, the price of a futures contract is still zero so at
least one market is efficient.

%Joe Doob's optional stopping theorem is a rigourous version
%of the notion that it is not possible to beat the market.


%\section{Examples}
%
%\subsection{European Options}
%A {\em forward} contract specifies an exchange of underlying
%for a fixed amount, the {\em forward}, at a specific future date, 
%the {\em expiration}.
%
%A {\em futures} is closely related to a forward contract and
%is usually traded on an exchange. Like a forward, a futures 
%has an underlying and an expiration. The exchange guarantees
%that on expiration the futures ``price'' will equal the price
%of the underlying. Prior to that date the ``price'' is
%determined by the market that is provided by the exchange to
%its clients.
%
%The price of a futures contract is always zero. It costs nothing
%to enter the contract. Each day as the quoted ``price'' moves
%your account is credited/debited by the change in the ``price''.
%
%A {\em call option} is the right, but not the obligation to
%exchange the underlying at a fixed amount, the {\em strike}
%at a given expiration.
%
%A {\em digital option} ...
%
%% digital vs binary european vs american
%
%More generally, an option can pay any function of the underlying
%at expiration. We will see later how to combine calls, puts,
%and digital options to reprice any piecewise linear payoff function.
%
%A {\em barrier option} ...
%
%\subsection{American Options}
%
%\subsection{Bermudan Options}
%
%\subsection{Exotic Options}
%
%\subsection{Variance Swaps}
