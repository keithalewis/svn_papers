\documentclass[11pt,fleqn]{amsart}

\newcommand{\R}{\mathbb{R}}
\newcommand{\tr}{\mathop{\rm{tr}}}
\newcommand{\B}{\mathcal B}
\newcommand{\M}{\mathcal M}

\newtheorem{ex}{Exercise}

\begin{document}
\title{Taking derivatives with respect to a matrix}
\author{Keith A. Lewis}
\address{KALX, LLC}
\email{kal@kalx.net}
\urladdr{http://kalx.net}
\begin{abstract}
If we have a function whose argument is a matrix, it is sometimes possible
to take the derivative of the function with respect to the matrix. This
is usually much cleaner than trying to take partial derivatives with
respect to the matrix entries since we only consider one case instead
of a case for each entry in the matrix.
\end{abstract}

\maketitle

\section{Fr\'echet Derivative}
If $X$ and $Y$ are normed vector spaces and $F\colon X \to Y$ is a
function, then the Fr\'echet derivative, $DF\colon X \to \B(X, Y)$,
where $\B(X,Y)$ is the space of bounded linear operators from $X$ to $Y$,
is defined by
\begin{align*}
	||F(x + h) - F(x) - DF(x)h|| = o(||h||),
\end{align*}
if it exists. (The {\em little o} notation, $f(h) = o(h)$ means
$f(h)/h \to 0$ as $h \to 0$.)
If $X$ and $Y$ are are one dimensional, the formula above is equivalent to
\begin{align*}
	\lim_{h\to 0} (F(x + h) - F(x))/h = F'(x)
\end{align*}
The existence of the derivative of $F$ at $x$ says $F$ looks like a linear
function in the neighborhood of $x$. No surprise.

For example, let $X = Y = \M_k$, the space of $k\times k$ matrices, and put
$F(x) = x^2$.  The derivative is not $2x$. Since $(x + h)^2 - x^2
= xh + hx + h^2$, the Frechet derivative is $DF(x)h = xh + hx$. Note
$DF(x)\colon M_k \to \M_k$ and is linear.

If we don't want to mention $F$ explicitly, we can write this as
$d(x^2) = x\,dx + dx\,x$, the non commutative way to write $2x$.
(Note the switch from $D$ to $d$.)

\begin{ex}
Show $d(x^n) = \sum_{j=1}^n x^{n - j} dx x^{j - 1} = x^{n-1}\, dx +
x^{n - 2}\,dx\,x + \cdots + dx\,x^{n - 1}$, the non commutative version
of $nx^{n - 1}$. 
\end{ex}

\begin{ex}
Define $R_x y = xy$ and $L_x y = yx$, and let $F(x) = x^n$.
Show $DF = \sum_{j=1}^n L_x^{n - j}R_x^{j - 1}$.
\end{ex}

A great example of the power of the Frechet derivative is computing
the maximum likelihood estimator for the covariance of a $d$-dimensional
brownian motion given ($d$-dimensional) observations $x_i$ at times $t_i$,
$i = 1,\dots,n$.

The function of interest is $F(\Lambda) = n\log\det \Lambda - \sum_i
y^T_i \Lambda y_i$, where $y_i = (x_i - x_{i-1})/\sqrt(t_i - t_{i-1})$,
and $\Lambda$ is the inverse of the covariance matrix. Here, $y^T$
indicates the transpose of $y$. The problem is to find the matrix,
$\Lambda$, that maximizes $F(\Lambda)$. Just like in one-dimensional
calculus, we take the derivative and set it equal to zero.

We need the fact that $\det(\exp A) = \exp(\tr A)$. This is clearly
true if $A$ is a diagonal matrix, and less clearly true if $A$ is upper
triangular.  Once you convince yourself of this fact, this proves the
general case since all matrices have an upper triangular representation.

If $B$ is invertible we have $\det B = \exp(\tr\log B)$. Taking
derivatives gives $d(\det B) = \exp(\tr\log B)\tr (B^{-1} dB) = (\det
B)\tr (B^{-1} dB)$.

\begin{ex}
Show $d(\exp A) = (\exp A) dA$ if $A$ and $dA$ commute, and
$d(\log B) = B^{-1} dB$, if $B$ and $dB$ commute, using the definition
of the Fr\'echet derivative and the power series representations
for $\exp$ and $\log$.
\end{ex}

We now see $dF(\Lambda) = (n/(\det \Lambda)) (\det \Lambda) \Lambda^{-1}
d\Lambda - \sum_i y_i^T d\Lambda y_i = n \Lambda^{-1} d\Lambda - \sum_i
y_i^T d\Lambda y_i$. Setting this to zero and taking $d\Lambda = e_j e'_k$
where the $e_j$'s are the standard basis of $\R^n$, we get $\Lambda_{j,
k} = (1/n)\sum_i (y_i)_j (y_i)_k$. You need to know $\tr x y^T = x^T y$.

This can be written in matrix form as $\Lambda = (1/n)\sum_i y_i y^T_i$.

\end{document}
