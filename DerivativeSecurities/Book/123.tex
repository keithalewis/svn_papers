\chapter{Simple Examples}

This chapter describes a computationally simple one period model
that illustrates some features and drawbacks of such a simple model.

\begin{section}{1-2-3 Example}

Suppose a stock has price $1$ at the beginning of a period and will have
price $3$ with probability $0.9$ and price $1$ with probability $0.1$
at the end of the period. Suppose further a bond has price $1$ at the
beginning of the period and price $2$ with certainty at the end of
the period. What is the price of a call option on the stock having
strike $2$ expiring at the end of the period?
\end{section}

The option can be replicated for a cost of $0.25$. Borrow $0.25$ and
buy $0.5$ shares of stock for $0.5$. At the end
of the period, if the stock price is $3$ then sell the half share
for $1.5$. Use $0.5$ to pay off the bond and $1$ to pay off the
option that is now in the money. If the stock price stays at $1$
then sell the half share for $0.5$ to pay off the bond. The option
is out of the money.

The discounted expected payoff of the option is
is $(1/2)(1\times 0.9 + 0\times 0.1) = 0.45$. 
Note $(1/2)(3\times 0.9 + 1\times 0.1) = 1.4 \not= 1$
so this method does not reprice the stock. However
$(1/2)(3\times 0.5 + 1\times 0.5) = 1$ reprices the
stock. Using the same probabilities we have
$(1/2)(1\times 0.5 + 0\times 0.5) = 0.25$, the correct cost
for replicating the option using the stock and bond.

\begin{section}{One period Trinomial Example}

Suppose a stock has price $100$ at the beginning of a period and
can have price $90$, $100$, or $110$ at the end of the period. Suppose
the interest rate is zero over the period. If a call with strike
$100$ has price $6$ then an arbitrage exists.

Borrow $90$, buy one share of stock, and sell two options. The initial
cost is $-90 + 100 + 2\times 6 = 2$. Put that money in your pocket
because you will get to keep it. If the stock is $90$ at the end
of the period, the value of your position is $-90 + 90 - 2\times 0 = 0$,
if the stock is $100$ the value is $-90 + 100 - 2\times 0 = 10$,
if the stock is $110$ the value is $-90 + 110 - 2\times 10 = 0$.
In no case will you lose money.

All solutions of $100 = 90p_1 + 100p_2 + 110p_3$ with $p_j \ge 0$
and $p_1 + p_2 + p3$ have the form $p_1 = p$, $p_2 = 1 - 2p$,
$p_3 = p$, with $0\le p \le 0.5$, and the expected payoff on
the call is therefore $10p$.

\end{section}

\begin{section}{Notes}
The 1-2-3 example is a question I have posed over the years when interviewing
quant candidates as a test of whether
or not they understood the difference between
`real world' and `risk-neutral' probabilities. 
As Mathematical Finance programs became popular
all candidates were able to produce the correct answer.
I hated to see such a good question go to waste,
so I decided to change the answer. 

If you show this problem to a trader they would probably tell you the
option is underpriced at 0.25 and simply buy it all day long. If
they don't hedge, the will make 0.90 on average, discounted by
$1/2$. which works out to making $0.45 - 0.25$ on average. 

\end{section}