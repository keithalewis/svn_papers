\section{The One Period Model}
The one period model is described by a vector, $x\in\R^m$, representing
the prices of $m$ instruments at the beginning of the period, a set
$\Omega$ of all possible outcomes over the period, and a bounded function
$X\colon\Omega\to\R^m$, representing the prices of the $m$ instruments
at the end of the period depending on the outcome, $\omega\in\Omega$.

\begin{definition}
Arbitrage exists if there is a vector $\xi\in\R^m$ such that
$\xi\cdot x < 0$ and $\xi\cdot X(\omega)\ge0$ for all $\omega\in\Omega$.
\end{definition}

The cost of setting up the position $\xi$ is $\xi\cdot x = \xi_1 x_1 +
\cdots + \xi_m x_m$. This being negative means money is made by putting on
the position. When the position is liquidated at the end of the period,
the proceeds are $\xi\cdot X$. This being non-negative means no money
is lost.

It is standard in the literature to introduce an arbitrary probability
measure on $\Omega$ and use the conditions $\xi\cdot x = 0$ and $\xi\cdot
X\ge0$ with $E[\xi\cdot X]>0$ to define an arbitrage opportunity, e.g.,
\cite{ShiKabKraMel1994} section 7.3, definition 1.  Making nothing
when setting up a position and having a nonzero probability of making a
positive amount of money with no estimate of either the probability or
amount of money to be made is not a realistic definition of an arbitrage
opportunity.  Traders want to know how much money they make up-front
with no risk of loss after the trade is put on.  %(See remarks...)
This is what Garman calls {\em strong arbitrage} \cite{Gar1985}.

Define the \emph{realized return} for a position, $\xi$, by
$R_\xi = \xi\cdot X/\xi\cdot x$,
whenever $\xi\cdot x\not=0$. If there exists $\zeta\in\R^m$
with $\zeta\cdot X(\omega) = 1$ for $\omega\in\Omega$ (a 
zero coupon bond) then the price is $\zeta\cdot x = 1/R_\zeta$.
Zero interest rates correspond to a realized return of 1.

Note that arbitrage is equivalent to
the condition $R_\xi < 0$ on $\Omega$ for some $\xi\in\R^m$.
In particular, negative interest rates do not necessarily imply arbitrage. 

The set of all arbitrages form a cone since this set is closed
under multiplication by a positive scalar and addition. The following 
version of the FTAP shows how to compute an arbitrage when it exists.

\begin{theorem}{(One Period Fundamental Theorem of Asset Pricing)}
Arbitrage exists if and only if $x$ does not belong to the smallest
closed cone containing the range of $X$. If $x^*$ is the nearest point in
the cone to $x$, then $\xi = x^* - x$ is an arbitrage.
\end{theorem}

\begin{proof}
If $x$ belongs to the cone, it is arbitrarily close to a finite sum
$\sum_j X(\omega_j) \pi_j$, where $\omega_j\in\Omega$ and $\pi_j > 0$ for
all $j$. If $\xi\cdot X(\omega) \ge0$ for all $\omega\in\Omega$ then $\xi\cdot
\sum X(\omega_j) \pi_j \ge0$, hence $\xi\cdot x$ cannot be negative. The
other direction is a consequence of the following with $\CC$ being the
smallest closed cone containing $X(\Omega)$.
\end{proof}

\begin{lemma}
If $\CC\subset\R^m$ is a closed cone and $x\not\in \CC$, then there
exists $\xi\in\R^m$ such that $\xi\cdot x < 0$ and $\xi\cdot y \ge0$
for all $y\in \CC$.
\end{lemma}

\begin{proof}
This result is well known, but we provide an elementary self-contained
proof.  Since $\CC$ is closed and convex, there exists $x^*\in \CC$
such that $\norm{x^* - x} \le \norm{y - x}$ for all $y\in \CC$.
We have $\norm{x^* - x} \le \norm{tx^* - x}$ for $t \ge 0$, so $0 \le
(t^2 - 1)\norm{x^*}^2 - 2(t - 1)x^{*}\cdot x = f(t)$. Because $f(t)$
is quadratic in $t$ and vanishes at $t = 1$, we have $0 = f'(1) =
2\norm{x^*}^2 - 2x^{*}\cdot x$, hence $\xi\cdot x^* = 0$.  Now $0 <
\norm{\xi}^2 = \xi\cdot x^* - \xi\cdot x$, so $\xi\cdot x < 0$.

Since $\norm{x^* - x} \le \norm{ty + x^* - x}$ for $t \ge 0$ and $y\in
\CC$, we have $0 \le t^2\norm{y}^2 + 2ty\cdot(x^* - x)$. Dividing by $t$
and setting $t = 0$ shows $\xi\cdot y \ge0$.  \end{proof}

Let $B(\Omega)$ be the Banach algebra of bounded real-valued functions on
$\Omega$. Its dual, $B(\Omega)^* = ba(\Omega)$, is the space of finitely
additive measures on $\Omega$ \cite{DunSch1954}.  If $\PP$ is the set of
non-negative measures in $ba(\Omega)$, then $\{\pair{X,\Pi}:\Pi\in\PP\}$
is the smallest closed cone containing the range of $X$,
where the angle brackets indicate the dual pairing.
There is no
arbitrage if and only if there exists a non-negative finitely additive
measure, $\Pi$, on $\Omega$ such that $x = \langle X,\Pi\rangle$,

If $V\in B(\Omega)$ is the payoff function of an instrument and $V =
\xi\cdot X$ for some $\xi\in\R^m$, then the cost of replicating the payoff
is $\xi\cdot x = \langle \xi\cdot X,\Pi\rangle = \langle V,\Pi\rangle$.
Of course the dimension of such perfectly replicating payoff functions
can be at most $m$.

If a zero coupon bond, $\zeta\in\R^m$, exists then the riskless realized
return is $R = R_\zeta = 1/\Pi(\Omega)$. If we let $P = \Pi R$, then $P$
is a probability measure and $x = \langle X/R,P\rangle = E[X]/R$.

% R^m/{R_\zeta = 1} convex set and separating hyperplane

\subsection{Examples}
This section illustrates the one period model with some examples.

\begin{example}{(Forward)}
Let $x = (1, s, 0)$, $\Omega = [0,\infty)$,
and $X(\omega) = (R, \omega, \omega - f)$.
\end{example}
This models a bond with riskless realized
return $R$, a stock, and a forward contract on the stock with forward $f$.
The smallest cone containing the range of $X$ is spanned
by $X(0) = (R, 0, -f)$ and $\lim_{\omega\to\infty}X(\omega)/\omega = (0, 1, 1)$.
Solving $(1, s) = a(R, 0) + b(0, 1)$ gives $a = 1/R$ and $b = s$.
This implies $0 = -f/R + s$ so $f = Rs$, the standard
relationship between forward and spot prices in the one period case. 
\cite{Hul2005}

\begin{example}{(Standard Binomial Model)}
Let $x = (1, s, v)$, $\Omega = \{d, u\}$, $0<d<u$.
and $X(\omega) = (R, s\omega, V(s\omega))$, where $V$ is any given function.
\end{example}
This is the usual parameterization for the one period binomial model with
a riskless bond having realized return $R$, and a stock having price $s$
that can go to either $sd$ or $su$.  The smallest cone containing the
range of $X$ is spanned by $X(d)$ and $X(u)$.  Solving $(1, s) = aX(d)
+ bX(u)$ for $a$ and $b$ yields $a = (R - d)/R(u - d)$ and $b = (u -
R)/R(u - d)$. The condition that $a$ and $b$ are nonnegative implies $d
\le R \le u$. The no arbitrage condition on the third component implies
\begin{equation*}
  v = \frac{1}{R}\left(\frac{u - R}{u - d} V(sd)
    + \frac{R - d}{u - d} V(su)\right).
\end{equation*}
In a binomial model, the option is a linear combination of the bond
and stock.
Solving $V(sd) = mR + nsd$ and $V(su) = mR + nsu$ for $n$ we see
the number of shares of stock to purchase in order to replicate the option
is
$n = (V(su) - V(sd))/(su - sd)$. 
Note that if $V$ is a call spread consisting of long one call with strike
slightly greater than $sd$ and short one call with strike slightly less
than $su$, then $\partial v/\partial s = 0$ since $V'(sd) = 0 = V'(su)$.

\begin{example}{(Binomial Model)}
Let $x = (1, s, v)$, $\Omega = \{S^+, S^-\}$, and $X(\omega) = (R,
\omega, V(\omega))$, where $V$ is any given function.
\end{example}
As above we find
\begin{equation*}
	v = \frac{1}{R}\left(\frac{S^+ - Rs}{S^+ - S^-} V(S^-) 
		+ \frac{Rs - S^-}{S^+ - S^-} V(S^+)\right)
\end{equation*}
and the number of shares of stock required to replicate the option is
$n = (V(S^+) - V(S^-)/(S^+ - S^-)$. Note $\partial v/\partial s = n$
indicates the number of stock shares to buy in order to replicate
the option.

\begin{example}
Let $x = (1, 100, 6)$, $\Omega = [90,110]$,
and $X(\omega) = (1, \omega, \max\{\omega - 100, 0\})$.
\end{example}
This corresponds to
zero interest rate, a stock having price 100 that will certainly end with
a price in the range 90 to 110, and a call with strike 100. One might
think the call could have any price between 0 are 10 without entailing
arbitrage, but that is not the case.

This model is not arbitrage free.  The smallest cone containing the
range of $X$ is spanned by $X(90)$, $X(100)$, and $X(110)$. It is easy
to see that $x$ does not belong to this cone since it lies above the
plane determined by the origin, $X(90)$ and $X(110)$.

Using $e_b$, $e_s$, and $e_c$ as unit vectors in the bond, stock, and call
directions, $X(90) = e_b + 90e_s$ and $X(110) = e_b + 110e_s + 10e_c$.
Grassman calculus \cite{Pea1999} yields $X(110)\wedge X(90)
= 90 e_b\wedge e_s + 110e_s\wedge e_b + 10e_c\wedge e_b + 900e_c\wedge e_s
= -900e_s\wedge e_c + 10e_c\wedge e_b - 20e_b\wedge e_s$. The vector
perpendicular to this is $-900e_b + 10e_s - 20e_c$.

After dividing by 10, we can read off an arbitrage from this: borrow 90
using the bond, buy one share of stock, and sell two calls. The amount
made by putting on this position is $-\xi\cdot x = 90 - 100 + 12 = 2$. At
expiration the position will be liquidated to pay $\xi\cdot X(\omega)
= -90 + \omega - 2\max\{\omega - 100, 0\} = 10 - |100 - \omega| \ge 0$
for $90\le\omega\le 110$.

\begin{example}
Let $x = (100, 9.1)$, $\Omega = [90,110]$,
and $X(\omega) = (\omega, \max\{\omega - 100, 0\})$.
\end{example}

Eliminating the bond does not imply the call can have any price
between 0 and 10 without arbitrage. The position
$\xi = (1, -11)$ is an arbitrage.
% multi period - gains must be increasing V(0) <= 0 , V(T) > 0 does not
% tell you what the drawdown is.

\subsection{An Alternate Proof}
The preceding proof of the fundamental theorem of asset pricing
does not generalized to multi-period models so we
give a proof here that does. 

Define $A\colon\R^m\to \R\oplus B(\Omega)$ by $A\xi = -\xi\cdot x \oplus
\xi\cdot X$.  This linear operator represents the account statements
that would result from putting on the position $\xi$ at the beginning
of the period and taking it off at the end of the period.  Define $\PP$
to be the set of $p \oplus P$ where $p > 0$ is in $\R$ and $P \ge 0$
is in $B(\Omega)$.  Arbitrage exists if and only if $\ran A = \{A\xi
: \xi\in \R^m\}$ meets $\PP$.  If the intersection is empty, then by
the Hahn-Banach theorem \cite{BanMaz1933} there exists a hyperplane $\HH$
containing $\ran A$ that does not intersect $\PP$. Since we are working
with the norm topology, clearly $1\oplus 1$ is the center of an open
ball contained in $\PP$, so the theorem applies.
The hyperplane consist of all $y\oplus Y\in \R\oplus
B(\Omega)$ such that $0 = y\pi + \langle Y,\Pi\rangle$ for some
$\pi\oplus\Pi\in\R\oplus ba(\Omega)$.

First note that $\langle \PP, \pi\oplus\Pi\rangle$ cannot contain
both positive and negative values. If it did, the convexity of $\PP$
would imply there is a point at which the dual pair is zero and thereby meet
$\HH$. We may assume that the dual pair is always positive and that $\pi
= 1$.  Since $0 = \langle A\xi, \pi\oplus\Pi \rangle = \langle -\xi\cdot
x, \pi\rangle + \langle \xi\cdot X, \Pi \rangle$ for all $\xi\in\R^m$
it follows $x = \langle X,\Pi\rangle$ for the nonnegative measure
$\Pi$. This completes the alternate proof.

This proof does not yield the arbitrage vector when it exists, however
it can be modified to do so. Define $\PP^+ = \{\pi\oplus\Pi : \langle
p\oplus P,\pi\oplus\Pi\rangle > 0, p\oplus P\in\PP\}$. The Hahn-Banach
theorem implies $\ran A\cap\PP \not=\emptyset$ if and only if $\ker
A^*\cap\PP^+ = \emptyset$, where $A^*$ is the adjoint of $A$ and
$\ker A^* = \{\pi\oplus\Pi : A^*(\pi\oplus\Pi) = 0\}$. If the later
holds we know $0 < \inf_{\Pi\ge0} \norm{-x + \langle X,\Pi\rangle}$
since $A^*(\pi\oplus\Pi) = -x\pi + \langle X,\Pi\rangle$. The same
technique as in the first proof can now be applied.

\bibliographystyle{plain}
\bibliography{ftapd}{}
