\chapter{Grassman's Geometric w} 
\label{ga}

It took nearly two millenia from the time of the geometric machinery
developed by Euclid until Descartes invented analytic geometry by
introducing coordinates to reduce geometric problems to algebra. Some
two hundred years later Hermann Grassmann introduced the next major
breakthrough by using coordinate-free algebra to conquer geometric
questions. For some reason, his theory is less well known, although far
more powerful.

During his time, non-Euclidean geometries were being considered, but
his theory was about Euclidean space. This is different from what we now
call a vector space in that there is no origin, simply points in space.
His notion was that if $P$ and $Q$ are points in space, $PQ$ represents
the line from $P$ to $Q$. If $R$ is a third point, $PQR$ represents the
plane determined by the three points, and so on. His theory worked in any
number of dimensions, which was quite revolutionary at the time. One
triumph of his theory was that he (and Clifford) were able to fit
Hamilton's 4-dimensional quaternions into a more general framework.

In Grassman's algebra, $P + Q$ is the midpoint between $P$ and $Q$ and
has a weight of 2. More generally, $aP + bQ$ has weight $a + b$ and is
the point at which a weight of $a$ at $P$ and a weight of $b$ at $Q$ would
balance on the line determined by $P$ and $Q$. The fundamental assumption
Grassmann made was that $PQ = 0$ if and only if $P = Q$. In his
algebra of points, the first consequence is that $PQ
= -QP$ since $0 = (P + Q)(P + Q) = PP + PQ + QP + QQ = PQ + QP$. This
incidence rule generalizes to higher dimensions.  $PQR = 0$ if and only
if $P$, $Q$, and $R$ lie on the same line, and so on.

One easy geometric consequence is that the medians of the triangle $PQR$
meet at a point.  $P + Q$ is the midpoint between $P$ and $Q$. $(P +
Q)R$ is the line from the midpoint to $R$. $P + Q + R$ is where all
the lines from the midpoints intersect because $P(Q + R)(P + Q + R) =
(PQ + PR)(P + Q + R) = PQP + PQQ + PQR + PRP + PRQ + RPR = PQR + PRQ =
P(QR + RQ) = 0$. Same story for $Q(R + P)$ and $R(P + Q)$.

Let $R(t) = (1 - t)P + tQ = P + t(Q - P)$, where $t$ is a scalar.
Clearly $R(0) = P$ and $R(1) = Q$. Also, $R(1/2)$ is the midpoint
between $P$ and $Q$ with weight 1.  If we interpret $Q - P$ as the vector
from $P$ to $Q$, we can recognize $R(t)$ as the instruction ``starting
from the point $P$, move $t$ multiples in the direction $P$ to $Q$.''
The points $R(t)$ for $t$ scalar are the points on the line determined
by $P$ and $Q$.

Note $R(t) \not= Q - P$ for any value of $t$ since $Q - P$ has weight
zero. Such `points' are called {\em vectors} and can be thought of
as $\lim_{t\to\infty}R(t)$.

\section{Dimension One}
Points are zero dimensional. Products of points are one dimensional.
$PQ = RS$ if and only if there exists a scalar $t$ such that $R = P +
t(Q - P)$ and $S = Q + t(Q - P)$, that is, the line segment from $R$
to $S$ is a translation of the segment from $P$ to $Q$ along the line
determined by $P$ and $Q$. Note $R$ and $S$ are on the line
determined by $P$ and $Q$ since $PQR = RSR = 0$ and $PQS = RSS = 0$.
Hence, there exist scalars $t$ and $u$ such that $R = R(t)$ and $S =
R(u)$.  Since $PQ = R(t)R(u) = (1-t)uPQ + t(1-u)QP = (u - t)PQ$ we have
$u - t = 1$ so $S = R(u) = R(t + 1) = -tP + (t + 1)Q = Q + t(Q - P)$.

\section{Dimension Two}
The product of of three points is two dimensional. $PQR = STU$ if
and only if there are scalars $t$ and $u$ such that
$S = R(t)$, $T = R(u)$ and ...

\section{Clifford Algebras}
His rule for vector products was that $vv$ was a real number
for any vector $v$ and defined $v\cdot w = (vw + wv)/2$ and
$v\wedge w = (vw - wv)/2$ so $vw = v\cdot w + v\wedge w$.
