\documentclass[11pt,fleqn]{amsart}
\usepackage{amsmath}

\begin{document}

\title{Variance Swaps: Discrete Model}
\author{Keith A. Lewis}

\begin{abstract}
This note describes a method for computing par variance rates using
only option data at discrete strikes. Upper and lower bounds for the
error introduced by discretization are derived assuming
flat extrapolation of implied volatility.
\end{abstract}

\maketitle

The value of the floating leg of a variance swap is
\begin{equation*}
	\sigma^2 = \frac{1}{t}e^{-rt} E[-2\log X/f + 2(X - f)/z],
\end{equation*}
where $t$ is the time in years to expiration, $r$ is the continuously
compounded risk-free rate, $X$ is the price of the
underlying at expiration under the risk-neutral measure and $f = E[X]$
is the forward. This implies the value of $z$, the expansion point,
may be chosen arbitraily. If we let $\phi(x) = \phi(x, f, z)
= -2\log x/f + 2(x - f)/z$, then $\phi'(x) = -2/x + 2/z$ and
$\phi''(x) = 2/x^2$. It is clear that $\phi(x)$ is concave up,
and has a single minimum at $x = z$ at which it is negative unless
$z = f$.

The discrete model uses the prices of options at $k_0$,\dots,$k_n$, a
low secant point $l < k_0$ and a high secant point $h > k_n$ to determine
the value of the variance rate par volatility, $\sigma$.

The Bloomberg implementation extimates the forward and at-the-money
option value from option data and adds that point to the data used to
fit the curve. It also assumes $z = f$, i.e., the expansion point equals
the forward. From this data, the weights $w_0$, \dots, $w_n$ are chosen
such that

\begin{equation*}
	\psi(x) = \sum_{k_j < f} w_j \max\{k_j - x, 0\}
		+ \sum_{k_j \ge f} w_j \max\{x - k_j, 0\}
\end{equation*}
matchs $\phi$ at all strikes and at both secant points. The weight at
strike $k_j$ is the slope of the segment to its immediate right minus
the slope of the segment to its immediate left. Note that there are
no options corresponding to the secant points. Adjusting these values
result in the weight of the low strike put and high strike call changing.

Note $\psi(x) > \phi(x)$ for $l < x < h$ and $\psi(x) > \phi(x)$ if
$x < l$ or $x > h$. For coarse stikes, this method can overvalue the
variance rate. If the range of strikes is not sufficiently wide, this
 implementation can undervalue the variance rate. Typically the high
strike put data is not problematic since the payoff function, $\phi(x)$
 is nearly linear for large $x$. Because of the singularity at zero, the
variance value will go to infinity as the low secant point tends to zero.

If we parameterize the low secant point by $l = \lambda k_0$ for
$0 < \lambda < 1$, then 

The weight added to the low strike put
beyond it's weight for $l = k_0$ is
\begin{eqnarray*}
	\frac{\phi(k_0) - \phi(l))}{k_0 - l} - \phi'(k_0) 
		&=& -2\frac{\log k_0/l}{k_0 - l} + \frac{2}{k_0} \\
		&=& \frac{2}{k_0} (1 + \frac{\log\lambda}{1 - \lambda})  \\
\end{eqnarray*}
where $\lambda = l/k_0$ is called the {\em low ratio}.

We use the fact that
\begin{equation*}
	e^{-rt} E[-2\log X/f] = \int_0^f (2/k^2) p(k)\,dk
		+ \int_f^\infty (2/k^2) c(k)\,dk,
\end{equation*}
where $p(k)$ is the value of a put having strike $k$ and
$c(k)$ is the value of a call having strike $k$.

In practice, the values of $p(k)$ and $c(k)$ are only known at a finite
number of strikes. Consider the first term on the right hand side of
the previous equation.  Assume $p(k)$ is given for $k$ having values
$k_0 < k_1 < \cdots < k_n = f$.  We assume that the implied volatility,
$\sigma_0$, is constant for $0 < k < k_0$. There is no need to assume it
is equal to the implied volatility of the put having the lowest strike.
Using $X = f\exp(- \sigma_0^2 t/2 + \sigma_0\sqrt{t} Z)$, where $Z$ is
a standard normal random variable, we can find a closed form solution
for the integral from zero to $k_0$.  Using integration by parts twice

\begin{eqnarray*}
	\int_0^{k_0} (2/k^2) p(k)\,dk
		&=& -2p(k_0)/k_0 + \int_0^{k_0} (2/k) p'(k)\,dk \\
		&=& -2p(k_0)/k_0 + (2\log k_0) p'(k_0) \\
		&\quad&\quad - \int_0^{k_0} (2\log k) p''(k)\,dk \\
		&=& -2p(k_0)/k_0 + e^{-rt}(2\log k_0) P(X \le k_0) \\
		&\quad&\quad -e^{-rt}E[2\log X 1(X \le k_0)].\\
\end{eqnarray*}
The last equation uses the fact that $p'(k) = e^{-rt}P(X \le k)$ and
so $e^{rt}p''(k)$ is the density function of the random variable $X$.

The closed form for the last expression can be found
from the fact that $E[Z 1(Z \le z)] = -\exp(-z^2/2)/\sqrt{2\pi}$
if $Z$ is a standard normal.
\begin{eqnarray*}
	E[\log X 1(X \le k_0)] &=& (\log f - \sigma_0^2 t/2) P(Z \le z_0)
		- \sigma_0\sqrt{t/2\pi} e^{-z_0^2/2},
\end{eqnarray*}
where $z_0 = (\log k_0/f + \sigma_0^2 t/2)/\sigma_0\sqrt{t}$.
Combining these results we have
\begin{eqnarray*}
	\int_0^{k_0} (2/k^2) p(k)\,dk
	&=& -2p(k_0)/k_0 + e^{-rt}(2\log k_0/f + \sigma_0^2 t) P(X \le k_0) \\
	&\quad&\quad + 2e^{-rt}\sigma_0\sqrt{t/2\pi} e^{-z_0^2/2} \\
\end{eqnarray*}

We turn to the integral from $k_0$ to $f$.  Using integration
by parts with constant of integration $2/f$,
\begin{eqnarray*}
\int_{k_0}^f (2/k^2) p(k)\,dk
	&=& (2/k_0 - 2/f) p(k_0)
		+ \int_{k_0}^f (2/k - 2/f) p'(k)\,dk\\
	&=& (2/k_0 - 2/f) p(k_0)
		+ \sum_{j < n} (2/k_j^* - 2/f) (p(k_{j+1}) - p(k_j)),\\
\end{eqnarray*}
for some $k_j^*$'s with $k_j \le k_j^* \le k_{j + 1}$ by the
mean value theorem for Riemann-Steiltjes integrals.
Since the integrand is monotone we can get error bounds
by evaluating at the end points of each interval.

Similar analysis can be performed on the integral from $f$ to infinity.
Assume $c(k)$ is given at $k$ having values $k_0 = f < \cdots < k_n$.

\begin{eqnarray*}
\int_f^{k_n} (2/k^2) c(k)\,dk
	&=& \int_f^{k_n} (2/k - 2/f) c'(k)\,dk \\
	&=& \sum_{j < n} (2/k_j^* - 2/f) (c(k_{j+1}) - c(k_j))\\
\end{eqnarray*}
for some $k_j^*$'s with $k_j \le k_j^* \le k_{j + 1}$. As before
we can get upper and lower bounds by evaluating the integrand
at the endpoints of each interval.

\begin{eqnarray*}
\int_{k_n}^\infty (2/k^2) c(k)\,dk
	&=& 2c(k_n)/k_n + \int_{k_n}^\infty (2/k) c'(k)\,dk \\
	&=& 2c(k_n)/k_n - (2\log k_n) c'(k_n) \\
	&\quad&\quad - \int_{k_n}^\infty (2\log k) c''(k)\,dk \\
	&=& 2c(k_n)/k_n + e^{-rt}(2\log k_n) P(X \ge k_n) \\
	&\quad&\quad -e^{-rt}E[2\log X 1(X \le k_n)].\\
\end{eqnarray*}

We assume the implied volatility, $\sigma_n$, is constant for strikes
greater than $k_n$ and equal to the implied volatility of the call
at strike $k_n$. We get a closed form solution for the
last term above. Since $E[Z 1(Z > z)] = \exp(-z^2/2)/\sqrt{2\pi}$,

\begin{eqnarray*}
	E[\log X 1(X \ge k_n)] &=& (\log f - \sigma_n^2 t/2) P(Z \ge z_n)
		+ \sigma_n\sqrt{t/2\pi} e^{-z_n^2/2}, \\
\end{eqnarray*}
where $z_n = (\log k_n/f + \sigma_n^2 t/2)/\sigma_n\sqrt{t}$.

Combining these results we have
\begin{eqnarray*}
	\int_{k_n}^\infty (2/k^2) c(k)\,dk
	&=& 2c(k_n)/k_n + e^{-rt}(2\log k_n/f + \sigma_n^2 t) P(X \ge k_n) \\
	&\quad&\quad - 2e^{-rt}\sigma_n\sqrt{t/2\pi} e^{-z_n^2/2} \\
\end{eqnarray*}

\end{document}
