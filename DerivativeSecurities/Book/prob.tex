% use E, F, ... for events!!!
% move new commands/the
\chapter{Probability Theory}

\begin{section}{Introduction}

Probability Theory had its origins in games of
chance and was not considered to be a part of mainstream mathematics until
fairly recently. It can be thought of as a generalization of the Propositional
Calculus for logic that involves combining statements (`propositions') that 
can be either `true' or `false' using the operations `and', `or', `not', etc.
George Boole \cite{Boole:1853} proposed a set of laws that propositions satisfy that
could be used to reduce (a very limited subset of) logical reasoning to algebraic 
manipulation. The notion human reasoning could be reduced to a mechanical computation
was quite stunning at the time. 

Probability theory deals with `events' that have a `probability' between zero 
and one. For events having probability zero or one, it reduces to the Propositional
Calculus. There is not uniform agreement on what `probability' means.  Is it
subjective or objective? Some schools interpret it as a frequency in order to give
it an objective meaning, but modern thinking is that this is too restrictive.
Bayesians consider it to be subjective and that it represents a
degree of belief conditional on available information. To reconcile the 
unsatisfactory notion of everyone having their own personal (`prior') probabilities, 
they claim that repeated application of Bayes' Theorem involving increasing
information would result in convergence of posterior probabilities.

% (\cite{Kolmogorov:1933},\cite{Kolmogorov:1950}).

%Is a probability a frequency? Is a probablity objective or subjective?

%Emile Borel - probability is subjective

%1923 Hausdorff - event is a measurable subset from a Borel field. (Stone's Theorem)

% complexity and probability

Schematically, given data $x_1$, $x_2$, \dots, and hypotheses $h_1$, $h_2$, \dots, one
wishes to determine which hypothesis was was most likely. Bayes' Theorem provides a
means of doing this if the hypotheses are assigned probabilities, but provides no
guidance on assigning these a priori probabilities. In 1964 Solomonoff \cite{Solomonoff:1964} 
obtained the remarkable result that there is a canonical prior distribution. 
This distribution is related to Kolomogorov complexity, computability, and artificial
intelligence \cite{AIT:2007}, but that is a fascinating story far beyond the scope of this simple note.


\end{section}

\begin{section}{Sample Spaces}
In probability theory,
all possible outcomes of trial are modeled by a set called the {\em sample space}.
The possible results of flipping a coin can be described by the set $\{H, T\}$, where $H$ represents
the outcome being heads and $T$ represents tails. Likewise, the outcome of the roll of
a die is an element of the set $\{1,2,3,4,5,6\}$. Of course it is possible that the
coin miraculously ends up balancing on its edge or the police raid your craps table 
while the die is in mid roll. Mathematics is just an idealization of the real world.

An element of the sample space is an {\em outcome} that represents complete knowledge of what occurred.
A subset of the sample space is called an {\em event}. E.g., the `an even number was
rolled' in the die example is the subset $\{2,4,6\}$.

Recall the {\em cartesian product} of two sets is $A\times B = \{(a,b)\colon a\in A, b\in B\}$.
Its elements correspond to all ways of choosing one element from the first set and one
element from the second set.  We write $A^n$ for the cartesian product of the set $A$ with itself $n$ times.
It can be thought of as repeated trials of sampling from $A$. If we let $n = \{0, 1, \dots, n-1\}$,
then every function $a\colon n\to A$ corresponds to an element of $A^n$. In general
$A^B = \{f\colon A\to B\}$ is the set of all functions from the set $B$ to the set $A$.


\end{section}

\begin{section}{Boolean Algebras}
In what follows, we will limit ourselves to Boolean algebras with a finite number of
elements that can be represented as sets, where `and' corresponds to set intersection,
`or' to set union, and `not' to set complement. A collection of subsets of a
sample space is a
{\em Boolean algebra} if it is closed under intersection and complement. By De Morgan's
Laws, it is also closed under intersection. 
An event is an {\em atom} if it is not empty and no proper subset belongs to the algebra.
Recall a {\em partition} of a set is a collection of pairwise disjoint subsets sets with union
equal to the set.

\begin{lemma}
The atoms of a Boolean algebra form a partition.
\end{lemma}
\begin{proof}
For each point in the set, the intersection of all events containing that point is an atom,
so the union of all atoms is the entire set. If two atoms have a nonempty intersection,
the atoms must be equal, so the atoms are pairwise disjoint.
\end{proof}

The importance of Boolean algebras is that they represent partial information. If the
set represents all possible outcomes, knowing to which atom an outcome belongs provides
some information about the outcome. For example, the partition $\{\{1,3,5\},\{2,4,6\}\}$
represents the information 'we know whether the outcome was even or odd' for the
roll of a die example.

\end{section}

\begin{section}{Probability Spaces}

The modern axioms of Probability Theory were proposed by Kolomogorov \cite{Kolmogorov:1933,Kolmogorov:1950}.
A {\em probability space} is a {\em sample space}, $\Omega$, representing possible
{\em outcomes}, an algebra of {\em events}, $\F$, of subsets of the samples space, and a {\em probability
measure}, $P$, from the algebra to the real numbers with the properties (i) $P(A)\ge0$ for
all $A\in\F$, (ii) $P(\Omega) = 1$, and (iii) $P(A_1\cup A_2) = P(A_1) + P(A_2)$ for $A_1$, $A_2\in\F$ disjoint.
For an event $A\in\F$, $P(A)$ represents the probability the
outcome belongs to $A$.

\begin{exercise}
Show $P(\bar{A}) = 1 - P(A)$, for $A\in\F$, 
where $\bar{A} = \{\omega\in\Omega\colon \omega\not\in A\}$.
\end{exercise}

\begin{exercise}
Show $P(A\cup B) = P(A) + P(B) - P(A\cap B)$ for $A$, $B\in\F$.
\end{exercise}

Given probability spaces $(\Omega_i, \F_i, P_i)$ one can form the {\em product space}
$\Omega = \Pi_i \Omega_i$, 

\begin{subsection}{Biased Coin Toss}
Let $\Omega = \{H,T\}$ represent the two possible outcomes of heads or tails, the
algebra of events is $\F = \{\{\}, \{H\}, \{T\}, \{H, T\}\}$ is the set of all
subsets of $\Omega$, and $P$ is defined by $P(\{H\}) = \theta$. From the properties
of a probability measure $1 = P(\Omega) = P(\{H,T\}) = P(\{H\}) + P(\{T\})$, hence
$P(\{T\}) = 1 - \theta$.
\end{subsection}

\begin{subsection}{Fair Die Roll}
Let $\Omega = \{1,2,3,4,5,6\}$, $\F$ be the algebra of all subsets of $\Omega$
and $P(\{\omega\}) = 1/6$ for all $\omega\in\Omega$. For example, the event
``the outcome of the roll is even'' corresponds to the subset $\{2,4,6\}$.
\end{subsection}

\begin{subsection}{Rolling a Pair of Dice.}
Let $\Omega' = \Omega\times\Omega$, where
$\Omega$ is as in the previous example. The sigma algebra is the algebra
of all subsets, and $P'(A\times B) = P'(A\times\Omega\cap\Omega\times B)
= P(A) P(B)$, for $A$, $B\subset\Omega$. For example, the event ``rolling
doubles'' corresponds to the six element subset $\{(1,1),\dots,(6,6)\}$
and has probability $6/36 = 1/6$.
\end{subsection}

This is a example of forming the {\em product space} that also represents
two independent rolls of one die. 

\begin{subsection}{Uniformly Distributed Random Variable.}
Let $\Omega = [0,1) = \{x:0\le x<1\}$, $\F$ be the algebra of all Borel measurable sets,
and $P(A)$ be the Lebesgue measure of $A$. 
\end{subsection}

We can use this probability space to model our first example by letting
$H$ correspond to the subset $[0,\theta)$ and $T$ correspond to
$[\theta, 1)$.
\end{section}

\begin{section}{Conditional Expectation}
Given two events, $A$ and $B$, define the {\em conditional probability}
of $A$ given $B$ by $P(A|B) = P(A\cap B)/P(B)$. This is the natural
way of localizing the probability measure to the set $B$. The
function $A\mapsto P(A|B)$ for $A\subset B$ is a probability measure.
($B$ is the new $\Omega$.) 
\begin{exercise}
Show $P(B|B) = 1$ and $P(A_1\cup A_2|B) = P(A_1|B) + P(A_2|B)$
if $A_1$ and $A_2$ are disjoint.
\end{exercise}

The general formula for the intersection of two events is $P(AB)
= P(A)P(B|A)$. (From here on we write $AB$ for $A\cap B$.)  Note 
$P(A|\Omega) = P(A)$ for all events $A$ so in some sense all 
probabilities are conditional.
\begin{exercise}
Show $P(AB|C) = P(A|C) P(B|AC)$ for all events $A$, $B$, $C$.
\end{exercise}

\end{section}

\begin{section}{Random Variables}

A {\em random variable}, $X$, is completely determined by its 
{\em cumulative distribution function}, $F_X$, defined by 
$F_X(x) = Prob(X\le x)$, the probability $X$ takes a value less than 
or equal to $x$ when sampled.
For any measurable set $A$, $Prop(X\in A) = \int_{\{x\in A\}} dF_X(x) 
= \int_{-\infty}^\infty 1_A(x)\,dF_X(x) = E[1_A(X)]$, 
where $1_A(x) = 1$ if $x\in A$ and $= 0$ if $x\not\in A$ is the
{\em indicator function} of the set $A$.

The c.d.f. is nondecreasing, right continuous, tends to 0 at
$-\infty$ and tends to 1 at $\infty$. Any such function is the
c.d.f. of a random variable. If $X$ is {\em discrete}, i.e.,
there exist $x_i$ and $p_i > 0$ with $\sum p_i = 1$ such that
$Prob(X = x_i) = p_i$, then its c.d.f. is piecewise constant with 
jumps of size $p_i$ at $x_i$. If there is a function $f$ such that 
$F(x) = \int_{-\infty}^x f(y)\,dy$ we say $X$is 
{\em continuously distributed}.

For any function $g$, $E[g(X)] = \int_{-\infty}^\infty g(x)\,dF(x)$.
Note $E[a + bX] = a + bE[X]$.
Define the {\em variance} of the random variable $X$ by
$\Var(x) = E[(X - E[X])^2]$.

\begin{exercise}
Show $\Var(X) = E[X^2] - E[X]^2$. 
\end{exercise}

\begin{exercise}
Show $\Var(a + bX) = b^2\Var(X)$.
\end{exercise}

\end{section}

\begin{section}{Bayes' Theorem}

From the definition of conditional probability, $P(A\cap B) = P(A)P(B|A)$. Since
$P(A\cap B) = P(B\cap A)$, we have $P(A)P(B|A) = P(B)P(A|B)$, hence
$P(A|B) = P(A)P(B|A)/P(B)$.
Note that if $A$ and $B$ are independent, then $P(A|B) = P(A)$ and
$P(B|A) = P(B)$. I.e., neither event provides additional information
about the other.

The usual statement of Bayes' Theorem \cite{DeGroot:1986}, page 66, is
\begin{theorem}
Let $A_1$, \dots, $A_n$ form a partition of $\Omega$, then for any event, $B$,
\begin{equation*}
	P(A_i|B) = \frac{P(A_j)P(B|A_j)}{\sum_j P(A_j)P(B|A_j))}.
\end{equation*}
\end{theorem}
\begin{exercise}
Show the denominator above is equal to $P(B)$.
\end{exercise}



$P(A|B_1\cdots B_n) = P(A)P(B_1|A)P(B_2|B_1 )\cdots P(B_n|B_1\cdots B_{n-1}A)/P(B_1\cdots B_n)$.
\end{section}

