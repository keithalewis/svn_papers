\chapter{Kelly Criterion}

\section{Introduction}
In \cite{kel56}, J. L. Kelly proposes an investment strategy
of maximizing expected logarithmic return. If $x\in\R^m$
are the prices of $m$ instruments at the beginning of a
period and $X\colon\Omega\to\R^m$ are the prices at the
end of the period as a function of the outcome over the
period, then given a probability distribution, $P$, on
$\Omega$ the Kelly criterion is to maximize $E[\log \xi\cdot X]$
given $\xi\cdot x = 1$. Using Lagrangian multipliers, put
$\Phi(\xi, \lambda) = E\log \xi\cdot X - \lambda(\xi\cdot x - 1)$.
We have $E[X/\xi\cdot X] - \lambda x = 0$. Using $\xi\cdot x = 1$
implies $\lambda = 1$, hence $x = E[X/\xi\cdot X]$. Note that
we automatically have $\xi\cdot x = 1$. We can also write the
solution as $x = E[X/R_\xi]$, where $R_\xi = \xi\cdot X/\xi\cdot x$
is the realized return.

If the one period model is arbitrage free and a zero coupon bond
exists, i.e., a $\zeta\in\R^m$ with $\zeta\cdot X(\omega) = 1$ for
all $\omega\in\Omega$, then there exists a probability measure, $P$,
such that $x = E[X/R_\zeta]$. In this case, the Kelly criteria
is to invest in the zero coupon bond.

For the binomial model $x = (1, s)$, $X(\pm) = (R, S_\pm)$
with $P(\{\pm\}) = p_\pm$, solving $\max_{m + ns = 1} E[\log(mR + nS)]$
for $m$, $n$ we get $n = p_+/(Rs - S_-) - p_-/(S_+ - Rs)$, m = 1 - ns$.