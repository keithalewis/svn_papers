\chapter{Linear Algebra}

\section{Vector Spaces}
A {\em vector space} is an abelian group with a scalar product that
distributes over addition, $t(v + w) = tv + tw$, where $t\in\R$ is a scalar
and $v$, $w$ are vectors. A {\em linear transformation} is a function
between vector spaces that preserves the vector space operations.
$T\colon V\to W$, $T(tv + w) = tTv + Tw$.
These are also called linear operators. If the dimension of the target space
is 1, then it is called a linear functional.
The space of all linear operators from $V$ to $W$ is denoted $\L(V,W)$.
Note this itself is a vector space if we define the scalar product
$(tT)v = t(Tv)$ and addition $(T + U)v = Tv + Uv$.
The dimension of a vector space is the
maximum cardinality of any set of independent vectors. Two vector
spaces are isomorphic if they have the same dimension.

A {\em subspace} is a subset of a vector space that is itself a vector space.

A {\em cone} is a subset of a vector space that is closed under
addition and multiplication by positive scalars. 

For $T\colon V\to W$ define {\em kernel} of the operator to be the subspace
of $V$, 
$\ker T = \{v\in V:Tv = 0\}$ and
the {\em range} to be the subspace of $W$, $\ran T = \{Tv:v\in V\}$.
The map $\bar T\colon V/\ker T\to\ran T$ given by 
$v + \ker T\mapsto Tv$ is one-to-one and onto.

\subsection{Duality}
The {\em dual} of a vector space is $V^* = \L(V,\R)$, the set of
all linear functionals from the vector space to the scalar field.
We will use the angle braket notation $\langle v, v^*\rangle =v^*(v)$. Give
a linear operator $T\colon V\to W$, define its {\em adjoint}
$T^*\colon W^*\to V^*$ by $\langle v, T^* w^*\rangle = \langle Tv, w^*\rangle$.
There is a canonical linear map $\iota\colon V\to V^{**}$ defined by
$\langle v^*, \iota(v)\rangle = \langle v, v^*\rangle$ for $v\in V$
and $v^*\in V^*$.
The dual of $\R^m$ is $\R^m$ with the pairing $\langle v,w\rangle
= \sum_i v_i w_i$.

For any subset $E\subset V$ define $E^\perp = \{v^*\in V^*: \langle e, v^*\rangle = 0 {\rm\ for\ all\ }e\in E\}$.
For any set, $E$, $E^\perp$ is a subspace of $V^*$.

For any subset $E\subset V$ define $E^+ = \{v^*\in V^*: \langle e, v^*\rangle > 0 {\rm\ for\ all\ }e\in E\}$.
For any set, $E$, $E^+$ is a cone.

If $T\colon V\to W$, $(\ran T)^\perp = \ker T^*$.

\subsection{Direct Sum}
Given vector spaces $V$, $W$, the {\em direct sum}, $V\oplus W$ is
a vector space with
elements $v\oplus w$ for $v\in V$ and $w\in W$ where the scalar
multiplication is $t(v\oplus w) = tv\oplus tw$ and the addition
is $(v\oplus w) + (v'\oplus w') = (v + v')\oplus(w + w')$

\subsection{Tensor Product}
For $v\in V$, $w\in W$ define the {\em tensor product} of vectors
$v\otimes w\in\L(W^*,V)$ by
$(v\otimes w)w^* = \langle w, w^*\rangle v$. The smallest
subspace of $\L(W^*,V)$ containing all such tensor products is denoted
$V\otimes W$. If $T\colon V\to W$ and $U\colon X\to Y$ define
$T\otimes U\colon V\otimes X\to W\otimes Y$ by $T\otimes U(v\otimes x)
= Tv\otimes Ux$.

\subsection{Topological Vector Spaces}
A {\em topological vector space} is a vector space with a topology
such that scalar multiplication and addition are continuous. If it has
a base for the topology consisting entirely of convex sets, then it
is a {\em locally convex} topological vector space.

\subsection{Banach Spaces}
A {\em Banach space} is a topological vector space where the topology
comes from a norm and the topology is complete.

\section{Hahn-Banach Theorem}
\begin{theorem}
If $M$ is a closed subspace of the locally convex topological vector space $V$ and
$P$ is a cone in $V$ with $M\cap P=\emptyset$, then there exists a hyperspace
$N\supset M$ with $N\cap P=\emptyset$.
\end{theorem}

\begin{corollary}
If $M$ is a closed subspace of the locally convex topological vector space $V$ and
$P$ is a cone in $V$,  then  $M\cap P=\emptyset$ if and only if $M^\perp\cap P^+\not=\emptyset$.
\end{corollary}

\section{Fr\'echet Derivative}
If $F\colon X\to Y$ is a function between normed linear spaces, 
the {\em Fr\'echet derivative}, $DF\colon X\to\L(X,Y)$, when it exists,
is defined by $F(x + h) - F(x) - DF(x)h = o(h)$.