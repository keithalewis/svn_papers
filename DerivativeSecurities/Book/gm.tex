\chapter{General Model} 
\label{gm}

A trade entails buying or selling an instrument. The price
of a trade depends on its size
and who is doing the transaction. Brokers
connect buyers and sellers and make the {\em spread},
the difference between the prices they charge
the two counterparties. 
The General Model does not have anything
interesting to say about brokers. The money they make is 
proportional to volume.

There are cash flows involved with holding a position.
Stocks pay dividends or require borrow costs, bonds pay coupons, 
foreign exchange positions held for more than 24 hours
involve carry payments proportional to the difference of the
short term interest rates of the currency pair.

\section{Assumptions}
In what follows we assume that instrument prices
are perfectly liquid, i.e., trades occur exactly at the
given time and specified price, no matter the
quantity. There is no bid/ask spread. The Notes section
indicates how to incorporate more realistic assumptions.

\section{Model}
The price of a instrument, $(X_t)$, is modeled by a semimartingale.
Every instrument has cash flows $(C_j)$ at $(U_j)$ associated with it,
where the $(U_j)$ are stopping times and each $C_j$ is measurable with
respect to $\F_{U_j}$. Trading in the market is modeled by positions,
$(\Xi_j)$ held over the interval $(T_j, T_{j+1})$, where the $T_j$
are increasing stopping times and each $\Xi_j$ is $\F_{T_j}$ measurable.

The {\em profit and loss} at time $t$ due to trading is \begin{equation*}
\sum_j \Xi_j\cdot(X_{T_{j + 1}\wedge t} - X_{T_{j}\wedge t} + \sum \{
C_k : T_j\wedge t < U_k < T_{j+1}\wedge t\}) \end{equation*} where
$t\wedge u = \min\{t, u\}$.

If we define the left-continuous adapted process $\Xi_t = \Xi_j$ 
for $t_j < t \le t_{j+1}$, then $A_t = A_0 + \int_0^t \Xi_s\,dX_s$
using the standard definition of integration with respect to
a ... [Protter]
