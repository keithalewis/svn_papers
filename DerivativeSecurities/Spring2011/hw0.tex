\documentclass[11pt,fleqn]{amsproc}

\newcommand{\R}{{\bf R}}
\newcommand{\Var}{\mathop{\rm{Var}}}
\newcommand{\Cov}{\mathop{\rm{Cov}}}

%\newtheorem{prob}{Problem}[section]
%\newtheorem{cor}[thm]{Corollary}
\newtheorem{xca}{Exercise}

\renewcommand{\labelenumi}{(\alph{enumi})}

\begin{document}

\title{Derivative Securities Bonus Homework}
\author{Due 7:10PM, 2 May 2011}

\maketitle

A Poisson process, $(N_t^\lambda)_{t\ge0}$, with hazard rate $\lambda$
satisfies $P(N_t^\lambda = k) = e^{-\lambda t}(\lambda t)^k/k!$, $k\ge0$
an integer.

\begin{xca}{(20 points)}
Let $X_t = \alpha t + aN_t^\lambda$. Find the {\em characteristic
function} $\phi(u) = E[e^{iuX_t}]$.

\end{xca}

\begin{xca}{(20 points)}
Use the L\'evy-Khinchine theorem to find the characteristic function
of the infinitely divisible distribution parameterized by $\gamma$
and $G = b1_{[c,\infty)}$ where $c\not=0$. (Recall $c = 0$ gives
Brownian motion.)

\end{xca}

\begin{xca}{(20 points)}
Find $\alpha$, $a$, and $\lambda$ in Excercise 1 in terms of
$\gamma$, $b$, and $c$ in Excercise 2.

\end{xca}

Every increasing function can be uniformly approximated by a sum of
step functions, so this shows every L\'evy process is approximated by
a Brownian motion plus a linear combination of Poisson processes.
\end{document}
