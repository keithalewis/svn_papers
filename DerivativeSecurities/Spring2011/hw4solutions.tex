\documentclass[11pt,fleqn]{amsproc}
\usepackage{textcomp}

\newcommand{\R}{{\bf R}}
\newcommand{\F}{\mathcal{F}}
\renewcommand{\H}{\mathcal{H}}
\newcommand{\K}{\mathcal{K}}
\newcommand{\M}{\mathcal{M}}
\newcommand{\N}{\mathcal{N}}
\renewcommand{\P}{\mathcal{P}}
\newcommand{\X}{\mathcal{X}}

\newcommand{\Var}{\mathop{\rm{Var}}}
\newcommand{\Cov}{\mathop{\rm{Cov}}}
\renewcommand{\ker}{\mathop{\rm{ker}}}
\newcommand{\ran}{\mathop{\rm{ran}}}

%\newtheorem{prob}{Problem}[section]
%\newtheorem{cor}[thm]{Corollary}
\newtheorem{xca}{Exercise}

\renewcommand{\labelenumi}{(\alph{enumi})}

\begin{document}

\title{Derivative Securities Homework 4}
\author{Due 7:10PM, 7 Mar 2011}

\maketitle

\begin{xca}{(25 points)}
Let $A(\Xi_0\oplus\cdots\oplus\Xi_{n-1}) = -\Xi_0\cdot X_0
\oplus (\Xi_0 - \Xi_1)\cdot X_1\oplus\cdots\oplus \Xi_{n-1}\cdot X_n
= A_0\oplus A_1\oplus\cdots\oplus A_n$.
\begin{enumerate}
\item Show $A_0 + \cdots + A_j = -\Xi_j\cdot X_j +
\sum_{0\le i < j}\Xi_i\cdot\Delta X_i$, where
$\Delta X_i = X_{i+1} - X_i$, $j < n$. 

\item Show $(\Xi_j)_0^{n-1}$ is self-financing if and only if
$\Delta (\Xi_j\cdot X_j) = \Xi_j\cdot \Delta X_j$.

\end{enumerate}

\end{xca}

Item (a) is clearly true for $j = 0$. Assume true for $j$, then
$A_0 + \cdots A_{j+1}
= \sum_{0\le i < j}\Xi_i\cdot\Delta X_i - \Xi_j\cdot X_j
+ (\Xi_j - \Xi_{j+1})\cdot X_{j+1}
= \sum_{0\le i < j}\Xi_i\cdot\Delta X_i
+ (X_{j+1} - X_j)\cdot X_j - \Xi_{j+1}\cdot X_{j+1}
= \sum_{0\le i < j + 1}\Xi_i\cdot\Delta X_i - \Xi_{j+1}\cdot X_{j+1}$.

From item (a) we have $\Delta \Xi_j\cdot X_j = \Xi_{j+1}\cdot X_{j+1}
- \Xi_j\cdot X_j = -A_{j+1} + \Xi_j\cdot X_j$. The strategy $(\Xi)_j$
is self-financing if and only if $A_j = 0$ for $0 < j < n$.

\begin{xca}{(25 points)}
Let $N$ be normally distributed. Show $E[\max\{e^N - k, 0\}]
= E[e^N]P(e^{N + \Var(N)} > k) - k P(e^N > k)$.

Hint: You may use $E[e^N f(N)] = E[e^N] E[f(N + Var(N))]$.

\end{xca}

Using the formula in the hint, $E[\max\{e^N - k, 0\}]
= E[(e^N - k)1(e^N - k > 0)] = E[e^N1(e^N - k > 0)] - kE[1(e^N - k > 0)]
= E[e^N]E[1(e^{N + \Var(N)} - k > 0)] - kP(e^N > k)]=
E[e^N]P(e^{N + \Var(N)} > k) - k P(e^N > k)$

Consider the two period model with $t_0 = 0$, $t_1 = 1$, and $t_2 = 2$,
where $X_j = (1, B_{t_j})$, $j = 0, 1, 2$, $C_j = 0$ with $(B_t)_{t\ge0}$
being standard Brownian motion starting from zero. Consider an European
call with strike 1 expiring at $t_2$ and the stop-loss/start-gain strategy
$\Xi_0 = (0, 0)$, $\Xi_1 = (-B_1 1(B_1 > 1), 1(B_1 > 1))$.

\begin{xca}{(25 points)}
Show that $E[\Xi_1\cdot X_2] = 0$ and find
$E[\max\{B_2 - 1, 0\}]$.

\end{xca}

One proof is $0 = \Xi_0\cdot X_0 =  E[\Xi_1\cdot X_2]$. We can also
use the independence of increments of Brownian motion to see
$E[(B_2 - B_1)1(B_1 > 1)] = E[B_2 - B_1]P(B_1 > 1) = 0$.

If $X$ is standard normal, then $E[(\sigma X - 1)^+] = \int_{\sigma
x - 1 > 0} (\sigma x - 1) \exp(-x^2/2)\,dx/\sqrt{2\pi} =
\int_{1/\sigma}^\infty \sigma x\exp(-x^2/2)\,dx/\sqrt{2\pi} -
\int_{1/\sigma}^\infty \exp(-x^2/2)\,dx/\sqrt{2\pi} = -\sigma
\exp(-x^2/2)/\sqrt{2\pi}|_{1/\sigma}^\infty - (1 - N(1/\sigma)) =
\sigma \exp(-1/2\sigma^2)/\sqrt{2\pi} - 1 + N(1/\sigma)$, where $N$
is the cumulative distribution function of a standard normal. Since
$\Var(B_2) = 2$ we have $E[(B_2 - 1)^+] = \sqrt{2}\exp(-1/4)/\sqrt{2\pi}
- 1 + N(1/sqrt{2})$.

\begin{xca}{(25 points)}
Create (on your own, please) a spreadsheet using Tukhi
that supports the results of the previous exercise.
Compute the variance of $\max\{B_2 - 1, 0\}$ and
$\max\{B_2 - 1, 0\} - (B_2 - B_1)1(B_1 > 1)$

Hint: See BeatBlackScholes.xls in Tukhi/Samples...

\end{xca}


\end{document}
