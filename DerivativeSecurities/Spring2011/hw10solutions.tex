\documentclass[11pt,fleqn]{amsproc}
\usepackage{textcomp}

\newcommand{\R}{{\bf R}}
\newcommand{\F}{\mathcal{F}}
\newcommand{\G}{\mathcal{G}}
\renewcommand{\H}{\mathcal{H}}
\newcommand{\K}{\mathcal{K}}
\newcommand{\M}{\mathcal{M}}
\newcommand{\N}{\mathcal{N}}
\renewcommand{\P}{\mathcal{P}}
\newcommand{\X}{\mathcal{X}}

\newcommand{\Var}{\mathop{\rm{Var}}}
\newcommand{\Cov}{\mathop{\rm{Cov}}}
\renewcommand{\ker}{\mathop{\rm{ker}}}
\newcommand{\ran}{\mathop{\rm{ran}}}

%\newtheorem{prob}{Problem}[section]
%\newtheorem{cor}[thm]{Corollary}
\newtheorem{xca}{Exercise}

\renewcommand{\labelenumi}{(\alph{enumi})}

\begin{document}

\title{Derivative Securities Homework 10}
\author{Due 7:10PM, 2 May 2011}

\maketitle

\begin{xca}{(50 points)}
Consider the one period model $x = (\sigma_1, \sigma_2, \sigma_3)$,
$\Omega = [1/2,2] \times [1/2, 2]$, $X(\omega_1, \omega_2)
= (|1 - \omega_1|, |1 - \omega_2|, |\omega_1 - \omega_2|)$.
\begin{enumerate}
\item Describe the range of $X$, $X(\Omega)$.
\item Find an arbitrage if $x = (0.2, 0.2, 0.5)$.
\end{enumerate}
\end{xca}

Considering $X$ at the nine points $\{1/2,1,2\}\times\{1/2,1,2\}$ we
see the range of $X$ is the union of the quadrilateral with vertices at
$(0,0,0)$, $(0,0.5,0.5)$, $(1, 0, 1)$ and $(1, 1, 1.5)$, the triangle
with vertices $(0,0,0)$, $(0,0.5,0.5)$ and $(0.5,0,0.5)$ and the triangle
with vertices $(0,0.5,0.5)$, $(0.5,0,0.5)$ and  $(0.5,0.5,0)$.  The point
$(0.2,0.2,0.5)$ is above the plane determined by  $(0,0,0)$, $(0,0.5,0.5)$
and $(0.5,0,0.5)$ So $\xi = (1, 1, -1)$ implements an arbitrage.

\begin{xca}{(50 points)}
Suppose the current exchange rates are {\small\tt EUR|GBP} $0.91$,
{\small\tt EUR|USD} $1.28$, and {\small\tt GBP|USD} $1.28/0.91 \approx
1.41$. An at-the-money straddle costs $\sigma_{\alpha|\beta}$ in currency
$\alpha$ at the beginning of the period and and pays $|X_{\alpha|\beta}
- x_{\alpha|\beta}|$ in currency $\beta$ at the end of the period,
where $x_{\alpha|\beta}$ is the exchange rate at the beginning of the
period and $X_{\alpha|\beta}$ is the exchange rate at the end of the
period. Assume no arbitrage and the {\small\tt EUR|GBP} rate will be in
the interval $[0.55, 2.2]$ and the {\small\tt EUR|USD} rate will be in
the interval $[0.36, 1.56]$ at the end of the period.
\begin{enumerate}
\item Express the cash flows on the straddles in the three currency
pairs above in terms of {\small\tt EUR} by converting at the appropriate
exchange rates.
\item Show $\sigma_{\tt GBP|USD}/x_{\tt EUR|GBP} > \sigma_{\tt EUR|GBP}
+ \sigma_{\tt EUR|USD}$ implies an arbitrage exists and find one.
\end{enumerate}
\end{xca}

We reduce to the previous problem by expressing everything in Euros.
Using the shorthand $x_{eg} = x_{\tt EUR|GBP}$, etc.,
we have
$\sigma_{eg}\to |X_{eg} - x_{eg}|/X_{eg}
= |1 - x_{eg}/X_{eg}|$,
$\sigma_{eu}\to |X_{eu} - x_{eu}|/X_{eu}
= |1 - x_{eu}/X_{eu}|$, and
$\sigma_{gu}/x_{eg} \to |X_{gu} - x_{gu}|/X_{eu}$.
Using $x_{gu} = x_{eu}/x_{eg}$
and $X_{gu} = X_{eu}/X_{eg}$
the last can be written 
$\sigma_{gu}\to |x_{eg}/X_{eg} - x_{eu}/X_{eu}|$.
Letting $\omega_1 = x_{eg}/X_{eg}$ and $\omega_2 = x_{eu}/X_{eu}$
reduces to the previous problem. Buy one {\tt EUR|GBP} straddle,
buy one {\tt EUR|USD} stradle and short one {\tt GBP|USD} straddle.


\end{document}

