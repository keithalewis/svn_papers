\documentclass[11pt,fleqn]{amsproc}
\usepackage{textcomp}

\newcommand{\R}{{\bf R}}
\newcommand{\F}{\mathcal{F}}
\newcommand{\Var}{\mathop{\rm{Var}}}
\newcommand{\Cov}{\mathop{\rm{Cov}}}

%\newtheorem{prob}{Problem}[section]
%\newtheorem{cor}[thm]{Corollary}
\newtheorem{xca}{Exercise}

\renewcommand{\labelenumi}{(\alph{enumi})}

\begin{document}

\title{Derivative Securities Homework 3}
\author{Due 7:10PM, 28 Feb 2011}

\maketitle

Your solution to each exercise should be clear, concise, and compelling.
Do not use graph paper, proffer more than one solution or a nonsensical one.
Do the problems in order, use only one side of the page and stapler.

\begin{xca}{(25 points)}
Consider a deterministic multi-period model with $X_0 = (D_{01}, D_{02})$,
$X_1 = (0, D_{12})$, $C_1 = (1, 0)$, $X_2 = (0, 0)$, and $C_2 = (0, 1)$.
The FTAP implies there exist nonegative deflators $\Pi_1$ and $\Pi_2$. (We
assume $\Pi_0 = 1$.) Express the zero coupon bond prices $D_{01}$,
$D_{02}$, and $D_{12}$ in terms of the deflators.

\end{xca}

\begin{xca}{(25 points)}
Assume a general multi-period model has zero coupon bonds starting
at any time $t_i$ and maturing at any time $t_j > t_i$. Assuming
deflators $\Pi_i$, show $D_{ij} = E_i \Pi_j/\Pi_i$, where $D_{ij}$ is the
price of the zero coupon bond at $t_i$ which matures at $t_j$.
\end{xca}

\begin{xca}{(25 points)}
Consider a two period model with a stock that pays dividend, $d$, at $t_1$.
Show $S_0 = dE_0[\Pi_1] + E_0[S_2\Pi_2]$, where $\Pi_1$ and
$\Pi_2$ are the deflators.

\end{xca}

\begin{xca}{(25 points)}
Let $m$ be Lebesgue measure on $\Omega = [0, 1)$ and $\Omega_{ij}
= [i/2^j, (i + 1)/2^j)$, $j > 0$, $0 \le i < 2^j$.  Let $\F_j$
be the algebra generated by the atoms $\Omega_{ij}$, $0\le i <
2^j$. Clearly $m(\Omega_{ij}) = 2^{-j}$. Find $E_1[X_2m]$, where
$E_1$ is restriction to $\F_1$ of the measure $X_2m$ defined by
$X_2m(\Omega_{i2}) = X_2(\Omega_{i2})m(\Omega_{i2})$ where $X_2(\omega)
= -1$ on $\Omega_{0,2}\cup\Omega_{2,2}$ and $X_2(\omega) = 1$ on
$\Omega_{1,2}\cup\Omega_{3,2}$.

\end{xca}

\end{document}
