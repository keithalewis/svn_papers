\documentclass[11pt,fleqn]{amsproc}
\usepackage{textcomp}

\newcommand{\R}{{\bf R}}
\newcommand{\F}{\mathcal{F}}
\newcommand{\G}{\mathcal{G}}
\renewcommand{\H}{\mathcal{H}}
\newcommand{\K}{\mathcal{K}}
\newcommand{\M}{\mathcal{M}}
\newcommand{\N}{\mathcal{N}}
\renewcommand{\P}{\mathcal{P}}
\newcommand{\X}{\mathcal{X}}

\newcommand{\Var}{\mathop{\rm{Var}}}
\newcommand{\Cov}{\mathop{\rm{Cov}}}
\renewcommand{\ker}{\mathop{\rm{ker}}}
\newcommand{\ran}{\mathop{\rm{ran}}}

%\newtheorem{prob}{Problem}[section]
%\newtheorem{cor}[thm]{Corollary}
\newtheorem{xca}{Exercise}

\renewcommand{\labelenumi}{(\alph{enumi})}

\begin{document}

\title{Derivative Securities Homework 6}
\author{Due 7:10PM, 28 March 2011}

\maketitle


\begin{xca}{(20 points)}
Show $f(t) = r(t) + t r'(t)$, where $f$ is the instantaneous forward rate and
$r$ is the spot rate. Draw a graph of $r(t)$ that implies a
negative forward rate.
\end{xca}

Since $D(t) = \exp(-tr(t)) = \exp(-\int_0^t f(s)\,ds)$ we have
$tr(t) = \int_0^t f(s)\,ds$ and so $tr'(t) + r(t) = f(t)$.
If $tr'(t) < -r(t)$, then $f(t)$ will be negative.

\begin{xca}{(20 points)}
A cash deposit has price $1$ at time $0$ and cash flow $c_j$
at time $t_j$.
Given the discount to time $t_{j-1} < t_j$ is $D(t_{j-1})$, find the
constant forward rate, $f$,
over the interval $[t_{j-1}, t_j]$ that correctly prices the cash deposit.
\end{xca}

Since $0 = -1 + c_jD(t_j)$ and $D(t_j) = D(t_{j-1})\exp(-f(t_j - t_{j-1}))$
we have $f = (\log c_jD(t_{j-1}))/(t_j - t_{j - 1})$.

\begin{xca}{(20 points)}
A forward has price 0 at time 0 and cash flows $-1$ at time $t$ and
$c_j$ at time $t_j > t$. Given the discount to time $t_{j-1} < t_j$ is
$D(t_{j-1})$, find the constant forward rate, $f$, over the interval
$[\max\{t, t_{j-1}\}, t_j]$ when

\begin{enumerate}
\item $t > t_{j-1}$
\item $t < t_{j-1}$, and given $D(t)$,
\end{enumerate}
that correctly prices the forward.

\end{xca}

In both cases $0 = -1D(t) + c_jD(t_j)$.
If $t > t_{j-1}$ then since
$D(t) = D(t_{j-1})\exp(-f(t - t_{j-1}))$, and
$D(t_j) = D(t_{j-1}\exp(-f(t_j - t_{j-1}))$ we have
$c = \exp(f(t_j - t))$
so $f = (\log c)/(t_j - t)$.

If $t < t_{j - 1}$ then $D(t)\exp(-f(t_{j-1} - t)) = D(t_{j-1})$
and $D(t_j) = D(t_{j-1})\exp(f(t_j - t_{j-1}))$ so
$cD(t_{j-1})/D(t) = \exp(f(t_j - t_{j-1}))$
and $f = (\log cD(t_{j-1})/D(t)/(t_j - t_{j-1})$.

\begin{xca}{(20 points)}
A swap pays $(c - F_\delta(t_{j-1}; t_{j-1}, t_j))\delta(t_{j-1}, t_j)$
at times $t_j$, $j = 1$, \dots, $n$.  Find a formula for the fixed coupon,
$c$, in terms of the discount function, $D(t) = D(0,t) = D(t_0, t)$,
that makes the swap value equal to zero using the fact that
$F_\delta(t; u, v) = (D(t, u)/D(t, v) - 1)/\delta(u, v)$.

\end{xca}

The par coupon, $c$, is determined by 
\begin{eqnarray*}
0 &=& \sum_{0 < j <= n}
(c - F_\delta(t_{j-1}; t_{j-1}, t_j))\delta(t_{j-1}, t_j)D(0,t_j)\\
&=& c\sum_{0 < j <= n} \delta(t_{j-1}, t_j)D(0,t_j)
 - \sum_{0 < j <= n} (1/D(t_{j-1},t_j) - 1)D(0,t_j)\\
&=& c\sum_{0 < j <= n} \delta(t_{j-1}, t_j)D(0,t_j)
 - \sum_{0 < j <= n} (D(0,t_{j-1})/D(0,t_j) - 1)D(0,t_j)\\
&=& c\sum_{0 < j <= n} \delta(t_{j-1}, t_j)D(0,t_j)
 - \sum_{0 < j <= n} (D(0,t_{j-1}) - D(0,t_j)\\
&=& c\sum_{0 < j <= n} \delta(t_{j-1}, t_j)D(0,t_j)
 - 1 + D(0,t_n)\\
\end{eqnarray*}
This imples $c = (1 - D(0,t_n))/\sum_{0 < j <= n} \delta(t_{j-1}, t_j)D(0,t_j)$.


\begin{xca}{(20 points)}
Complete the functions {\tt add(const cd\&)} and {\tt add(const fra\&)}
functions in the {\tt yc::pwflat} class and e-mail your Excel add-in to
{\tt kal12@nyu.edu} and {\tt bs1589@nyu.edu}.

\end{xca}

\end{document}
