\documentclass[11pt,fleqn]{amsproc}
\usepackage{textcomp}

\newcommand{\R}{{\bf R}}
\newcommand{\F}{\mathcal{F}}
\newcommand{\Var}{\mathop{\rm{Var}}}
\newcommand{\Cov}{\mathop{\rm{Cov}}}

%\newtheorem{prob}{Problem}[section]
%\newtheorem{cor}[thm]{Corollary}
\newtheorem{xca}{Exercise}

\renewcommand{\labelenumi}{(\alph{enumi})}

\begin{document}

\title{Derivative Securities Homework 3}
\author{Due 7:10PM, 28 Feb 2009}

\maketitle

Your solution to each exercise should be clear, concise, and compelling.
Do not use graph paper, proffer more than one solution or a nonsensical one.
Do the problems in order, use only one side of the page and stapler.

\begin{xca}{(25 points)}
Consider a deterministic multi-period model with $X_0 = (D_{01}, D_{02})$,
$X_1 = (0, D_{12})$, $C_1 = (1, 0)$, $X_2 = (0, 0)$, and $C_2 = (0, 1)$.
The FTAP implies there exist nonegative deflators $\Pi_1$ and $\Pi_2$. (We
assume $\Pi_0 = 1$.) Express the zero coupon bond prices $D_{01}$,
$D_{02}$, and $D_{12}$ in terms of the deflators.

\end{xca}

For a deterministic model every random variable is a single number.
Using $X_0 = (X_1 + C_1)\Pi_1$ gives $D_{01} = \Pi_1$ and $D_{02} =
D_{12}\Pi_1$. Using $X_1\Pi_1 = (X_2 + C_2)\Pi_2$ gives $D_{12}\Pi_1 =
\Pi_2$. This implies $D_{02} = \Pi_2$ and $D_{12} = \Pi_2/\Pi_1$.

\begin{xca}{(25 points)}
Assume a general multi-period model has zero coupon bonds starting
at any time $t_i$ and maturing at any time $t_j > t_i$. Assuming
deflators $\Pi_i$, show $D_{ij} = E_i \Pi_j/\Pi_i$, where $D_{ij}$ is the
price of the zero coupon bond at $t_i$ which matures at $t_j$.
\end{xca}

Using $X_i\Pi_i = \sum_{i < k < j} E_i[C_k\Pi_k] + E_i[(X_j + C_j)\Pi_j]$
we have $D_{ij}\Pi_i = E_i[\Pi_j]$ so $D_{ij} = E_i \Pi_j/\Pi_i$.

\begin{xca}{(25 points)}
Consider a two period model with a stock that pays dividend, $d$, at $t_1$.
Show $S_0 = dE_0[\Pi_1] + E_0[S_2\Pi_2]$, where $\Pi_1$ and
$\Pi_2$ are the deflators.

\end{xca}

We have $S_0 = E_0[(S_1 + d)\Pi_1] = dE_0[\Pi_1] + E_0[E_1[S_2\Pi_2]]
= dE_0[\Pi_1] + E_0[S_2\Pi_2]$.

\begin{xca}{(25 points)}
Let $m$ be Lebesgue measure on $\Omega = [0, 1)$ and $\Omega_{i,j}
= [i/2^j, (i + 1)/2^j)$, $j > 0$, $0 \le i < 2^j$.  Let $\F_j$
be the algebra generated by the atoms $\Omega_{i,j}$, $0\le i <
2^j$. Clearly $m(\Omega_{i,j}) = 2^{-j}$. Find $E_1[X_2m]$, where
$E_1$ is restriction to $\F_1$ of the measure $X_2m$ defined by
$X_2m(\Omega{i,2}) = X_2(\Omega_{i,2})m(\Omega_{i,2})$ and $X_2(\omega)
= -1$ on $\Omega_{0,2}\cup\Omega_{1,2}$ and $X_2(\omega) = 1$ on
$\Omega_{1,2}\cup\Omega_{3,2}$.

\end{xca}

Since $E_1[X_2m]$ is simply the restriction to $\F_1$ of the measure
$X_2m$ that is defined on $\F_2$, we have $E_1[X_2m](\Omega_{0,1})
= X_2m(\Omega_{0,1}) = X_2m(\Omega_{0,2}) + X_2m(\Omega_{1,2}) =
X_2(\Omega_{0,2})m(\Omega_{0,2}) + X_2(\Omega_{1,2})m(\Omega_{1,2}) =
-1(1/4) + 1(1/4) = 0$ and $E_1[X_2m](\Omega_{1,1}) = X_2m(\Omega_{2,2})
+ X_2m(\Omega_{3,2}) = -1/4 + 1/4 = 0$.  This determines the measure on
the atoms of $\F_1$.

\end{document}
