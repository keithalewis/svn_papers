\documentclass[11pt,fleqn]{amsproc}

\newcommand{\R}{{\bf R}}
\newcommand{\Var}{\mathop{\rm{Var}}}
\newcommand{\Cov}{\mathop{\rm{Cov}}}

%\newtheorem{prob}{Problem}[section]
%\newtheorem{cor}[thm]{Corollary}
\newtheorem{xca}{Exercise}

\renewcommand{\labelenumi}{(\alph{enumi})}

\begin{document}

\title{Derivative Securities Homework 1}
\author{Due 7:10PM, 7 Feb 2011}

\maketitle

\begin{xca}{(20 points)}
If today's USD/JPY exchange rate is $82$, the one year USD zero price is 99,
and the one year JPY zero price is 100, what is today's one year forward 
USD/JPY exchange rate assuming no arbitrage and perfect liquidity?
\end{xca}

Consider the zero cost portfolio: $-1$ USD for $82$ JPY now, borrowing
$1$ USD now and paying $1/.99$ USD in one year, investing $82$ JPY now
are receiving $82$ JPY in one year, and a forward exchange contract
for converting $1/.99$ USD into $X(1/.99)$ JPY in one year, where $X$
is the forward exchange rate. No arbitrage implies
$82$ JPY must equal $X/.99$ JPY so the forward exchange rate
is $X = 82(.99) = 81.18$.

\begin{xca}{(30 points)}
Consider the one period model $x = (1, s, v)$,
$\Omega = \{u, d\}$ and $X = (R, s\omega, f(s\omega))$.
\begin{enumerate}
\item Show no arbitrage implies $v = (qF^+ + (1 - q)F^-)/R$, where
$q = (R - d)/(u - d)$, $F^+ = f(su)$, and $F^- = f(sd)$.
\item Find a payoff, $f$, such that $\partial v/\partial s \not= n$, where
$n = (f (su) - f (sd))/(su - sd)$.
\item Show $ns = (\partial/\partial R)(Rv)$ for any payoff $f$.
\end{enumerate}
\end{xca}

The solution can be found in the paper on the Fundamental Theorem
of Asset Pricing: Discrete Time.

\begin{xca}{(20 points)}
Show a regular tetrahedron and a regular octohedron joined at a face
has $8 - 1 + 4 - 1 - 2 = 8$ sides. Hint: consider the points
$A = O + e_1 - e_2$, $B = O + e_1 + e_2$, $C = O + h e_3$ for some $h$, and
$D = O + e_2 + h e_3$.
\end{xca}

If $E = (A + B)/2$, then the length of $CE$ is $\sqrt{3}$ and
that of $OC$ is $h = \sqrt{2}$. We show the product
$ABCD = 0$ so the four points are coplanar.
We have $(O + e_1 - e_2)(O + e_1 + e_2)(O + \sqrt{2}e_3)
(O + 2e_2 + \sqrt{2}e_3) = 
[O\cdot e_1\cdot \sqrt{2} e_3\cdot 2e_2
+e_1\cdot O\cdot \sqrt{2} e_3\cdot 2e_2]
+[e_1\cdot e_2\cdot O\cdot \sqrt{2}e_3
+e_1\cdot O\cdot e_2\cdot \sqrt{2}e_3]
+[-e_2\cdot e_1\cdot O\cdot \sqrt{2}e_3
-e_2\cdot e_1\cdot \sqrt{2}e_3\cdot O]$.
The terms in the brackets cancel so the product is zero.

By symmetry, $A'B'CD = 0$ where $A' = O - e_1 - e_2$
and $B' = O - e_1 + e_2$. 

As pointed out by others, $BB'C'D$ is also zero, where
$C' = -\sqrt{2} e_3$ so the correct answer is
$(8 - 1) + (4 - 1) - 3 = 7$ sides.

\begin{xca}{(30 points)}
Consider the one period model consisting of a bond, call, and
put. Let $x = (1, c, p)$, $\Omega = [L,H]$, and
$X(\omega) = (R, (\omega - K)^+, (K - \omega)^+)$ where
$0\le L < K < H$.

\begin{enumerate}

\item Find the constraints on $c$ and $p$
that preclude arbritrage.

\item Suppose the underlying stock were also available
for trading so $x = (1, c, p, s)$ and
$X(\omega) = (R, (\omega - K)^+, (K - \omega)^+, \omega)$.
Find the constraints on $c$, $p$, and $s$
that preclude arbritrage.

\end{enumerate}
\end{xca}

The smallest cone containing the range of $X$ is
$\{aX(L) + bX(K) + cX(H)\colon a,b,c\ge 0\}$.
Let $O$ be the origin, and use $L$ as shorthand for $X(L)$, etc.
The point $x = e_b + ce_c + pe_p$ is in the cone if and
only if $OxKH/OLKH \ge 0$, $OLxH/OLKH \ge 0$, and $OLKx/OLKH \ge 0$.

It is easy to see that the first condition will reduce to $p \ge 0$ and
the third to $c \ge 0$. We have $OLKH =
O(O + Re_b + (K - L)e_p)(O + Re_b)(O + Re_b + (H - K)e_c)
= O((K - L)e_p)(O + Re_b)((H - K)e_c)
= O(K - L)e_p Re_b (H - K)e_c
= (K-L)R(H-K)Oe_pe_be_c$.
Using the distributive law and the fact products of repeated
points are zero we have $OLxH =
O(O + Re_b + (K - L)e_p)(O + e_b + ce_c + pe_p)(O + Re_b + (H - K)e_c)
= O\cdot Re_b\cdot pe_p\cdot (H - K)e_c
+ O\cdot (K - L)e_p\cdot e_b\cdot (H - K)e_c
+ O\cdot (K - L)e_p\cdot ce_c\cdot Re_b
= R(H - K)pOe_be_pe_c + (K - L)(H - K)Oe_pe_be_c + R(K-L)cOe_pe_ce_b$
so the second condition is $-R(H - K)p + (K - L)(H - K) - R(K - L)c \ge0$.
We can write this as $p/(K - L) + c/(H - K) \le 1/R$.

Since $\omega - K = (\omega - K)^+ - (K - \omega)^+$ we must
have $s - K/R = c - p$ so $s = K/R + c - p$. Adding the stock
does not expand the market in this model.

\end{document}
