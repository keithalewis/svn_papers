\documentclass[fleqn]{amsart}
\usepackage{amssymb,amsmath,amsthm,hyperref}

\def\F{\mathcal{F}}
\def\Var{\mathop{\rm Var}}
\def\Cov{\mathop{\rm Cov}}
\newcommand{\RR}{\mathbb{R}}

\title{A Quant's Job}
\author{Keith A. Lewis}
\email{kal@kalx.net}
\address{KALX, LLC \tt{\url{http://kalx.net}}}

\begin{document}
\maketitle

\begin{section}{What Quants Do}
Specify a set of possible outcomes, $\Omega$,
and algebras of sets on $\Omega$, $(\F_t)_{t\ge0}$, where $\F_t$ represents
information available at time $t$.

Then next step is to specify a short rate process, $(R_t)_{t\ge0}$.
This process has to be better than adapted, it must be predictable, and
represents the notion that 1 unit invested at time $t$ will pay $1 +
R_t\,dt = \exp(R_t\,dt)$ at time $t + dt$.  Define the {\em stocastic
discount} to be $D_t = \exp(-\int_0^t R_s\,ds)$.

The final step is to specify a vector valued stochatic process,
$(X_t)_{t\ge0}$ representing the prices of market instruments
such that $(X_tD_t)_{t\ge0}$ is a martingale.

To do a good job, $X_0$ must be todays prices and $(X_t)_{t\ge0}$ should
permit dynamics that capture possible market moves.

To do a great job, the parmeters of the stochastic process should
correspond as closely as possible to market prices.
\end{section}

\begin{section}{Why Quants Do It}
If $(\Xi_t)_{t\ge0}$ 
Being able to find a strategy, $\Xi$, that permits 
\end{section}

\end{document}
