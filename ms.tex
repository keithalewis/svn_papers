\documentclass[12pt,letterpaper,fleqn]{report}
\usepackage[latin1]{inputenc}
\usepackage{amsmath}
\usepackage{amsfonts}
\usepackage{amssymb}
\usepackage{lmodern}
%\usepackage{kpfonts}
%\usepackage{listings}
%%\lstset{language=C++, basicstyle=\ttfamily}
\author{Keith A. Lewis}
\title{Monotonic Streams}
\begin{document}
\maketitle
Time series are data indexed by a monotonic stream of times.
We define fundamental operations
directly on the stream of times and show how to attach data to
those. The algorithms can be implemented efficiently and are well
suited to columnar data.

An example of a {\em stream} is a sequence \(t_0, t_1, \dots\) where the
\(t_j\) are of some type \(T\).  The primitive operations on a stream
\(s\) are \(*s\) returning the current {\em item} in the stream, \(+s\)
which moves to the {\em next} item in the stream, and \(?s\) which returns
a boolean indicating if the stream is {\em done}. This interface is more
general than an array of items, \(s[i] = t_i\), since it permits lazy
evaluation and infinite streams. A {\em monotonic stream} is a stream of
non-decreasing items under some order, e.g., time. It satisfies \(*s \le
*+s\) where we assume operations are evaluated right to left and \(?s\)
and \(?+s\) are true. The stream is {\em strict} if \(*s < *+s\) holds.

A stream can be tuple valued and we indicate the $i$-th element
by $s<i>$ with the convention $*s = s<0>$. 

More generally, given two streams, $s$ and $t$ we can

Streams are similar to C++ iterators or C\# {\tt IEnumerator}s.
A typical use scenario in a program is {\tt while (?s) \{ }{\em use\
}{\tt *s; +s; \}}.  Note advancing to the next item in the stream and
testing for remaining elements are separate operations.  In C++ the
STL library uses {\tt end()} and C\# requires calling {\tt MoveNext}
and testing the return value before calling {\tt Current}.

The basic operations to combine two streams into new streams 
are denoted {\tt [} and {\tt ]}. If $s$ and $t$ are streams
then

The union, intersection, and difference of streams is well-defined
mathematically but they have canonical implementations
that preserve monotonicity.

Let \(s\vee s'\) be the {\em join} of monotonic streams that preserves
their order.  An implementation satisfies \(*(s\vee s') = *s\) if \(*s <
*s'\) and \(*s'\) otherwise.  Moving to the next item increments \(s\)
in the first case and \(s'\) in the second if neither are done. The
join is done when both streams are done. Defining {\em meet}, \(s\wedge
s'\), for the intersection and {\em less}, \(s\backslash s'\), for the
difference is equally straightforward.


Given a {\em predicate}, i.e., a function \(p\colon T\to \mathrm{Bool}\),
and a stream \(s\) define \(s|p\) to be the stream of items that satisfy
the predicate. Incrementing this stream has the effect {\tt while (?s \&\&
!p(*s)) \{ +s; \}}.

Given a monotonic stream \(s\) and any function \(f\colon T\to U\)
define \(f(s)\) to be the stream of type \(U\) with \(*f(s) = f(*s)\),
\(+f(s)\) calls \(+s\), and \(?f(s) = ?s\).

It is also useful to define
two piece-wise constant functions by \(f[s(t) = f(t_j)\) if \(t\in [t_j,
t_{j+1})\) and \(f]s(t) = f(t_j)\) if \(t\in (t_j, t_{j+1}]\). The former
is right continuous and is used for market prices (prices can jump),
the latter is used for positions (execution is not instantaneous).
Note \(f[s(t)\) depends only on \(t_j \le t\) and \(f]s(t)\) only on
\(t_j < t\), i.e., they cannot look at future stream values.

Given a time \(t\) let \(s > t\) denote advancing the stream until
we find an item greater than \(t\). Similarly for \(s \ge t\). Let
\(s < t\) and \(s \le t\) denote the streams that stop at \(t\)
either excluding or including \(t\). These two operations do
not have an immediate effect, unlike the previous two.

A type of concatenation for monotonic streams that is useful for describing
trading strategies is denoted by \(s = s_0,s_1,\dots,s_{n-1}\).
The initial index \(i\) is 0. Define \(*s = *s_i\) and
\(+s\) by \(s_{i+1}>*s_i\), \(i = i + 1\) modulo \(n\).
If after \(s_{i+1}>*s_i\) we have \(?s_{i+1}\) is false, apply the same 
procedure to the the next stream.
Define \(?s\) by all \(?s_j\) being false. 
Let \(@ s\) denote the current index of the concatenated
monotonic streams. It is also useful to use the condition
\(s_{i+1}\ge *s_i\) instead of strict inequality. In this case
we use a period instead of a comma.

It is also useful to attach buffers to monotonic streams,
e.g., a queue of fixed size or a range based on the contents of
the buffer such as the last 5 minutes of ticks. The
buffer can also be an accumulator (and contain only one item) such
as a weighted moving average. Buffers take streams to
streams of streams like {\tt SelectMany} in LINQ.

\section{Trading Strategies}
A simple trading strategy is to fix two values \(k_1 < k_2\) and buy
an instrument when the price is below \(k_1\) then sell when the
price goes above \(k_2\). If \(s\) is the stream of tick times and
\(x\) is the instrument price then \(u = s|(x]s < k_1)\) gives
the stream of enter signals. The predicate is evaluated at
each tick to give \((x]s)(*s) < k_1\). Note we use \(x]s\) and
not \(x[s\) because strategies cannot be executed instantaneously.
Similarly, the exit signal
stream is \(s|x]s > k_2\). 
Monotonic stream concatenation gives the
is the stream of trading times, \(u = (s|s]x < k_1),(s|s]x > k_1)\).
Buy one share when \(@u = 0\)
and sell one share when \(@u = 1\) is the trading strategy.

This strategy stops generating trades when the price moves away
from the \([k_1,k_2]\) range. Suppose we want to reset the
range based on the market close each week. Let \(eow\) be
the stream of end of week closing times. The stream
\(u = (s|s < (x[eow) - h),(s|s > (x[eow) + h)\) has the effect of
setting \(k_1 = x[eow - h\) and \(k_2 = x[eow + h\) at the
end of each week.

Consider the following strategy:
if market open is less than last week's market close, buy one share.
If market close is greater than last week's market close and you are long, close out.
Always close out at the end of week even if it means taking a loss.

Define the stream \(eow\) to be the weekly market closing time,
\(mo\) to be daily market open, and \(mc\) to be daily market close. Let
\(x(t)\) be the price of the underlying stock at time \(t\).
The start gain component can be expressed by \(sg = mo|(x < x[eow)\). 
This produces the
stream of market open times where \(x(*mo) < x[eow(*mo)\), i.e.,
the market open price is less than the previous week's close.
The stop loss component is \(sl = mc|(x > x[eow)\). This produces the
stream of market close times where \(x(*mc) > x[eow(*mc)\), i.e.,
the market close is greater than the previous week's close.

Stream concatenation expresses the timing of when these conditions
are in effect. The expression \(slsg = sg,(sl\vee eow)\) gives the
appropriate monotonic stream for determining the position. First
we look for a market open that satisfies the entry criterion. We
then switch to looking for either the stop loss or end of week
criteria that occur after that.

In the original specification it is not clear if we should do multiple
trades during a week. The above implementation allows that. Suppose
we wish to limit the strategy to doing at most one trade per week.
That is specified by \(s = sg,(sl\vee ewo).eow\).
After the exit signal immediately start looking for the end of week.
If the exit and end of week coincide, the period operator ensures
we move on to looking at the next market open.
The trades are buy 1 share if \(@s = 0\), sell one share if \(@s = 1\), and do
nothing if \(@s = 2\).

\end{document}
