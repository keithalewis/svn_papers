\documentclass[11pt,letterpaper,fleqn]{report}
\usepackage[latin1]{inputenc}
\usepackage{amsmath}
\usepackage{amsfonts}
\usepackage{amssymb}
\usepackage{lmodern}
\usepackage{kpfonts}
\usepackage{bm}
\usepackage{hyperref}

\newtheorem{theorem}{Theorem}[section]
\newtheorem{definition}{Definition}[section]

\newcommand{\reason}[1]{{#1}\equiv\ }

\author{Keith A. Lewis}
\title{Notes on the \(\lambda\)-ccalculus}
\begin{document}
\maketitle
\section{The \texorpdfstring{\(\lambda\)-} calculus}
The \(\lambda\)-calculus involves {\em variables}, {\em expressions}, 
{\em \(\lambda\)-abstraction}, and {\em application}.
Variables are denoted by lowercase roman letters. Expressions
are denoted by uppercase roman letters. The standard notation
for \(\lambda\)-abstraction is \(\lambda v . E\), where \(v\) is
a variable and \(E\) is an expression. We will use \((v\to E)\) instead.
Application is denoted by \((E F)\) where \(E\) and \(F\) are expressions.

Parentheses can be implicit. Removing outer parentheses in an expression
does not cause ambiguity. We assume abstraction associates right to left
so \(v\to w\to E\) is \(v\to (w\to E)\), or \(vw\to E\) for short.
Application associates left to right so \(EFG\) is \((EF)G\) and not \(E(FG)\).

If \(v\) is a variable, then \(v\) is an expression. If \(E\) is an expression
then so is \(v\to E\). If \(F\) is an expression then \(EF\) is too.

A single variable is {\em free}. The free variables in \(v\to E\) are the free
variables in \(E\) except for \(v\). The free variables in \(EF\) are the union
of the free variables in \(E\) or \(F\). Variables that are not free are {\em bound}.
\begin{definition}{(\(\alpha\)-conversion)}
Replacing all occurrences of a bound variable in an expression
by a variable that is not free gives an equivalent expression.
\end{definition}
\begin{definition}{(\(\beta\)-reduction)}
The expression \((v\to E)F\) is equivalent to
\(E\) with all occurences of \(v\) in \(E\) replaced by \(F\).
\end{definition}
\begin{definition}{(\(\eta\)-equivalence)}
The expression \((v\to E)v\) is equivalent to \(E\)
if \(v\) is not free in \(E\).
\end{definition}

This completes the definition of the (untyped) \(\lambda\)-calculus.
Simple as it is, it is powerful enough to express any function that can
be can be computed.
% and gives a negative answer to the 
% \href{http://en.wikipedia.org/wiki/Entscheidungsproblem}{Entscheidungsproblem}.

\section{Examples}
The simplest example of \(\lambda\)-abstraction is \(x\to y\) where \(x\)
and \(y\) are distinct variables.
For any expression \(M\), \((x\to y)M\) can be \(\beta\)-reduced to \(y\).
The expression \(x\to x\) is the identity function: for any
expression \(M\), \((x\to x)M\) reduces to \(M\). We use the notation
\(I = x\to x\) to give this expression the {\em name} \(I\).

The next simplest example is \(x\to y\to z = xy\to z\). For any
expressions \(M\) and \(N\), \(((x y\to z)M)N\) reduces to \(z\).
Since \(x\) and \(y\) are bound, there are only two cases to consider
when not all variables are distinct.
Let \(T = x y\to x\) and \(F = x y\to y\).
Observe \(TM\) reduces to \(y\to M\) so \((TM)N\) is \(M\),
and \(FM\) reduces to \(I\) so \((FM)N\) is \(N\).

Every expression is a {\em fixed point} of the identity but an amazing
fact about the \(\lambda\)-calculus is that every expression has at
least one fixed point.  Let \(Y = f \to(x\to fxx)(x\to fxx)\), so
\[
Y\underline{f}\equiv \bm{(x\to fxx)\underline{(x\to fxx)}}\equiv
\bm{f\underline{(x\to fxx)(x\to fxx)}}\equiv f\bm{Yf},
\]
hence \(Yf\) is a fixed point of \(f\).

\section{Logic}

Writing \(x\to y\to \cdots v\to E\) as \(xy\dots v\to E\),
The expression \(? = xyz\to (x y) z\) is the \(\lambda\)-calculus
equivalent of an if-then-else construct since
\begin{align*}
&\underline{?}T\\
\reason{name}&\bm{(\underline{x}yz\to (\underline{x} y) z)}\underline{T}\\
\reason{reduction}&yz\to (\underline{\bm{T} y) z}\\
\reason{T}&\underline{yz\to \bm{y}}\\
\reason{name}&\bm{T}\\
\end{align*}
and
\begin{align*}
&\underline{?}F\\
\reason{name}&\bm{(\underline{x}yz\to (\underline{x} y) z)}\underline{F}\\
\reason{reduction}&yz\to (\underline{\bm{F} y) z}\\
\reason{F}&\underline{yz\to \bm{z}}\\
\reason{name}&\bm{F}\\
\end{align*}
so \(((?T)M)N\equiv M\) and \(((?F)M)N\equiv N\). 

\(not = x\to (xF)T\) since \(not\ F\equiv (FF)T\equiv T\)
and \(not\ T\equiv (TF)T\equiv F\).

\(and = xy\to (xy)F\) since \(and\ F\equiv y\to (Fy)F\equiv y\to F\) is the
constant function \(F\) and \(and\ T\equiv y\to (Ty)F\equiv y\to y\) is the identity function.

\(or = xy\to (xT)y\) since \(or\ F\equiv y\to (FT)y\equiv y\to y\)
is the identity functions and \(or\ T\equiv y\to (TT)y\equiv y\to T\)
the constant function \(T\).

%\(pair\colon xyp\to (p x)y\), \(head\colon z\to z T\), \(tail\colon z\to z F\)
%so \(head\ pair\ x y \equiv head\ (p x)y\equiv ((T x)y)\).

%Fixed points \(II = I\), \(?T = T\), \(?F = F\).

\section{Church Numerals}
The \(\lambda\)-calculus is powerful enough to describe basic arithmetic.
To simplify exposition we will give names to expressions. Choose any two
variables \(s\) and \(x\) and define \(\mathit{0} = sx \to x\),
\(\mathit{1} = sx\to s x\), \(\mathit{2} = sx\to s s x\), \dots,
where the {\it name}s {\it 0}, {\it 1}, {\it 2}, \dots are not (are to) to be
confused with the numbers 0, 1, 2,\dots.
In general
{\it n} is \(sx\to s^\mathrm{n}x\) where \(s^\mathrm{n}\) means
\(s\) repeated \(n\) times.
Note {\it 0E} is \(x\to x\) for any expression \(E\).
%% n -> (x -> (s ((n s) x)))
%% n ->  x -> (s ((n s) x))  , x -> (E) = x -> E
%% n ->  x ->  s ((n s) x)   , x -> (E) = x -> E
%% n ->  x ->  s  (n s) x    , M(NO) = MNO
%

Let \(inc = nsx\to s (n s) x\).
\begin{align*}
&\underline{inc}\ \mathit{0}&\\
\reason{name}
&\bm{(\underline{n}sx\to s (\underline{n} s) x)}\underline{\mathit{0}}\\
\reason{reduction}
&sx\to s (\underline{\mathbf{0} s}) x\\
\reason{\mathit{0}}
&sx\to s (\underline{\bm{x\to x})x}\\
\reason{reduction}
&\underline{sx\to s \bm{x}}\\
\reason{\mathit{1}}
&\bm{1}\\
\end{align*}

Similarly, {\it 2} is \(inc\ \mathit{1}\) or \(inc\ inc\ \mathit{0}\).
All numbers come from {\it inc}rementing {\it 0}. 
%
%Let \(add := m\to n\to s\to x\to m s (n s) x\). Note {\it 1E} 
%is \(x\to Ex\) for any expression \(E\).
%\begin{align*}
%\underline{add}\ \mathit{1}\\
%\bm{(\underline{m}\to n\to s\to x\to \underline{m} s (n s) x)}\ \underline{\mathit{1}}
%&\reason{name}\\
%n\to s\to x\to \underline{\bm{1} s} (n s) x
%&\reason{reduction}\\
%n\to s\to x\to \underline{\bm{(y\to sy)} (n s) x}
%&\reason{\mathit{1},\ conversion}\\
%\underline{n\to s\to x\to \bm{s(n s) x}}&\reason{reduction}\\
%\bm{inc}&\reason{name}
%\end{align*}
%We have tediously demonstrated \((add\ 1)\ 0\) is \(1\).
%You should try to reduce \(m\to n\to m\ inc\ n\) to \(add\).
%Note {\it 2E} reduces to \(x\to EEx\). In general {\it nE}
%reduces to \(x\to E^nx\) where \(E^n\) is \(E\) repeated (the number) n times.
%
%Let \(mul := m\to n\to s\to m n s\). Show this reduces to \(m\to n\to m (add\ n) 0\).
%Let \(exp := m\to n\to m n\). Note \((exp\ 2)\ 3\) reduces to
%\(2 3\) which reduces to \((x\to 33x)3\), \((33)3\), \((3333)3\)
%\begin{align*}
%\end{align*}
%
%\section{Combinators}
%Note {\it 0}, {\it 1}, \dots, \(inc\), \(add\) and \(mul\) have no free variables.
%Such expressions are called {\em combinators}.
%The identity function \(I := x\to x\) is the simplest example. The next simplest
%example is \(K := x\to y\to x\). 
%Since \((KM)N\) reduces to \(M\) for any expression \(N\) we can
%think of \(KM\) as a function that takes on the constant value \(M\).
%
%The combinator {\it 0} can be written \(KI\).
%\begin{align*}
%KI&\\
%(x\to y\to x)I&\reason{name}\\
%y\to I&\reason{reduction}\\
%y\to (x\to x)&\reason{name}\\
%s\to (x\to x)&\reason{conversion}\\
%\mathit{0}&\reason{name}
%\end{align*}
%The easy way to figure this out is to note \(KM\) is 
%\((x\to y\to x)M\) which
%reduces to \(y\to M\) and work backward. Since
%{\it 0} is \(s\to x\to x\), convert \(s\) to \(y\), and can use this fact to write 
%\(K(x\to x)\), hence \(KI\).
%
%% K(y -> z -> (x z) y z)
%% KK(z -> (x z) y z)
%% KK(S(z -> (x z))(z -> (y z))
%Let \(S := x\to y\to z\to (x z) y z\). Since
%\begin{align*}
%(S(x\to M))(x\to N)&\\
%((x\to y\to z\to (x z) y z)(x\to M))(x\to N)&\reason{name}\\
%((y\to z\to ((x\to M) z) y z))(x\to N)&\reason{reduction}\\
%((z\to ((x\to M) z) (x\to M) z))&\reason{reduction}\\
%\end{align*}
%we can replace expressions of the form \(x\to MN\)
%by \(S(x\to M)(x\to N)\).
%
%For example
%\begin{align*}
%inc&\\
%n\to s\to x\to s (n s) x&\reason{name}\\
%K(s\to x\to s (n s) x)&reason{x\to M}\\
%KK(x\to s (n s) x)&\reason{x\to M}\\
%KKS(x\to s)(x\to (n s) x&\reason{x\to MN}\\
%KKS(Ks)(S(x\to ns)(x\to s)&\reason{s\to MN}\\
%\end{align*}
%
\end{document}
