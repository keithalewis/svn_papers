\documentclass[11pt,fleqn]{article}
\usepackage{amssymb,amsmath,amsthm,hyperref}
\usepackage{enumerate}

\newcommand{\G}{{\cal G}}
\newcommand{\R}{\mathbb{R}}
\newtheorem{lemma}{Lemma}[section]
\newtheorem{ex}{Excercise}[section]

\title{Grassmann Algebra}
\author{Keith A. Lewis}

\begin{document}
\maketitle

\section{Introduction}

Hermann Grassman invented a calculus for geometry in ...
His first book was not well received...
His second book was not either...
Peano picked up on his work...
Cartan picked up on a subset of his work...
Introduced a notation for the exterior product that is arguably wrong...
Rota et al solved the problem of the interior product...

\section{Foundations}

Let \(E\) be Euclidean space. In modern (Bourbaki) terms, the
Grassmann algebra is the algebra generated by \(E\) modulo the relation
\(P^2 = 0\).

In the spirit of Peano\cite{Pea1891}, we can describe the Grassmann algebra axiomatically.

\begin{enumerate}[i]

\item The Grassmann algeba \(\G\) contains the real numbers and \(E\).

\item If \(a\in\R\), \(P\in E\) then \(aP = Pa \in \G\).

\item If \(A,B \in \G\) then \(A + B \in \G\) and \(AB \in \G\).

\item If \(P,Q\in E\) then \(PQ = 0\) if and only if \(P = Q\).

\end{enumerate}

An immediate consequence of this is that \(PQ = -QP\)
since \(0 = (P + Q)(P + Q) = PQ + QP\).

The key element of Grassmann's algebra is that products vanishing
indicate linear dependence.

\begin{lemma}
If \(P_0, \dots P_k\in E\) and \(P_0\cdots P_k = 0\) then there
exist \((t_j)\) not all zero such that \(\sum_0^k t_j P_j = 0\).
\end{lemma}

\section{Points}

If \(P,Q,R\in E\), \(PQR = 0\), and \(PQ\not=0\) then \(R = (1 - t)P +
tQ\) for some \(t\).  Writing \(R = R(t) = (1 - t)P + tQ = P + t(Q - P)\)
we can interpret \(Q - P\) as the vector from \(P\) to \(Q\). Note \(P =
R(0)\) and \(Q = R(1)\).  Also note \(Q - P \not= R(t)\) for any \(t\).

We interpret \(aP + bQ\) as the point \(R(b/(a + b))\) having
weight \(a + b\).

Since \(PR = tPQ\) and \(RQ = (1-t)PQ\) we have \(R = \frac{RQ}{PQ} P +
\frac{PR}{PQ} Q\).  Note \(\frac{RQ}{PQ} + \frac{PR}{PQ} = 1\) so \(PR +
RQ = PQ\) if the points are collinear.

More generally, if \(P_0\cdots P_k\not=0\) and \(P_0\cdots P_k P = 0\)
then the same technique yields \(P = \frac{PP_1\cdots P_k}{P_0\cdots
P_k}P_0 + \cdots + \frac{P_0\cdots P}{P_0\cdots P_k}P_k\).

\begin{ex}
If \(P,Q,R\in E\), show \(P\), \(P + Q\), and \(P + Q + R\) are
collinear.
\end{ex}
This proves the medians of a triange meet the centroid.

\section{Lines}

For \(P,Q,R,S\in E\), when do we have \(PQ = RS\)?
Since \(PQR = 0\) and \(PQS = 0\) there exist \(r,s\in\R\) such
that \(R = (1 - r)P + rQ\) and \(S = (1 - s)P + sQ\).
It follows that \(1 = (1 - r)s - r(1 - s) = s - r\) so
\(R = P + rv\) and \(S = Q + rv\) where \(v = Q - P\).

This shows we can interpret \(PQ\) as the line determined by
\(P\) and \(Q\) having magnitude \(Q - P\).

\section{Vectors}
A {\em vector} is an element having the form \(Q - P\) for \(P,Q\in E\).
Since \(Q - P \not= (1 - t)P + tQ\) for any \(t\) we interpret
vectors as points at infinity. Note \(v \not= -v\) so we get points
at \(+\infty\) and \(-\infty\).

\bibliographystyle{plain}
\bibliography{grassmann}{}

\end{document}
