%\documentclass[fleqn]{article}
\documentclass[fleqn]{amsart}
\usepackage{amssymb,amsmath,amsthm,hyperref,setspace}

\newcommand{\R}{\mathbf{R}}

\newcommand{\BB}{\mathcal{B}}
\newcommand{\CC}{\mathcal{C}}
\newcommand{\FF}{\mathcal{F}}
\newcommand{\GG}{\mathcal{G}}
\newcommand{\HH}{\mathcal{H}}
\newcommand{\PP}{\mathcal{P}}
\newcommand{\QQ}{\mathcal{Q}}
\newcommand{\XX}{\mathcal{X}}

\newcommand{\weakstar}{weak$^*$\ }

\newcommand{\ran}{\mathop{\mathrm{ran}}}
\renewcommand{\ker}{\mathop{\mathrm{ker}}}
\providecommand{\Var}{\mathop{\mathrm{Var}}}
\renewcommand{\Re}{\mathop{\mathrm{Re}}}
\renewcommand{\Im}{\mathop{\mathrm{Im}}}
\newcommand{\Arg}{\mathop{\mathrm{Arg}}}

\providecommand{\abs}[1]{\lvert#1\rvert}
\providecommand{\norm}[1]{\lVert#1\rVert}
\providecommand{\pair}[1]{\langle#1\rangle}

\newcommand{\onehalf}{\frac{1}{2}}

\newtheorem{definition}{Definition}[section]
%\newtheorem{theorem}[definition]{Theorem}
\newtheorem{theorem}{Theorem}[section]
\newtheorem{corollary}[theorem]{Corollary}
\newtheorem{lemma}[theorem]{Lemma}
\newtheorem{example}{Example}

%\doublespace

\title{Option Pricing}
\author{Keith A. Lewis}
\date{\today}
\begin{document}
%\begin{abstract}
%\end{abstract}
\address{KALX, LLC \tt{\url{http://kalx.net}}}
%\thanks{
%For the memories.
%}

\maketitle

\section{Introduction}
As my friend Peter Carr admonishes me at regular intervals, ``Never use
the word `volatility' without an adjective.'' Black-Scholes implied
volatility is so seductively useful we often forget how limited it
is when it comes to pricing options in the real world.
Perhaps we should banish the word altogether and talk only about
`distribution' instead. As pointed out by Ross (other cite?), the so-called
risk-neutral cumulative distribution of the underlying is the 
derivative of put price with respect to strike. After all, this
gives us a direct method of obtaining the distribution from market
data.

Unfortunately, this parameterization is not very tractable. Options
trade at only a limited number of strikes and taking a derivative
introduces more noise. The Holy Grail is to find a model
that fits current prices and does not have parameters that
change over time. 

\section{General Model}
Let \((X_t)_{t\ge0}\) be any stochastic process having independent
increments. Define the
futures process to be
\[
	F_t = f e^{-\kappa_t(s) + s X_t}
\]
where \(\kappa_t(s) = \log E e^{s X_t}\) so \(E F_t = f\) for all \(t\).
Futures are martingales because their price is always zero.
If \(X_t\) is Brownian motion, then \(\kappa_t(s) = s^2t/2.\)

Recall \(\kappa(s) = \log E e^{sX}\) is the {\em cumulant} of the
random variable \(X\). Note \(\kappa'(0) = E[X]\) and \(\kappa''(0) =
\Var(X)\). The higher order derivatives have a less cogent relationship
with the moments.

The price of a put with strike $k$ expiring at time $t$, assuming zero
interest rates, is
\begin{align*}
E\max\{k - F_t, 0\} &= E(k - F_t)1(F_t \le k) \\
&= kE1(F_t \le k) - E F_t1(F_t \le k)\\
&= kP(F_t \le k) - f P^*(F_t \le k),\\
\end{align*}
where \(dP^*/dP = e^{-\kappa_t(s) + s X_t}\).
Note \(P^*\) is a probability measure. This trivial result
goes by the name Esscher transform.

The {\em characteristic} function of the random variable \(X\)
is \(\phi(u) = \log E e^{iuX}\). Note \(\phi(u) = \kappa(iu)\)
where \(\kappa\) is the cumulant so \(E X = -i\phi'(0)\)
and \(\Var X = -\phi''(0)\).

If \((X_t)\) is a L\'evy process then \(E e^{iuX_t} = e^{t\phi(u)}\)
where \(\phi(u)\) is the characteristic function of \(X_1\).
It is also the case that \(X_1\) is infinitely divisible and we
can derive a more interesting and useful result.

Prior to L\'evy and Khintchine, Kolomogorov (cite?) derived a
parameterization for the characteristic function of infinitely divisible
distributions having finite variance. There exists a number \(\gamma\)
and a non-decreasing function \(G(x)\) such that

\[
\phi(u) = i\gamma u + \int_{-\infty}^\infty K_u(x)\,dG(x),
\]
where \(K_u(x) = (e^{iux} - 1 - iux)/x^2\).
Note \(\phi'(u) = i\gamma + i\int_{-\infty}^\infty (e^{iux} - 1)/x\,dG(x)\)
and \(\phi''(u) = -\int_{-\infty}^\infty e^{iux}\,dG(x)\)
so \(E X = -i\phi'(0) = \gamma\) and \(\Var X = -\phi''(0)
= \int_{-\infty}^\infty dG(x) = G(\infty) - G(-\infty)\).

\begin{lemma}
If \(X\) is infinitely divisible with Kolomogorov parameters \(\gamma\)
and \(G\), then \(Ee^{isX}e^{iuX} = Ee^{isX}Ee^{iuX^*}\) where
\(X^*\) has Kolomogorov parameters \(\gamma^*
= \gamma + \int_{-\infty}^\infty (e^{isx} - 1)/x\,dG(x) = -i\phi'(s)\)
and \(dG^*(x) = e^{isx}dG(x)\).
\end{lemma}
\begin{proof}
We have
\begin{align*}
E e^{isX}e^{iuX} &= E e^{i\gamma(s+u) + \int_{-\infty}^\infty K_{s+u}(x)\,dG(x)}\\
	&= Ee^{isX} e^{i\gamma u
		+ \int_{-\infty}^\infty (K_{s+u}(x) - K_s(x))\,dG(x)}
\end{align*}
A simple calculation shows
\(K_{s+u}(x) - K_s(x) = iu(e^{isx} - 1)/x + e^{isx}K_u(x)\)
so \(E e^{isX}e^{iuX} = Ee^{isX}Ee^{iuX^*}\) where \(X^*\) is
infinitely divisible with Kolomogorov parameters
\(\gamma^* = -i\phi'(s)\) and \(dG^*(x) = e^{isx}\,dG(x)\).
\end{proof}

We call \(X^*\) the \(K\)-transform of \(X\).

If \(X\) is standard normal, then \(\gamma = 0\), \(G = 1_{[0,\infty)}\)
and \(\phi(u) = -u^2/2\) so \(\gamma^* = is\) and \(dG^* = dG\).
We have \(e^{-s^2/2}Ee^{iu(is + X)} = e^{-s^2/2}e^{-su - u^2/2}
= e^{-(s + u)^2/2} = Ee^{isX}e^{iuX}\).

\begin{corollary}
If \(f\) and its Fourier transform are integrable, then
\(Ee^{isX}f(X) = Ee^{isX} Ef(X^*)\) where \(X^*\) is the
K-transform of \(X\).
\end{corollary}
\begin{proof}
If \(f\) and its Fourier transform are integrable, then
\(f(x) = \int_{-\infty}^\infty e^{iux}\hat{f}(u)\,du/2\pi\), where
\(\hat{f}(u) = \int_{-\infty}^\infty e^{-iux}f(x)\,dx\) is the
Fourier transform of \(f\).
\begin{align*}
E e^{isX}f(X) &= \int_{-\infty}^\infty E e^{iuX}e^{iuX} \hat{f}(u)\,du/2\pi\\
	&= Ee^{isX}\int_{-\infty}^\infty E e^{iuX^*} \hat{f}(u)\,du/2\pi\\
	&= Ee^{isX} E f(X^*)
\end{align*}
\end{proof}

If \((X_t)\) is Brownian motion, then the usual formula for the
Black-Scholes/Merton put price is \(E\max\{k - F_t, 0\}
= kN(-d_1) - fN(-d_2)\) where \(N\) is the standard normal
cumulative distribution, \(d_1 = (\log f/k + \sigma^2t/2)/\sigma\sqrt{t}\)
and \(d_2 = d_1 - \sigma\sqrt{t}\).
The next corollary generalizes this
to any L\'evy distribution.

\begin{corollary}
If \((X_t)\) is a L\'evy process and \(X_1\) has Kolomogorov
parameters \(\gamma\), \(G\),
then \(E[e^{\sigma X_t}f(X_t)] = E[e^{\sigma X_t}] E[f(X_t^*)]\) where
\((X_t^*)\) is L\'evy and \(X_t^*\) has Kolomogorov
parameters \(\gamma^* = -it\kappa'(\sigma)\)
and \(dG^*(x) = te^{\sigma x}\,dG(x)\).
\end{corollary}
\begin{proof}
Apply the lemma with \(s = -i\sigma\) using \(\phi_t(u) = t\phi(u)\).
\end{proof}

%For a put option we want \(f = 1_{(-\infty, z]}\), however this is
%not integrable. If we use \(f(x) = e^{\zeta x}1_{(-\infty, z]}(x)\)
%which is integrable when \(\zeta > 0\), as is its Fourier transform
%\(\hat{f}(u) = e^{(\zeta - iu)z}/(\zeta - iu)\) we get
%\(Ef(X_t) = \int_{-\infty}^\infty Ee^{iuX_t}
%e^{(\zeta - iu)z}/(\zeta - iu)\,du\) and
%\(dEf(X_t)/dz = \int_{-\infty}^\infty Ee^{iuX_t}
%e^{(\zeta - iu)z}\,du\).

\section{\(K\) Model}
We propose a model where \(X_1\) has the Kolomogorov parameters
\(\gamma = 0\) and 
\begin{equation*}
G(x) =
\begin{cases}
ae^{x/\alpha} & \text{if $x < 0$,}\\
1 - be^{-x/\beta} & \text{if $x > 0$.}
\end{cases}
\end{equation*}
so
\begin{equation*}
G'(x) =
\begin{cases}
a/\alpha e^{x/\alpha} & \text{if $x < 0$,}\\
(1 - a - b)\delta_0 & \text{if $x = 0$,}\\
b/\beta e^{-x/\beta} & \text{if $x > 0$}\\
\end{cases}
\end{equation*}
Note that this paramterization is preserved under the
\(K\)-transformation.
The characteristic function
of the \(K\) model has a closed form solution:
%\[
%\kappa(s) =
%\frac{a}{\alpha}(-s + k_\alpha(u))
%+(1 - a - b)s^2/2
%+\frac{b}{\beta}(s - k_{-\beta}(u)),
%\]
%where \(k_\alpha(s) = (1/\alpha + s)\log (1 + \alpha s)\).
%Note \(k_\alpha(s) = k_1(\alpha s)/\alpha\).
\[
\phi(u) =
\frac{a}{\alpha}(-iu + k_\alpha(u))
-(1 - a - b)u^2/2
+\frac{b}{\beta}(iu - k_{-\beta}(u)),
\]
where \(k_\alpha(u) = (1/\alpha + iu)\log (1 + i\alpha u)\).

%Next note that
%\(\Re e^{k_1(u)}
%= e^{-u\arctan u}\sqrt{1 + u^2} \cos(\arctan u + u\log\sqrt{1 + u^2})\).
%It turns out \(e^{k_1(u)}\) is well approximated by \(e^{iBu - Au^2}\)
%where \(A = 1 - 1/\sqrt{2}\) and \(B = \sqrt[4]{2}\).
%In particular the power series expansions agree to fourth order.
% < 5e-6 on [0,0.3], 1e-4 on [0,0.5] and .004 on [0,1], 0.011 on [0,1.5]

Recall that the density function of \(X_t\) is
\begin{align*}
f_t(x) &= \int_{-\infty}^\infty E e^{iuX_t} e^{-iux}\,du/2\pi \\
	&= \int_{-\infty}^\infty e^{t\phi(u) - iux}\,du/2\pi \\
	&= \int_{-\infty}^\infty e^{
		i(\frac{-at}{\alpha} + \frac{bt}{\beta} - x)u
		+ \frac{atk_1(\alpha u)}{\alpha^2}
		+ \frac{btk_1(-\beta u)}{\beta^2}
		- \frac{1}{2}(1 - a - b)u^2t
		}\,du/2\pi \\
\end{align*}
where we use \(k_\alpha(u) = k_1(\alpha u)/\alpha\).

It turns out that \(\Re e^{b k_1(a u)}\) is well approximated by
\(e^{-Au^2}\cos Bu\) where
\(A = \frac{1}{2}(a^2b + a^2b^2 - a^2\sqrt{b(b^3 + 2b - 1)})\) and
\(B = a\sqrt[4]{b^4 + 2b^2 - b}\).

\end{document}

% Ross - distribution
% Esscher, F. (1932). "On the Probability Function in the Collective Theory of Risk". Skandinavisk Aktuarietidskrift 15: 175–95.

