\documentclass[fleqn]{amsart}
\usepackage{amssymb,amsmath,amsthm,hyperref}
\def\Var{\mathop{\rm Var}}
\def\Cov{\mathop{\rm Cov}}
\def\ker{\mathop{\rm ker}}
\def\ran{\mathop{\rm ran}}
\newcommand{\R}{\mathbb{R}}
\newcommand{\F}{\mathcal{F}}
\newcommand{\G}{\mathcal{G}}
\newcommand{\B}{\mathcal{B}}

\newtheorem{definition}{Definition}[section]
%\newtheorem{theorem}[definition]{Theorem}
\newtheorem{theorem}{Theorem}[section]
\newtheorem{corollary}[theorem]{Corollary}
\newtheorem{lemma}[theorem]{Lemma}
\newtheorem{example}{Example}
\newtheorem{exercise}{Exercise}

\title{Finitely Additive Measures}
\author{Keith A. Lewis}
\email{kal@kalx.net}
\address{KALX, LLC \tt{\url{http://kalx.net}}}

\begin{document}
\maketitle

\begin{section}{Introduction}

Although finitely additive measures don't have as many nice convergence
properties as countably additive measures, they can still be useful.
They require less mathematical machinery so proofs using them can
be made simpler.

This note collects basic facts about finitely additive measures
with an emphasis on algebras having a finite number of
elements. The last section collects basic facts about
vector space duality. A reference for this material
is Dunford and Schwartz, {\em Linear Operators, Volume 1}.

%The main result is that the dual space of bounded functions
%on a set is the space of finitely additive measures on the set.
%Although the results for interchange of limits with the
%dual pairing require more assumptions than the countably
%additive case, 

%if $\Omega$ is a set and $B(\Omega)$ is
%the Banach algebra of bounded functions on $\Omega$, then the
%dual, $B(\Omega)^*=ba(\Omega)$, is the space of finitely additive
%measures on $\Omega$. Note the measures are definted for
%{\em all} subsets of $\Omega$. More generally, if $\F$ is an
%algebra of sets on $\Omega$ and $B(\Omega, \F)$ is the set of
%all bounded $\F$-measurable functions on $\Omega$, then
%$B(\Omega, \F)^*=ba(\Omega, \F)$, is the space of finitely
%additive measures on $\F$.

\end{section}

\begin{section}{Algebras of Sets}

An algebra, $\F$, on a set $\Omega$ is a collection of subsets of $\Omega$
that is closed under complements (with respect to $\Omega$) and unions. By
De Morgan's laws it is also closed under intersection.

A set $A\in\F$ is an {\em atom} of $\F$ if $B\in\F$ is a subset of $A$
implies $B$ is either $A$ or the empty set.

A {\em partition} of a set $\Omega$ is a collection of subsets of
$\Omega$ that are pairwise disjoint having union $\Omega$, i.e.,
$\{A_i\}$ is a partition of $\Omega$ if
$A_i \cap A_j = \emptyset$ for $i\not=j$ and
$\cup_i A_i = \Omega$.

\begin{lemma}
If $\F$ is an algebra having a finite number of elements, then every
element is a finite union of atoms.

\end{lemma}

\begin{proof}
We prove this by showing the atoms of $\F$ form a partition of $\Omega$.
For $\omega\in\Omega$, let $A_\omega$ be the intersection of all
sets in $\F$ containing $\omega$. Clearly $A_\omega$ is an atom
containing $\omega$. The sets $\{A_\omega\colon\omega\in\Omega\}$
form a partition. Their union is $\Omega$ and if $A_\omega\cap
A_{\omega'}\not=\emptyset$ then $A_{\omega} = A_{\omega'}$ since $A_\omega$
is an atom.

\end{proof}

Partitions are used to represent partial
information. Knowing which atom $\omega$ belongs to is intermediate
between not knowing $\omega$ at all, the partition corresponding
to $\Omega$, and knowing $\omega$ exactly, the partition
corresponding to the singletons of $\Omega$.

\begin{example}
Let $\Omega = [0, 1)$ and $\F_j$ be the algebra generated by
the partition $\{[i2^{-j}, (i + 1)2^{-j})\colon 0\le i < 2^j\}$.
Knowing which atom $\omega\in\Omega$ belongs to is equivalent
to knowing the first $j$ digits in the base 2 representation of $\omega$.
\end{example}

\end{section}

\begin{section}{Finitely Additive Measures}

Let $\F$ be an algebra of sets on $\Omega$. A {\em finitely additive
measure} is a function $\mu\colon\F\to\R$ such that $\mu(A\cup B) =
\mu(A) + \mu(B)$ when $A\cap B=\emptyset$. We denote the set of
all finitely additive measures having domain $\F$ by $ba(\Omega, \F)$.
This is a {\em Banach space} with {\em norm} $\|\mu\|
= \sum_{A} |\mu(A)|$ where the sum is over the atoms of $\F$.

\begin{exercise}
Show $\mu(A\cup B) = \mu(A) + \mu(B) - \mu(A\cap B)$ for any $A,B\in\F$.
\end{exercise}

\begin{exercise}
Show by induction that $\mu(A_1\cup \cdots\cup A_n) = \mu(A_1) + \cdots
+ \mu(A_n)$ for any pairwise disjoint $A_1,\dots,A_n\in\F$.
\end{exercise}

We say a function $f\colon\Omega\to\R$ is
$\F$-{\em measurable} if $f^{-1}(U)\in\F$ for any open set $U\subset\R$,
where $f^{-1}(U) = \{\omega\in\Omega\colon f(\omega)\in U\}$.

\begin{exercise}
The function $f\colon\Omega\to\R$ is $\F$-measurable if
$f^{-1}(F)\in\F$ for any closed set $F\subset\R$.

\end{exercise}

\begin{exercise}
If $\F$ is finite then being $\F$-measurable is equivalent to
being constant on the atoms of $\F$. (Hint: Consider
$f^{-1}(\{a\})$ for $a\in\R$.)

\end{exercise}

Let $B(\Omega,\F)$ be the space of bounded $\F$-measurable
functions on $\Omega$ with norm $\|f\| = \sup_{\omega\in\Omega}
|f(\omega)|$. We will not prove this in general, but
the dual is $B(\Omega,\F)^*=ba(\Omega,\F)$.

If $\F$ is finite, every linear functional on $B(\Omega,\F)$ corresponds
to a finitely additive measure. Let $\Lambda\colon B(\Omega,\F)\to\R$
be a linear functional. For any $A\in\F$, the characteristic function
$1_A$ is bounded and $\F$-measurable. Define $\lambda\colon\F\to\R$
by $\lambda(A) = \Lambda(1_A)$. Since $1_A + 1_B = 1_{A\cup B}$ if
$A$ and $B$ are disjoint, $\lambda$ is additive. 

Conversely, given a finitely additive measure $\lambda\colon\F\to\R$
define the linear functional $\Lambda\colon B(\Omega,\F)\to\R$
by $\Lambda(f) = \sum_A f(A)\lambda(A)$, where the sum is over
the atoms of $\F$. Since $f$ is constant on atoms, $f(A)$ is
well defined. We also write this as $\Lambda(f) = \int_\Omega f\,d\lambda$
= $\int_\Omega f(\omega)\,\lambda(d\omega)$
as is done for the infinite dimensional case.

\begin{exercise}
Show this correspondence is one-to-one and onto and that
$\|\Lambda\| = \|\lambda\|$.
\end{exercise}

Recall $\|\Lambda\| = \sup_{\|f\|\le1} |\Lambda(f)|$.

There is a natural way to multiply a measure, $\mu$, by a bounded
function $f$: $(f\mu)(A) = f(A)\mu(A)$ whenever $A$ is an atom.
More generally, $f\mu(B) = \sum_A f(A)\mu(A)$ where the sum
is over the atoms contained in $B$.

\begin{exercise}
Define the bounded linear operator $M_f\colon B(\Omega,\F)\to B(\Omega, \F)$
by $M_f g = fg$, where $f\in B(\Omega,\F)$.
Show $M_f^*\mu = f\mu$ as defined above.
\end{exercise}

\end{section}

\begin{section}{Random Variables}
If a measure, $P$, is positive and has total mass one we call it a {\em
probability measure} and write $E[X] = \int X\,dP$ for the {\em expected
value} of the {\em random variable} $X\in L^1(\Omega, P, \F)$.

Define $\phi\colon L^1(\Omega, P, \F)\to ba(\Omega,\F)$ by
$\phi(X) = XP$. This is an isometry because $\|\phi(X)\|
= |X|P(\Omega)
= \int_\Omega |X|\,dP = \|X\|_1$. The inverse map is the
Radon-Nikodym derivative with respect to $P$.

\end{section}

\begin{section}{Conditional Expectation}

Given partial information, $\G\subseteq\F$, the
{\em conditional expectation} of $X$ given $\G$ is the $\G$ measurable
random variable $E[X\mid\G]$ with the property $\int_B E[X\mid\G]\,dP =
\int_B X\,dP$ for all $B\in\G$.

\begin{exercise}
Show $E[X\mid\G](B) = \sum_{A\subseteq B} X(A)P(A)/\sum_{A\subseteq B}P(A)$
where the sum is over atoms of $\F$ contained in $B\in\G$.
\end{exercise}

Define $E_\G\colon ba(\Omega,\F) \to ba(\Omega, \G)$ bye
$E_\G\mu = \mu|_\G$, the restriction of the measure $\mu$ defined
on $\F$ to the subset $\G$.

The relationship between conditional expectation of random variables
and restriction of measures is given by the following.

\begin{exercise}
Let $\G$ be a subalgebra of $\F$.
Show $\phi(E[X|\G]) = E_\G XP$.
\end{exercise}

\end{section}

\begin{section}{Appendix: Duality}

This section collects basic facts about vector space duality.

Let $V$ be a vector space with a topology that makes scalar multiplication
and vector addition continuous (a {\em topological vector space}).
The {\em dual} of $V$ is $V^* =
\{v^*\colon V\to\R\}$ where $v^*$ is linear and continuous. If $\B(V,W)$
denotes the space of all linear, continuous transformations from the TVS $V$ to
the TVS $W$, then $V^* = \B(V,\R)$.  We will also use angle brackets to denote
the duality pairing: $v^*(v) = \langle v,v^* \rangle$.

\begin{subsection}{Examples}
\begin{itemize}
\item The dual of $\R^n$ can be identified with $\R^n$ with
the dual pairing $\langle f,g\rangle = \sum_j f_j g_j$.
\item If $\ell^p = \{(f_j)\colon \|f\|_p = (\sum_j |f_j|^p)^{1/p} < \infty\}$
then $(\ell^p)^* = \ell^q$ where $\frac{1}{p} + \frac{1}{q} = 1$,
$1\le p < \infty$ with
the dual pairing $\langle f,g\rangle = \sum_j f_j g_j$.
\item Let $\ell^\infty = \{(f_j)\colon \|f\|_\infty = \max_j |f_j| < \infty\}$.
We have $(\ell^1)^* = \ell^\infty$ and $(\ell^\infty)^*$ is
the set of all finitely additive measures on the index set
with the dual pairing $\langle f,\mu \rangle = \sum_j f_j \mu(\{j\})$.
\end{itemize}
\end{subsection}

\begin{exercise}
Show $\lim_{p\to\infty}\|x\|_p = \|x\|_\infty$.
\end{exercise}

The most basic fact about (nonzero) linear functionals is that their
kernel is a hyperplane.  Recall for any linear transformation, $T\colon
V\to W$, the kernel is defined by $\ker T = \{v\in V\colon Tv = 0\}$.

\begin{lemma}
The kernel of a linear functional is a hyperplane.
\end{lemma}

\begin{proof}
Suppose $v^*\in V^*$ is not zero, then there exists $v_0\in V$ such
that $\langle v_0,v^*\rangle \not=0$. For any $v\in V$ we have $v
= v - v_0\langle v,v^*\rangle/\langle v_0,v^*\rangle + v_0\langle
v,v^*\rangle/\langle v_0,v^*\rangle$.  Since $\langle v - v_0\langle
v,v^*\rangle/\langle v_0,v^*\rangle, v^*\rangle = \langle v, v^*\rangle -
\langle v_0,v^*\rangle \langle v,v^*\rangle/\langle v_0,v^*\rangle = 0$,
we have every vector in $V$ is a linear combination of a vector in the
kernel of $v^*$ and $v_0$. This shows the kernel is a hyperplane.

\end{proof}
The dual of a vector space might contain only the 0 linear functional.
(E.g., $L^p([0,1])$ with Lebesgue measure when $p < 1$.) If the topology
of a TVS, $V$, has a basis of convex sets, the Hahn-Banach theorem
implies that for every $v\in V$ there exists $v^*\in V^*$ with $\langle v,
v^*\rangle \not=0$.

\begin{exercise}
If a TVS, $V$, has a topological basis of convex sets,
show that if $v\not=w$, $v,w\in V$ there exists $f\in V^*$
such that $f(v)\not= f(w)$.
\end{exercise}

In this case we say there are enough linear functionals to
distinguish points.

There is a natural embedding of any TVS into its double dual,
$\iota\colon V\to V^{**}$, by $\langle v^*,\iota v\rangle
= \langle v, v^*\rangle$ for $v\in V$ and $v^*\in V^*$.

If $T\colon V\to W$ is a continuous linear transformation, define its
dual by $T^*\colon W^*\to V^*$ by $T^*w^*(v) = w^*(Tv)$ (or equivalently
$\langle Tv, w^*\rangle = \langle v, T^*w^*\rangle$) for $v\in V$ and
$w^*\in W^*$.

Recall the {\em kernel} of a nonzero linear transformation $T\colon V\to
W$ is $\ker T = \{v\in V\colon Tv = 0\}$ and the {\em range} is $\ran T =
\{Tv\colon v\in V\}$.

For $M\subset V$ define the {\em annihilator} $M^\perp = \{v^*\in
V^*\colon \langle m, v^*\rangle = 0, m\in M\}$. For $N\subset V^*$
define the {\em preannihilator} $^\perp N = \{v\in V\colon \langle v,
n^*\rangle = 0, n^*\in N\}$.

\begin{exercise}
For $T\in\B(V,W)$ show $(\ran T)^\perp = \ker T^*$ and $^\perp(\ran T^*)
= \ker T$.
\end{exercise}

\begin{exercise}
For $T\in\B(V,W)$ show $(\ker T)^\perp \supseteq \ran T^*$ and
$^\perp(\ker T^*)\supseteq \ran T$.

\end{exercise}

\end{section}

\end{document}
