\documentclass[12pt,letterpaper,fleqn]{report}
\usepackage[utf8]{inputenc}
\usepackage{amsmath}
\usepackage{amsfonts}
\usepackage{amssymb}
\usepackage{lmodern}
\usepackage{kpfonts}
\DeclareMathOperator{\sgn}{sgn}

\title{Cash Flow Framework}
\author{Keith A. Lewis}
\date{\today}
\begin{document}
\maketitle
This document describes a unified approach to describing all securities and
derivative instruments and their basic accounting. The term {\em instrument}
is used for both primary securities and their derivatives.

Mathematical notation for accurately describing the dependencies between
primary securities and their derivative is introduced and used for stocks, bonds, convertible
bonds, foreign exchange, commodities, options (including Bermudan and American),
and in fact all known instruments.

Section 1 defines transactions for recording historical exchanges between counterparties. 
Section 2 specifies {\em contracts} used to define derivatives securities. 
Section 3 introduces {\em models} for primary securities. 
Section 4 uses trades and positions to specify derivative securities
The last section describes
the interaction between models and valuation.

\section{Transaction}
A {\em transaction} is an exchange at a time between a {\em buyer}
and a {\em seller}',
\(\tau = \langle t;a,i,c;a',i',c'\rangle\),
where \(t\) is the time of the transaction,
\(a\) represents the {\em amount}, 
\(i\) the {\em instrument}, 
and \(c\) the {\em counterparty}. 
As far as mathematical models go, this is
quite accurate. Account statements and order blotters contain these items.
There are also commissions such as brokers fees where a better name for
seller might be {\em intermediary}.

Buyers play an active role whereas sellers play a passive role, at least
until after the initial transaction. Buyers are price takers and sellers
are price makers. Broker-dealers play both roles.

Transactions determine {\em profit and loss}. Given
transactions \(\{\tau_j\}\) define
\begin{align*}
A_t(i) &= \sum_{t_j = t, i_j = i} a_j
+ \sum_{t_j = t, i_j' = i} a_j'\\
B_t(i) &= \sum_{s \le t} A_s(i)\\
\end{align*}
where we assume the counterparties \(c_j = c\) and \(c_j' = c'\) are fixed,
something easy to undo.
The first quantity is the {\em account} of how much in instrument \(i\)
the buyer transacted at time \(t\). The second is the current {\em balance}
in instrument \(i\) at time \(t\).

\section{Contracts}
A {\em contract} specifies three processes \((C_t)\), \((C^<_t)\),
and \((C^>_t)\) that are non-zero at a finite number of times. The first is
the set of non-optional cash flows, the second is the set of buyer determined
cash flows and the third is the set of seller determined cash flows.

If the buyer holds amount \(a\) at time \(t\) they receive \(aC_t\) from
the counterparty. The buyer has the right to specify how much of
\(C^<_t\) they want depending on how much they hold, and the seller
can decide how much of \(C^>_t\) depending on the outstanding position.

Unlike transactions that simply record historical facts, contracts depend on
what occurs in the future. This is modeled as a {\em sample space} \(\Omega\).
The information available at time \(t\) is modeled as an {\em algebra} of
subsets of \(\Omega\). Subsets of \(\Omega\) are {\em events} and in the real
world people know if two events have both happened (set intersection), 
whether one or the other event happened (set union), and if an event didn't
happen (set complement with respect to \(\Omega\)). An algebra is just a collection
of subsets that is closed under these operations.

The processes are {\em adapted} to the algebras in that they can only depend
on the information available at the time of the cash flow.
Let \(\mathcal{A}_t\) be the algebra of information available
at time \(t\), and 
let \(\mathcal{A}^<_t\) and \(\mathcal{A}^>_t\) 
be sub-algebras indicating the information the
buyer and seller can decide, respectively, at time \(t\).
The mathematical definition of adapted is that \(\{C_t = c\}\) is
a set in the algebra \(\mathcal{A}_t\) for any number \(c\).

\subsection{Notes}
The buyer's algebra is not available to the seller, and vice versa. Modulo insider trading.

\section{Models}
A {\em model} of market prices is a function 
\(X\colon \mathbf{T}\times\mathbb{R}\times\mathbf{I}\times\mathbf{C}
\times\mathbf{I}\times\mathbf{C}\to\mathbb{R}\) where \(\mathbf{T}\) is the set of {\em trading times},
\(\mathbf{I}\) is the set of available
instruments, and \(\mathbf{C}\) is the set of counterparties. It represents the
notion that the buyer has the transaction
\(\langle t;a,i,c;-aX(t;a,i,c;i',c'),i',c'\rangle\) available at time \(t\).
Classical models assume 
\(X(t;a,i,c;i',c') = X(t,i)\) where \(i'\) is the base currency.
A simple model incorporating a bid-ask spread is
\(X(t;a,i,c;i',c') = X(t,i)(1 + \epsilon\sgn(a))\).

Fix up the sample space being implicit!!!

\section{Trades and Positions}
A buyer specifies a finite number of {\em trades} 
\(\Gamma_{t_k}\) and \(\Gamma^<_{u_k}\)
while a seller specifies \(\Gamma^>_{v_k}\). If we let
\(\Gamma_t = \Gamma_{t_k}1(t = t_k)\) and similarly
for the optional cash flows we can
define the {\em position} \(\Xi_t = \sum_{s < t}\Gamma_s
-\Gamma^<_s - \Gamma^>_s\).
Note \(\Xi\) is piecewise constant and left continuous. Positions
are not executed immediately.
Contracts can specify constraints on \(\Gamma^<\) and/or \(\Gamma^>\).
E.g., \(0\le \Gamma^<_t\le\Xi_t\) -- you can't exercise options you
don't own.
The account process is 
\[
A_t(i) = (\Xi_t - \Gamma^<_t - \Gamma^>_t)C(t) - \Gamma_t X_t + \Gamma^<_tC^<_t + \Gamma^>_tC^>_t
\]
You get paid the fixed coupons on your current position, pay for your new trades,
and receive option cash flows based on your and the issuers exercise choices. Note
this model does not make any assumptions about optimal exercise, it only tries to
faithfully represent the relevant real-world possibilities.

An {\em elementary trade} is a pair of transactions of the form
\(\langle t;\Gamma,i,c;-\Gamma X,i',c'\rangle\), 
\(\langle u;-\Gamma;i,c;\Gamma X',i',c'\rangle\)
where \(t<u\), \(X\) is the price of getting in, and \(X'\) is the price of
getting out. The buyer's profit/loss on the trade is \(B_u(i)\) in units
of \(i'\) at time \(u\). If there are no cash flows in the interval
then \(B_u(i) = \Gamma(X' - X)\).
Typically \(i'\) is the {\em base currency}.

{\em Marking to market} is done by specifying a model that can convert every
instrument to a base currency and trading out of every open position into
that currency. The formulas for the account and balance still hold if we
consider positions and prices to be vectors indexed by \(i\) and replace scalar
products with inner products.

A {\em closed out} trading strategy is one where \(\Xi_t\) is zero in all
indices except for the base currency after some point in time. 
We say {\em arbitrage} exists if there is a closed out trading strategy such that
\(A_0 > 0\) and \(A_t \ge 0\) for all \(t\).

\section{Bonds}
Schematically, a bond has a price \(P\) at time \(t_0\) and
pays coupon \(c\) at \(t_j\), \(0<j\le n\) plus principal 1 at \(t_n\).
In the notation above, \(X_0 = P\) and \(C_j = c\) for \(0<j<n\) and \(C_n = 1 + c\).

If the bond is callable at time \(u < t_n\) for price \(p\), the issuer
has that cash flow available. The seller's sample space is \(\Omega^> = \{u,\infty\}\)
representing calling at time \(u\) or not. The algebras associated
with that are \(\mathcal{A}_t = \{\{u,\infty\}\}\) if \(t < u\)
and \(\mathcal{A}_t = \{\{u\},\{\infty\}\}\) if \(t\ge u\). (Recall a finite
algebra is specified by a {\em partition} of the sample space consisting
of its {\em atoms}.) 
The non-optional cash flows are 
\(C_j = c1(t_j < \omega) + 1(j = n, t_n < \omega)\) and the seller determined
cash flows are
\(C^>_j = p1(t_j = \omega)\).

If the bond is putable at time \(v\) for price \(q\), the buyer
has that cash flow available. Their sample space is \(\Omega^< = \{v,\infty\}\)
representing returning the bond to the issuer at time \(v\) or not. The
algebras and cash flows are similar to the above.

More generally, bonds can have call and put schedules: \(\{u_k,p_k\}\) and
\(\{v_l,q_l\}\). The seller's sample space is \(\Omega^> = \{u_1,...,\infty\}\)
with algebras \(\mathcal{A}^>_t = \cup_{u_k<t}\{\{u_k\}\}\cup\{\{u_k\ge t\}\}\)
and similarly for \(\mathcal{A}^<_t\). The non-optional cash flows are
\(C_j = c1(t_j < \min\{\omega^<,\omega^>\}) + 1(j = n, t_n < \min\{\omega^<,\omega^>\})\),
the seller cash flows are \(C_j^> = p_j1(t_j = \omega^>)\) and the buyer cash
flows are \(C_j^< = q_j1(t_j = \omega^<)\).

\end{document}