\documentclass[11pt,letterpaper,fleqn]{article}
\usepackage[latin1]{inputenc}
\usepackage{amsmath}
\usepackage{amsfonts}
\usepackage{amssymb}

\newcommand{\R}{{\bf{R}}}

\author{Keith A. Lewis}
\title{The Black-Scholes/Merton PDE}
\begin{document}
\maketitle

In \cite{BlaSch1973}, equation (7), Fischer Black and Myron Scholes derived a partial
differential equation for the price of options that can be replicated using
dynamic hedging under certain assumptions. In \cite{Mer1973}, equation (33), Robert
Merton derives a similar.

Their model of prices is $X(t) = (R(t), S(t))$ where $dR/R = \rho\,dt$ 
and $dS/S = \mu\,dt + \sigma\,dB$ describe the bond and stock prices, and
$\rho$, $\mu$, and $\sigma$ are constant.
Black and Scholes also assume what they call ``ideal conditions'', e.g.,
European options having no dividends or transaction costs, whereas Merton considers
more general notions and extensions to their theory.

Ross picked up on Merton's notion of this being just the first step in a more
general theory. \cite{Ross}

Given any portfolio $\Xi(t) = (M(t), N(t))$
consisting of $M$ shares in the bond and $N$ shares in the stock, where
$M$ and $N$ are also It\=o diffusions, the value of the portfolio at time $t$ is 
$V(t) = \Xi(t)\cdot X(t) = M(t)R(t) + N(t)S(t)$. A portfolio is {\em self financing} if $d(MR + NS) = M dR + N dS$. The It\=o calculus says 
\[
d(MR + NS) = 
M\,dR + N\,dS + R\,dM + S\,dN + dM\,dR + dN\,dS
\]
so the self financing condition implies $0 = R\,dM + S\,dN + dM\,dR +
dN\,dS$.  The financial interpretation is that changing the position $M$
in the bond and $N$ in the stock that were put on at time $t$ to $M +
dM$ and $N + dN$ at time $t + dt$ costs $dM\,(R + dR) + dN\,(S + dS) = 0$.

Since sums and products of It\=o diffusions are It\=o diffustions, 
there exists a function $v\colon [0,\infty)\times\R^2\to\R$ such that
$V(t) = v(t, R(t), S(t))$. Similar functions, $m$ and $n$, exist for
$M$ and $N$.  By the It\=o formula applied to $V$ and the self financing
condition, 
\[
(v_t + \rho R v_r + \frac{1}{2} \sigma^2 S^2 v_{ss}) dt + v_s dS 
= M dR + N dS.
\]
Equating the $dS$ terms gives $N = v_s$. Using $M =
(V - v_s S)/R$, and equating the $dt$ terms yields 
\[
v_t + \rho R v_r + \frac{1}{2}\sigma^2 S^2 v_{ss} = \rho(V - v_s S),
\]
the Black-Scholes/Merton
partial differential equation.

From \cite{BlaSch1973} equation (7) is
\[v_t = \rho V - \rho S v_s - \frac{1}{2}\sigma^2S^2v_{ss}\]
and from \cite{Mer1973} equation (33) is
\[\frac{1}{2}\sigma^2S^2v_{ss} + \rho S v_s - v_t - \rho V = 0\]

This gives necessary conditions on $M$ and $N$. We show that
they also satisfy the self financing condition.

Since $n(t, r, s) = v_s(t, r, s)$,
\[
dN = (v_{st} + \rho R v_{sr} + \frac{1}{2} \sigma^2 S^2 v_{sss})\,dt
	+ v_{ss}\,dS.
\]
Using $m(t, r, s) = (v - v_s s)/r$, gives
\begin{align*}
dM &= \bigl((v_t - v_{st}S)/R + \rho(v_r - v_{sr}S - (v - v_s S)/R)\\
 &- \frac{1}{2}\sigma^2 S^2(v_{ss} + v_{sss}S)/R\bigr)\,dt - (v_{ss}S)/R\,dS.
\end{align*}

The $dS$ term of $R\,dM + dM\,dR + S\,dN + dN\,dS = R\,dM + S\,dN + dN\,dS$ is 
$-v_{ss}S + v_{ss}S + 0= 0$
and the $dt$ term is
\begin{align*}
&\bigl((v_t - v_{st}S) + \rho((v_r - v_{sr}S)R - (v - v_s S)) \\
&\quad -\frac{1}{2}\sigma^2 S^2(v_{ss} + v_{sss}S)\bigr)\\ 
&+ (v_{st} + \rho R v_{sr} + \frac{1}{2} \sigma^2 S^2 v_{sss})S\\
&+ \sigma^2 S^2 v_{ss}
\end{align*}

Canceling and combining terms yields
\[
v_t + \rho(v_r R - (v - v_s S))
+\frac{1}{2}\sigma^2 S^2v_{ss}.
\]
Since $v$ satisfies the Black-Scholes/Merton PDE, this
quantity is zero and hence $M$ and $N$ are
self financing.

%This gives necessary conditions on $M$ and $N$. We must show they
%are sufficient. Given $t$, $r$, and $s$, there exists a
%unique strong solution to $(dR, dS) = (\rho R, \mu S) dt
%+ (0, \sigma S)dB$ that satisfies the initial condition 
%$R(t) = r$ and $S(t) = s$ \cite{Oks}.
%For a European option that pays $\phi(S_T)$ at time $T$,
%define $V(t, r, s) = E^(t, r, s)[\phi(S_T)R_T/R_t | \F_t]$.
%The Kolomogorov backward equation \cite{Oks} states
%$V_t = \rho r V_r + \mu_s s V_s + \frac{1}{2} \sigma^2 s^2 V_ss$
%and $V(T, r, s) = \phi(s)$. Since $V_r = -V_r/r$ it follows ...
\bibliographystyle{plain}
\bibliography{bsm}{}
\end{document}
