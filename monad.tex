\documentclass[12pt,letterpaper,fleqn]{report}
\usepackage[latin1]{inputenc}
\usepackage{amsmath,amscd}
\usepackage{amsfonts}
\usepackage{amssymb}
\usepackage{lmodern}
\usepackage{kpfonts}
\usepackage{bm}
\usepackage[mathscr]{euscript}

\author{Keith A. Lewis}
\title{Programs and Monads}
\begin{document}
\maketitle
The typed \(\lambda\)-calculus is a simple language for expressing computer
programs. A program takes a type, performs a calculation, and returns a
type as the result. Cartesian monoidal categories provide a model of
the typed \(\lambda\)-calculus. A short description is that there is an associative 
bifunctor \(\times\)
with adjoint \(\Rightarrow\).
In particular, \(\mathrm{Hom}(A\times B,C)\) is isomorphic to 
\(\mathrm{Hom}(A,B\Rightarrow C)\) that is natural in \(A\) and \(C\)	.
There is a terminal object \(1\) such that \(1\times A = A\times 1\), and
various isomorphisms \(A\times B\cong B\times A\), \(A\times(B\times C)
\cong (A\times B)\times C\).

A {\em monad}, \(M\), is an endofunctor on a category together with two
{\em natural transformations} \(\nu\colon I\to M\) ({\em unit}) 
and \(\mu\colon M^2\to M\) ({\em join}),
where \(I\) is the identity functor, for which the following diagrams commute
\[
\begin{CD}
M^3 @>{M\mu}>>M^2\\
@V{\mu M}VV	@VV\mu V\\
M^2 @>>\mu> M
\end{CD}
\qquad
\begin{CD}
M @>{M\nu}>>M^2\\
@V{\nu M}VV	@VV\mu V\\
M^2 @>>\mu> M
\end{CD}
\]
and \(\mu(M\nu) = \mu(\nu M)\). Recall \((M\mu)_x = M(\mu_x)\) and
\((\mu M)_x = \mu_{M x}\).

\subsection{List monad}
A list of type \(X\) a is an object of the type \([X] = X\times (X\times(\cdots (X\times 1)))\).
If \(f\colon X\to Y\) define \([f]\colon [X]\to [Y]\) by
\(f = f\times (f\times(\cdots (f\times 1)))\).
The {\em head}, or {\em car} in lisp speak, is the projection on the first term
and the {\em tail}, or {\em cdr} in lisp speak, is the projection on the second term.
If \(f\colon X\to Y\) define \(\mathrm{fmap} f\colon [X]\to [Y]\)
by \(\mathrm{head}((\mathrm{fmap} f)([X]) = f(\mathrm{head}[X])\)
and \(\mathrm{tail}((\mathrm{fmap} f)([X]) = (\mathrm{fmap} f)(\mathrm{tail}[X])\).

Or \((\mathrm{fmap} f)(X\times [X]) = f(X)\times (\mathrm{fmap} f)([X])\).

\begin{align*}
	\nu\colon& t\to M t\\
	\mu\colon& (t\to u)\to (M t\to Mu).
\end{align*}
The function \(\nu\) is called {\em unit} and it ``wraps up'' its argument.
The function \(\mu\) is called {\em lift} and it takes a function to
a function that acts on wrapped up arguments.
They satisfy the rule \((\mu f)\nu = \nu f\).

A more general function than \(\mu\) is {\em bind}
\[
	\beta\colon \nu . 
\]

It can be used to define lift by
\((\mu f)\nu = \beta \nu f\).
%by \((\mathrm{lift} f) M t = \mathrm{bind} (M t) (\lambda x. f(x)\).
\end{document}